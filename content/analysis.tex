Our initial focus was on the inclusion of a minimum number of entries per buffer and the maximum basket buffer limit. 
As we'll see in Section \ref{sec:result}, we then opted to keep the minimum number of entries set to its default setting (minimum 10 entries per buffer).

For both the nightly and the release testing, the data derivation job comes from a 2022 dataset with four input files with $160327$ events. 
The MC job comes from a 2023 $t\bar{t}$ standard sample simulation job with six input files with $140000$ events. 
The specific datasets for both are noted in Appendix \ref{sec: deriv job dataset}.


The corresponding input files for both data and MC jobs were ran with the various configurations of Athena v24.0.16 and its specified basket buffer limit.
The four configurations tested all kept minimum 10 entries per basket and modified the basket limitation in the following ways: 

\begin{enumerate}
    \item ``$\textit{default}$" - Athena's default setting, and basket limit of $128\times1024$ bytes
    \item ``$\textit{no-lim}$'' - Removing the basket limit completely
    \item ``$\textit{256k}$'' - Limit basket buffer to $256\times1024$ bytes
    \item ``$\textit{512k}$'' - Limit basket buffer to $512\times1024$ bytes
\end{enumerate}

Interesting results come from the comparison of the \textit{no-lim} and the \textit{default} configurations. 
The \textit{256k} and \textit{512k} configurations were included for completeness and proved to be a helpful sanity check throughout. 
Building and running these configurations of Athena are illustrated in a GitHub repositiory. \cite{Kraus}
