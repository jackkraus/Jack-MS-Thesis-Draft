% Results
% The progression of the memory usage throughout the job can be shown by plotting the proportional set size (Figure \ref{fig: mc-prmon-pss} and Figure \ref{fig: data-prmon-pss}), resident set size (Figure \ref{fig: mc-prmon-rss} and Figure \ref{fig: data-prmon-rss}), and the virtual memory versus the wall time (Figure \ref{fig: mc-prmon-vmem} and Figure \ref{fig: data-prmon-vmem}.) 

% % Graphs: Bad News
% In Figure \ref{fig: mc-prmon-pss}, Athena \textit{default} uses less memory than any of the basket buffer limit configurations.
% There increased memory usage is about $12.7\%$ in MC proportional set size or as little as  $2.8\%$ for Data as seen in Figure \ref{fig: data-prmon-pss}, see Appendix A. 
% A similar trend persists for the other figures, where MC jobs are taking the greatest hit in memory usage for Figure \ref{fig: mc-prmon-rss} and Figure \ref{fig: mc-prmon-vmem}, when compared to the data derivation jobs in Figure \ref{fig: data-prmon-rss} and Figure \ref{fig: data-prmon-vmem}. 

\subsection{Athena v22.0.16 testing}
The results of this run of testing was done on a nightly build of Athena that is based on version 22.0.16 and looked at both the presence of an upper limit to the basket size (a ``basket-cap") and the presence of a minimum number of entries per branch, the default setting is 10 entries.
PHYSLITE MC derivation production, from Table \ref{Tab: night-mc-outputfs}, sees a 9.9\% increase in output file size when compared to the $\textit{default}$ Athena configuration, and since this configuration only differs by the elimination of the ``min-number-entries" we assume the minimum number of entries per branch should be kept at 10 and left alone.
Table \ref{Tab: night-mc-outputfs} also shows the potential for PHYSLITE MC DAOD output file size reduction by elminiating our ``basket-cap'' (basket buffer limit) altogether.

% Data file sizes 
% Athena version: Nightly from v22.0.16 
% input file: # TODO
% PHYS/PHYSLITE

\begin{table}[!ht]
    \centering
    \resizebox{\textwidth}{!}{
        \begin{tabular}{|c|c|c|c|}
            \hline
                Presence of features (Data) & Max PSS (MB) ($\Delta$\% default) & PHYS outFS (GB) ($\Delta$\%) & PHYSLITE outFS (GB) ($\Delta$\%) \\ \hline
                basket-cap, min-num-entries (default)  & 27.1 ( + 0.0 \%) & 3.22 ( + 0.0 \%) & 1.03 ( + 0.0 \%) \\ \hline
                \st{basket-cap} \st{min-num-entries} & 27.8 ( + 2.5 \%) & 3.22 ( + 0.2 \%) & 1.04 ( + 0.2 \%) \\ \hline
                \st{basket-cap} min-num-entries & 27.8  ( + 2.5 \%) & 3.22 ( - 0.0 \%) & 1.03 ( - 0.4 \%) \\ \hline
                basket-cap,  \st{min-num-entries} & 27.3 ( + 0.7 \%) & 3.22 ( + 0.2 \%) & 1.04 ( + 0.7 \%) \\ \hline
            \end{tabular}
    }
    \label{Tab: night-data-outputfs}
    \caption{Comparing the maximum proportional set size (PSS) and PHYS/PHYSLITE output file sizes (outFS) for data jobs while varying the presence of features in Athena PoolWriteConfig.py for 160327 entries.}
\end{table}

% Data file sizes 
% Athena version: Nightly v22.0.16 
% input file: # TODO
% PHYS/PHYSLITE: PHYSLITE

% \begin{table}[!ht]
%     \centering
%     \resizebox{\textwidth}{!}{
%         \begin{tabular}{|l|l|l|l|l|l|}
%             \hline
%                 Athena configurations - PHYS-LITE (data) - File sizes & Max PSS (MB)  & \% of MaxPSS from default & nentries & Output File Size (GB) & \% of output from default \\ \hline
%                 (default) With basket-cap and min-num-entries & 27.109 & 0.00\% & 160327 & 1.034 & 0.00\% \\ \hline
%                 Without both basket-cap and min-num-entries & 27.813 & 2.53\% & 160327 & 1.036 & 0.21\% \\ \hline
%                 No basket-cap but with min-num-entries & 27.814 & 2.53\% & 160327 & 1.030 & -0.39\% \\ \hline
%                 With basket-cap but without min-num-entries & 27.298 & 0.69\% & 160327 & 1.042 & 0.71\% \\ \hline
%             \end{tabular}
%     }
%     \caption{Athena v22.0.16: Comparing the difference in PHYSLITE derivation jobs and their Athena configurations and results of output file size}
%     \label{Tab: night-data-physlite-outputfs}
% \end{table}

% -------------------------------------------------------------------------------------------------------------------------------



% mc file sizes 
% Athena version: Nightly v22.0.16 
% input file: # TODO
% PHYS/PHYSLITE: PHYS
\begin{center}
    \begin{table}[!ht]
        \centering
        \resizebox{\textwidth}{!}{
            \begin{tabular}{|c|c|c|c|}
                \hline
                Presence of features (MC) & Max PSS (MB) ($\Delta$\% default) & PHYS outFS (GB) ($\Delta$\%) & PHYSLITE outFS (GB) ($\Delta$\%) \\ \hline
                basket-cap, min-num-entries (default)  & 14.1 ( + 0.0 \%) & 5.8 ( + 0.0 \%) & 2.6 ( + 0.0 \%) \\ \hline
                \st{basket-cap} \st{min-num-entries} & 16.1 ( + 12.1 \%) & 6.0 ( + 2.9 \%) & 2.7 ( + 5.1 \%) \\ \hline
                \st{basket-cap} min-num-entries & 16.0  ( + 11.5 \%) & 5.7 ( - 2.8 \%) & 2.5 ( - 5.6 \%) \\ \hline
                basket-cap,  \st{min-num-entries} & 14.2 ( + 0.4 \%) & 6.2 ( + 5.4 \%) & 2.9 ( + 9.9 \%) \\ \hline
                \end{tabular}
        }
        \caption{Comparing the maximum proportional set size (PSS) and PHYS/PHYSLITE output file sizes (outFS) for MC jobs while varying the presence of features in Athena PoolWriteConfig.py for 140000 entries.}
        \label{Tab: night-mc-outputfs}
    \end{table}
\end{center}

% mc file sizes 
% Athena version: Nightly v22.0.16 
% input file: # TODO
% PHYS/PHYSLITE: PHYS

% \begin{center}
%     \begin{table}[!ht]
%         \centering
%         \resizebox{\textwidth}{!}{
%             \begin{tabular}{|l|l|l|l|l|l|}
%                 \hline
%                     Athena configurations - PHYS-LITE (mc) - File sizes & Max PSS (MB) & \% of MaxPSS from default & nentries & Output File Size (GB) & \% of outputfs from default \\ \hline
%                     (default) With basket-cap and min-num-entries & 14.13 & 0.00\% & 160327 & 2.59 & 0.00\% \\ \hline
%                     Without both basket-cap and min-num-entries & 16.08 & 12.13\% & 160327 & 2.72 & 5.06\% \\ \hline
%                     No basket-cap but with min-num-entries & 15.97 & 11.51\% & 160327 & 2.45 & -5.58\% \\ \hline
%                     With basket-cap but without min-num-entries & 14.19 & 0.42\% & 160327 & 2.87 & 9.90\% \\ \hline
%                 \end{tabular}
%         }
%         \caption{Athena v22.0.16: Comparing the difference in PHYSLITE derivation jobs and their Athena configurations and results of output file size}
%         \label{Tab: night-mc-physlite-outputfs}
%     \end{table}
% \end{center}


\subsection{Athena v24.0.16 testing}
Switching from the nightly build to the stable Athena version 24.0.16, pre-existing derivation jobs were ran for Data and MC to compare between configurations of differing basket size limits.
The results for this set of testing are found from Table \ref{Tab: 24.0.16-data-outputfs} through Table \ref{Tab: 24.0.16-MC-PHYSLITE-no-lim-top-compfs}. 
The following tables are the DAOD output-file sizes of the various Athena configurations for PHYS/PHYSLITE over their respective Data/MC AOD input files.

% data-24.0.16 file sizes
% Athena version 24.0.16
% input file: # TODO
% PHYS/PHYSLITE 

\begin{table}[!ht]
    \centering
    \resizebox{\textwidth}{!}{
    \begin{tabular}{|c|c|c|c|}
    \hline
        Athena Configs (Data) & Max PSS (MB) ($\Delta$\% \text{default}) & PHYS outFS (GB) ($\Delta$\% ) & PHYSLITE outFS (GB) ($\Delta$\% ) \\ \hline
        (default)  & 27.9 ( + 0.0 \%) & 3.3 ( + 0.0 \%) & 1.0 ( + 0.0 \%) \\ \hline
        256k\_basket & 28.2  ( + 1.3 \%) & 3.3 ( - 0.1 \%) & 1.0 ( - 0.3 \%) \\ \hline
        512k\_basket & 28.5 ( + 2.2 \%) & 3.3 ( + 0.0 \%) & 1.0 ( - 0.3 \%) \\ \hline
        1.3 MB (ROOT MAX) & 28.6 ( + 2.7 \%) & 3.3 ( - 0.1 \%) & 1.0 ( - 0.3 \%) \\ \hline
    \end{tabular}
    }
    \caption{Comparing the maximum proportional set size (PSS) and PHYS/PHYSLITE output file sizes (outFS) for Data jobs over various Athena configurations for 160327 entries.}
    \label{Tab: 24.0.16-data-outputfs}
\end{table}

% data-24.0.16 file sizes
% Athena version 24.0.16
% input file: # TODO
% PHYS/PHYSLITE: PHYSLITE


% \begin{table}[!ht]
%     \centering
%     \resizebox{\textwidth}{!}{
%         \begin{tabular}{|l|l|l|l|l|l|}
%             \hline
%                 Athena configurations - PHYS-LITE (data) - File sizes & Max PSS (MB) & \% of MaxPSS from default & nentries & Output File Size (GB) & \% of outputfs from default \\ \hline
%                 data-test-athena-default & 27.8591 & 0.00\% & 160327 & 1.0334 & 0.00\% \\ \hline
%                 data-test-athena-no-limit & 28.6432 & 2.74\% & 160327 & 1.0302 & -0.31\% \\ \hline
%                 data-test-athena-256k-basket & 28.2166 & 1.27\% & 160327 & 1.0303 & -0.30\% \\ \hline
%                 data-test-athena-512k-basket & 28.4852 & 2.20\% & 160327 & 1.0307 & -0.26\% \\ \hline
%             \end{tabular}
%     }
%     \caption{Athena v24.0.16 Data PHYSLITE output file size}
%     \label{Tab: 24.0.16-data-physlite-outputfs}
% \end{table}


% -------------------------------------------------------------------------------------------------------------------------------


% mc-24.0.16 file sizes
% Athena version 24.0.16
% input file: # TODO
% PHYS/PHYSLITE 


\begin{table}[!ht]
    \centering
    \resizebox{\textwidth}{!}{
        \begin{tabular}{|c|c|c|c|}
            \hline
            Athena configurations (Data) & Max PSS (MB) ($\Delta$\% \text{default}) & PHYS outFS (GB) ($\Delta$\% ) & PHYSLITE outFS (GB) ($\Delta$\% ) \\ \hline
            (default) & 15.00 ( + 0.00 \%) & 5.88 ( + 0.00 \%) & 2.59 ( + 0.00 \%) \\ \hline
            no\_limit & 16.90 ( + 11.27 \%) & 5.72 ( - 2.80 \%) & 2.45 ( - 5.55 \%) \\ \hline
            256k\_basket & 15.28  ( + 1.87 \%) & 5.80 ( - 1.35 \%) & 2.51 ( - 3.11 \%) \\ \hline
            512k\_basket & 16.41 ( + 8.60 \%) & 5.74 ( - 2.46 \%) & 2.46 ( - 5.11 \%) \\ \hline
            \end{tabular}
    }
    \caption{Athena v24.0.16: Comparing the maximum proportional set size (PSS) and PHYS/PHYSLITE output file sizes (outFS) for MC jobs over various Athena configurations for 140000 entries.}
    \label{Tab: 24.0.16-mc-outputfs}
\end{table}

% mc-24.0.16 file sizes
% Athena version 24.0.16
% input file: # TODO
% PHYS/PHYSLITE: PHYSLITE


% \begin{table}[!ht]
%     \centering
%     \resizebox{\textwidth}{!}{
%         \begin{tabular}{|l|l|l|l|l|l|}
%             \hline
%                 Athena configurations - PHYS-LITE (MC) - File sizes & Max PSS (MB) & \% of MaxPSS from default & nentries & Output File Size (GB) & \% of outputfs from default \\ \hline
%                 mc-test-athena-default & 15.00 & 0.00\% & 140000 & 2.59 & 0.00\% \\ \hline
%                 mc-test-athena-no-limit & 16.90 & 11.27\% & 140000 & 2.45 & -5.55\% \\ \hline
%                 mc-test-athena-256k-basket & 15.28 & 1.87\% & 140000 & 2.51 & -3.11\% \\ \hline
%                 mc-test-athena-512k-basket & 16.41 & 8.60\% & 140000 & 2.46 & -5.11\% \\ \hline
%             \end{tabular}
%     }
%      \caption{Athena v24.0.16 MC PHYSLITE output file size}
%     \label{Tab: 24.0.16-mc-physlite-outputfs}
% \end{table}


% -------------------------------------------------------------------------------------------------------------------------------


\subsection{Highly compressible branches}
% =====================
% ====== TOP TEN ======
% =====================

Derivation production will produce branches that are initially very large and occupy memory and compress that down to a smaller size.
One thing that we tend to want to avoid is having too much repetition resulting in high compression factors.
If we have many highly compressible branches, more memory will be used, specifically within I/O buffers.
The following tables are an investigation into the highly compressible branches (e.g., the branches with compression factor over $\mathcal{O}(1)$) and might possibly lead to some insight on the increase in memory among the decrease in disk space.

% The tables start with Data PHYS, then move to Data PHYSLITE, then MC PHYS, and lastly MC PHYSLITE. For each, shown are the $\textit{default}$ Athena configuration and the $\textit{no-lim}$ configuration. 
% % ======= DATA PHYS =======
% % data-24.0.16 Top branches sorted by CF 
% Athena version 24.0.16, 
% ========= default configuration ========
% input file: # TODO
% PHYS/PHYSLITE: PHYS

\begin{table}[!ht]
    \centering
    \resizebox{\textwidth}{!}{
        \begin{tabular}{|l|l|l|l|l|}
            \hline
                Athena v24.0.16 (default) Data PHYS branch & number-of-entries & total-file-size (bytes) & compressed-file-size (bytes) & compression factor \\ \hline
                PrimaryVerticesAuxDyn.trackParticleLinks & 160327 & 2251059204 & 23882568 & 94.26 \\ \hline
                PhotonsAuxDyn.caloClusterLinks & 160327 & 91495309 & 1894438 & 48.30 \\ \hline
                ElectronsAuxDyn.caloClusterLinks & 160327 & 49019260 & 1132973 & 43.27 \\ \hline
                TauJetsAuxDyn.tauTrackLinks & 160327 & 228764558 & 5685495 & 40.24 \\ \hline
                AntiKt4EMTopoJetsAuxDyn.GhostTrack & 160327 & 496823421 & 13152863 & 37.77 \\ \hline
                xTrigDecisionAux. & 160327 & 177976859 & 4895024 & 36.36 \\ \hline
                AntiKt4EMPFlowJetsAuxDyn.NumChargedPFOPt1000 & 160327 & 135741017 & 3806096 & 35.66 \\ \hline
                AntiKt4EMPFlowJetsAuxDyn.NumChargedPFOPt500 & 160327 & 135739566 & 3932073 & 34.52 \\ \hline
                InDetTrackParticlesAuxDyn.TTVA\_AMVFVertices & 160327 & 76347222 & 2493804 & 30.61 \\ \hline
                AntiKt4EMPFlowJetsAuxDyn.GhostTrack & 160327 & 323552826 & 10644950 & 30.39 \\ \hline
            \end{tabular}
    }
    \caption{Top 10 branches sorted by compression factor, Data PHYS [Athena v24.0.16 default configuration.]}
    \label{Tab: 24.0.16-data-PHYS-default-top-CF}
\end{table}

% data-24.0.16 Top branches sorted by CF
% Athena version 24.0.16, 
% ========== no-lim configuration ===========
% input file: # TODO
% PHYS/PHYSLITE: PHYS

\begin{table}[!ht]
    \centering
    \resizebox{\textwidth}{!}{
        \begin{tabular}{|l|l|l|l|l|}
            \hline
                Athena v24.0.16 (no-lim) Data PHYS branch & number-of-entries & total-file-size (bytes) & compressed-file-size (bytes) & compression factor \\ \hline
                PrimaryVerticesAuxDyn.trackParticleLinks & 160327 & 2250366454 & 22798293 & 98.71 \\ \hline
                PhotonsAuxDyn.caloClusterLinks & 160327 & 91425389 & 1634442 & 55.94 \\ \hline
                ElectronsAuxDyn.caloClusterLinks & 160327 & 48984043 & 1020564 & 48.00 \\ \hline
                TauJetsAuxDyn.tauTrackLinks & 160327 & 228626126 & 5514502 & 41.46 \\ \hline
                xTrigDecisionAux. & 160327 & 177879857 & 4501170 & 39.52 \\ \hline
                AntiKt4EMPFlowJetsAuxDyn.NumChargedPFOPt1000 & 160327 & 135667100 & 3475099 & 39.04 \\ \hline
                AntiKt4EMTopoJetsAuxDyn.GhostTrack & 160327 & 496528896 & 12973377 & 38.27 \\ \hline
                AntiKt4EMPFlowJetsAuxDyn.NumChargedPFOPt500 & 160327 & 135666222 & 3591140 & 37.78 \\ \hline
                InDetTrackParticlesAuxDyn.TTVA\_AMVFVertices & 160327 & 76308054 & 2370985 & 32.18 \\ \hline
                AntiKt4EMPFlowJetsAuxDyn.GhostTrack & 160327 & 323350266 & 10503879 & 30.78 \\ \hline
            \end{tabular}
    }
    \caption{Top 10 branches sorted by compression factor, Data PHYS [Athena v24.0.16 without limit to the basket buffer.]}
    \label{Tab: 24.0.16-data-PHYS-no-lim-top-CF}
\end{table}

% % Data-24.0.16 Top branches sorted by totalfs
% Athena version 24.0.16, default configuration
% input file:
% PHYS/PHYSLITE: PHYS

\begin{table}[!ht]
    \centering
    \resizebox{\textwidth}{!}{
        \begin{tabular}{|l|l|l|l|l|}
            \hline
                Athena v24.0.16 (default) Data PHYS branch & number-of-entries & total-file-size (bytes) & compressed-file-size (bytes) & compression factor \\ \hline
                PrimaryVerticesAuxDyn.trackParticleLinks & 160327 & 2251059204 & 23882568 & 94.26 \\ \hline
                AntiKt10LCTopoJetsAux. & 160327 & 935703679 & 70232305 & 13.32 \\ \hline
                AntiKt4EMTopoJetsAuxDyn.GhostTrack & 160327 & 496823421 & 13152863 & 37.77 \\ \hline
                AntiKt4EMPFlowJetsAuxDyn.GhostTrack & 160327 & 323552826 & 10644950 & 30.39 \\ \hline
                AntiKt10UFOCSSKJetsAux. & 160327 & 254528416 & 20299079 & 12.54 \\ \hline
                TauJetsAuxDyn.tauTrackLinks & 160327 & 228764558 & 5685495 & 40.24 \\ \hline
                AntiKt4EMTopoJetsAuxDyn.NumTrkPt1000 & 160327 & 213203749 & 12248572 & 17.41 \\ \hline
                AntiKt4EMTopoJetsAuxDyn.NumTrkPt500 & 160327 & 213201702 & 16729947 & 12.74 \\ \hline
                xTrigDecisionAux. & 160327 & 177976859 & 4895024 & 36.36 \\ \hline
                egammaClustersAuxDyn.constituentClusterLinks & 160327 & 157257667 & 9692217 & 16.23 \\ \hline
            \end{tabular}
    }
    \caption{Top 10 branches sorted by Total File Size, Data PHYS [Athena v24.0.16 default configuration.]}
    \label{Tab: 24.0.16-Data-PHYS-default-top-totalfs}
\end{table}

% Data-24.0.16 Top branches sorted by totalfs
% Athena version 24.0.16, no-lim configuration
% input file: 
% PHYS/PHYSLITE: PHYS

\begin{table}[!ht]
    \centering
    \resizebox{\textwidth}{!}{
        \begin{tabular}{|l|l|l|l|l|}
            \hline
                Athena v24.0.16 (no-lim) Data PHYS branch & number-of-entries & total-file-size (bytes) & compressed-file-size (bytes) & compression factor \\ \hline
                PrimaryVerticesAuxDyn.trackParticleLinks & 160327 & 2250366454 & 22798293 & 98.71 \\ \hline
                AntiKt10LCTopoJetsAux. & 160327 & 935195108 & 68645038 & 13.62 \\ \hline
                AntiKt4EMTopoJetsAuxDyn.GhostTrack & 160327 & 496528896 & 12973377 & 38.27 \\ \hline
                AntiKt4EMPFlowJetsAuxDyn.GhostTrack & 160327 & 323350266 & 10503879 & 30.78 \\ \hline
                AntiKt10UFOCSSKJetsAux. & 160327 & 254392444 & 18988430 & 13.40 \\ \hline
                TauJetsAuxDyn.tauTrackLinks & 160327 & 228626126 & 5514502 & 41.46 \\ \hline
                AntiKt4EMTopoJetsAuxDyn.NumTrkPt1000 & 160327 & 213074279 & 11985880 & 17.78 \\ \hline
                AntiKt4EMTopoJetsAuxDyn.NumTrkPt500 & 160327 & 213073302 & 16642791 & 12.80 \\ \hline
                xTrigDecisionAux. & 160327 & 177879857 & 4501170 & 39.52 \\ \hline
                egammaClustersAuxDyn.constituentClusterLinks & 160327 & 157165174 & 9461105 & 16.61 \\ \hline
                TauJetsAuxDyn.tauTrackLinks & 160327 & 228626126 & 5514502 & 41.46 \\ \hline
            \end{tabular}
    }
    \caption{Top 10 branches sorted by total file size in bytes for Data PHYS [Athena v24.0.16  without limit to the basket buffer.]}
    \label{Tab: 24.0.16-Data-PHYS-no-lim-top-totalfs}
\end{table}

% % Data-24.0.16 Top branches sorted by compfs
% Athena version 24.0.16, default configuration
% input file:
% PHYS/PHYSLITE: PHYS

\begin{table}[!ht]
    \centering
    \resizebox{\textwidth}{!}{
        \begin{tabular}{|l|l|l|l|l|}
            \hline
                Athena v24.0.16 (default) Data PHYS branch & number-of-entries & total-file-size (bytes) & compressed-file-size (bytes) & compression factor \\ \hline
                AntiKt10LCTopoJetsAux. & 160327 & 935703679 & 70232305 & 13.32 \\ \hline
                PrimaryVerticesAuxDyn.trackParticleLinks & 160327 & 2251059204 & 23882568 & 94.26 \\ \hline
                AntiKt10UFOCSSKJetsAux. & 160327 & 254528416 & 20299079 & 12.54 \\ \hline
                AntiKt4EMTopoJetsAuxDyn.NumTrkPt500 & 160327 & 213201702 & 16729947 & 12.74 \\ \hline
                AntiKt4EMTopoJetsAuxDyn.GhostTrack & 160327 & 496823421 & 13152863 & 37.77 \\ \hline
                AntiKt4EMTopoJetsAuxDyn.NumTrkPt1000 & 160327 & 213203749 & 12248572 & 17.41 \\ \hline
                AntiKt4EMPFlowJetsAuxDyn.GhostTrack & 160327 & 323552826 & 10644950 & 30.39 \\ \hline
                AntiKt10UFOCSSKSoftDropBeta100Zcut10JetsAux. & 160327 & 127532256 & 10616786 & 12.01 \\ \hline
                AntiKt4EMPFlowJetsAuxDyn.NumTrkPt500 & 160327 & 135729288 & 10252068 & 13.24 \\ \hline
                AntiKt10LCTopoTrimmedPtFrac5SmallR20JetsAux. & 160327 & 101082000 & 9760437 & 10.36 \\ \hline
            \end{tabular}
    }
    \caption{Top 10 branches sorted by compressed file size in bytes, Data PHYS [Athena v24.0.16 default configuration.]}
    \label{Tab: 24.0.16-Data-PHYS-default-top-compfs}
\end{table}

% Data-24.0.16 Top branches sorted by compfs
% Athena version 24.0.16, no-lim configuration
% input file: 
% PHYS/PHYSLITE: PHYS

\begin{table}[!ht]
    \centering
    \resizebox{\textwidth}{!}{
        \begin{tabular}{|l|l|l|l|l|}
            \hline
                Athena v24.0.16 (no-lim) Data PHYS branch & number-of-entries & total-file-size (bytes) & compressed-file-size (bytes) & compression factor \\ \hline
                AntiKt10LCTopoJetsAux. & 160327 & 935195108 & 68645038 & 13.62 \\ \hline
                PrimaryVerticesAuxDyn.trackParticleLinks & 160327 & 2250366454 & 22798293 & 98.71 \\ \hline
                AntiKt10UFOCSSKJetsAux. & 160327 & 254392444 & 18988430 & 13.40 \\ \hline
                AntiKt4EMTopoJetsAuxDyn.NumTrkPt500 & 160327 & 213073302 & 16642791 & 12.80 \\ \hline
                AntiKt4EMTopoJetsAuxDyn.GhostTrack & 160327 & 496528896 & 12973377 & 38.27 \\ \hline
                AntiKt4EMTopoJetsAuxDyn.NumTrkPt1000 & 160327 & 213074279 & 11985880 & 17.78 \\ \hline
                AntiKt4EMPFlowJetsAuxDyn.GhostTrack & 160327 & 323350266 & 10503879 & 30.78 \\ \hline
                AntiKt4EMPFlowJetsAuxDyn.NumTrkPt500 & 160327 & 135660076 & 10130123 & 13.39 \\ \hline
                AntiKt10UFOCSSKSoftDropBeta100Zcut10JetsAux. & 160327 & 127450857 & 9931240 & 12.83 \\ \hline
                egammaClustersAuxDyn.constituentClusterLinks & 160327 & 157165174 & 9461105 & 16.61 \\ \hline
            \end{tabular}
    }
    \caption{Top 10 branches sorted by compressed file size in bytes, Data PHYS [Athena v24.0.16  without limit to the basket buffer.]}
    \label{Tab: 24.0.16-Data-PHYS-no-lim-top-compfs}
\end{table}


% % ======= DATA PHYSLITE =======
% % data-24.0.16 Top branches sorted by CF 
% Athena version 24.0.16, 
% ========= default configuration ========
% input file: # TODO
% PHYS/PHYSLITE: PHYSLITE

\begin{table}[!ht]
    \centering
    \resizebox{\textwidth}{!}{
        \begin{tabular}{|l|l|l|l|l|}
            \hline
                Athena v24.0.16 (default) Data PHYSLITE branch & number-of-entries & total-file-size (bytes) & compressed-file-size (bytes) & compression factor \\ \hline
                PrimaryVerticesAuxDyn.trackParticleLinks & 160327 & 2251059204 & 23881552 & 94.26 \\ \hline
                AnalysisTauJetsAuxDyn.tauTrackLinks & 160327 & 38015602 & 1216895 & 31.24 \\ \hline
                InDetTrackParticlesAuxDyn.TTVA\_AMVFVertices & 160327 & 76338802 & 2493392 & 30.62 \\ \hline
                AnalysisJetsAuxDyn.GhostTrack & 160327 & 323534910 & 10626155 & 30.45 \\ \hline
                PrimaryVerticesAuxDyn.neutralParticleLinks & 160327 & 17396255 & 813683 & 21.38 \\ \hline
                AnalysisJetsAuxDyn.SumPtChargedPFOPt500 & 160327 & 135733638 & 6631263 & 20.47 \\ \hline
                AnalysisPhotonsAuxDyn.caloClusterLinks & 160327 & 14665940 & 746015 & 19.66 \\ \hline
                EventInfoAux. & 160327 & 8689410 & 447575 & 19.41 \\ \hline
                TauTracksAux. & 160327 & 8689410 & 447575 & 19.41 \\ \hline
                AnalysisMuonsAux. & 160327 & 8690706 & 448873 & 19.36 \\ \hline
            \end{tabular}
    }
    \caption{Top 10 branches sorted by compression factor, Data PHYSLITE [Athena v24.0.16 default configuration.]}
    \label{Tab: 24.0.16-data-PHYSLITE-default-top-CF}
\end{table}


% data-24.0.16 Top branches sorted by CF
% Athena version 24.0.16, 
% ========== no-lim configuration ===========
% input file: # TODO
% PHYS/PHYSLITE: PHYSLITE

\begin{table}[!ht]
    \centering
    \resizebox{\textwidth}{!}{
        \begin{tabular}{|l|l|l|l|l|}
            \hline
                Athena v24.0.16 (no-lim) Data PHYSLITE branch & number-of-entries & total-file-size (bytes) & compressed-file-size (bytes) & compression factor \\ \hline
                PrimaryVerticesAuxDyn.trackParticleLinks & 160327 & 2250525954 & 23380488 & 96.26 \\ \hline
                InDetTrackParticlesAuxDyn.TTVA\_AMVFVertices & 160327 & 76299634 & 2370547 & 32.19 \\ \hline
                AnalysisTauJetsAuxDyn.tauTrackLinks & 160327 & 38009482 & 1206332 & 31.51 \\ \hline
                AnalysisJetsAuxDyn.GhostTrack & 160327 & 323375652 & 10609747 & 30.48 \\ \hline
                AnalysisJetsAuxDyn.SumPtChargedPFOPt500 & 160327 & 135667174 & 6317135 & 21.48 \\ \hline
                PrimaryVerticesAuxDyn.neutralParticleLinks & 160327 & 17396255 & 813639 & 21.38 \\ \hline
                AnalysisPhotonsAuxDyn.caloClusterLinks & 160327 & 14666063 & 746490 & 19.65 \\ \hline
                EventInfoAux. & 160327 & 8689508 & 447743 & 19.41 \\ \hline
                TauTracksAux. & 160327 & 8689508 & 447743 & 19.41 \\ \hline
                AnalysisMuonsAux. & 160327 & 8690808 & 449046 & 19.35 \\ \hline
            \end{tabular}
    }
    \caption{Top 10 branches sorted by compression factor, Data PHYSLITE [Athena v24.0.16 without limit to the basket buffer.]}
    \label{Tab: 24.0.16-data-PHYSLITE-no-lim-top-CF}
\end{table}
% % Data-24.0.16 Top branches sorted by totalfs
% Athena version 24.0.16, default configuration
% input file:
% PHYS/PHYSLITE: PHYSLITE

\begin{table}[!ht]
    \centering
    \resizebox{\textwidth}{!}{
        \begin{tabular}{|l|l|l|l|l|}
            \hline
                Athena v24.0.16 (default) Data PHYSLITE branch & number-of-entries & total-file-size (bytes) & compressed-file-size (bytes) & compression factor \\ \hline
                PrimaryVerticesAuxDyn.trackParticleLinks & 160327 & 2251059204 & 23881552 & 94.26 \\ \hline
                AnalysisJetsAuxDyn.GhostTrack & 160327 & 323534910 & 10626155 & 30.45 \\ \hline
                AnalysisJetsAuxDyn.SumPtChargedPFOPt500 & 160327 & 135733638 & 6631263 & 20.47 \\ \hline
                AnalysisJetsAuxDyn.NumTrkPt1000 & 160327 & 135722038 & 7723756 & 17.57 \\ \hline
                AnalysisJetsAuxDyn.NumTrkPt500 & 160327 & 135720588 & 10243242 & 13.25 \\ \hline
                AnalysisLargeRJetsAuxDyn.constituentLinks & 160327 & 93806466 & 6306000 & 14.88 \\ \hline
                InDetTrackParticlesAuxDyn.TTVA\_AMVFVertices & 160327 & 76338802 & 2493392 & 30.62 \\ \hline
                TauTracksAuxDyn.trackLinks & 160327 & 60797427 & 3392673 & 17.92 \\ \hline
                AnalysisTauJetsAuxDyn.tauTrackLinks & 160327 & 38015602 & 1216895 & 31.24 \\ \hline
                PrimaryVerticesAuxDyn.neutralParticleLinks & 160327 & 17396255 & 813683 & 21.38 \\ \hline
            \end{tabular}
    }
    \caption{Top 10 branches sorted by total file size, Data PHYSLITE [Athena v24.0.16 default configuration.]}
    \label{Tab: 24.0.16-Data-PHYSLITE-default-top-totalfs}
\end{table}

% Data-24.0.16 Top branches sorted by totalfs
% Athena version 24.0.16, no-lim configuration
% input file: 
% PHYS/PHYSLITE: PHYSLITE

\begin{table}[!ht]
    \centering
    \resizebox{\textwidth}{!}{
        \begin{tabular}{|l|l|l|l|l|}
            \hline
                Athena v24.0.16 (no-lim) Data PHYSLITE branch & number-of-entries & total-file-size (bytes) & compressed-file-size (bytes) & compression factor \\ \hline
                PrimaryVerticesAuxDyn.trackParticleLinks & 160327 & 2250525954 & 23380488 & 96.26 \\ \hline
                AnalysisJetsAuxDyn.GhostTrack & 160327 & 323375652 & 10609747 & 30.48 \\ \hline
                AnalysisJetsAuxDyn.SumPtChargedPFOPt500 & 160327 & 135667174 & 6317135 & 21.48 \\ \hline
                AnalysisJetsAuxDyn.NumTrkPt1000 & 160327 & 135659862 & 7553049 & 17.96 \\ \hline
                AnalysisJetsAuxDyn.NumTrkPt500 & 160327 & 135658948 & 10152473 & 13.36 \\ \hline
                AnalysisLargeRJetsAuxDyn.constituentLinks & 160327 & 93740946 & 5750641 & 16.30 \\ \hline
                InDetTrackParticlesAuxDyn.TTVA\_AMVFVertices & 160327 & 76299634 & 2370547 & 32.19 \\ \hline
                TauTracksAuxDyn.trackLinks & 160327 & 60763794 & 3266591 & 18.60 \\ \hline
                AnalysisTauJetsAuxDyn.tauTrackLinks & 160327 & 38009482 & 1206332 & 31.51 \\ \hline
                PrimaryVerticesAuxDyn.neutralParticleLinks & 160327 & 17396255 & 813639 & 21.38 \\ \hline
            \end{tabular}
    }
    \caption{Top 10 branches sorted by total file size in bytes for Data PHYSLITE [Athena v24.0.16  without limit to the basket buffer.]}
    \label{Tab: 24.0.16-Data-PHYSLITE-no-lim-top-totalfs}
\end{table}

% % Data-24.0.16 Top branches sorted by compfs
% Athena version 24.0.16, default configuration
% input file:
% PHYS/PHYSLITE: PHYSLITE

\begin{table}[!ht]
    \centering
    \resizebox{\textwidth}{!}{
        \begin{tabular}{|l|l|l|l|l|}
            \hline
                Athena v24.0.16 (default) Data PHYSLITE branch & number-of-entries & total-file-size (bytes) & compressed-file-size (bytes) & compression factor \\ \hline
                PrimaryVerticesAuxDyn.trackParticleLinks & 160327 & 2251059204 & 23881552 & 94.26 \\ \hline
                AnalysisJetsAuxDyn.GhostTrack & 160327 & 323534910 & 10626155 & 30.45 \\ \hline
                AnalysisJetsAuxDyn.NumTrkPt500 & 160327 & 135720588 & 10243242 & 13.25 \\ \hline
                AnalysisJetsAuxDyn.NumTrkPt1000 & 160327 & 135722038 & 7723756 & 17.57 \\ \hline
                AnalysisJetsAuxDyn.SumPtChargedPFOPt500 & 160327 & 135733638 & 6631263 & 20.47 \\ \hline
                AnalysisLargeRJetsAuxDyn.constituentLinks & 160327 & 93806466 & 6306000 & 14.88 \\ \hline
                TauTracksAuxDyn.trackLinks & 160327 & 60797427 & 3392673 & 17.92 \\ \hline
                InDetTrackParticlesAuxDyn.TTVA\_AMVFVertices & 160327 & 76338802 & 2493392 & 30.62 \\ \hline
                egammaClustersAuxDyn.constituentClusterLinks & 160327 & 15912038 & 1452452 & 10.96 \\ \hline
                AntiKt10UFOCSSKJetsAuxDyn.GhostAntiKtVR30Rmax4Rmin02PV0TrackJets & 160327 & 15345300 & 1218509 & 12.59 \\ \hline
            \end{tabular}
    }
    \caption{Top 10 branches sorted by compressed file size in bytes, Data PHYSLITE [Athena v24.0.16 default configuration.]}
    \label{Tab: 24.0.16-Data-PHYSLITE-default-top-compfs}
\end{table}

% Data-24.0.16 Top branches sorted by compfs
% Athena version 24.0.16, no-lim configuration
% input file: 
% PHYS/PHYSLITE: PHYSLITE

\begin{table}[!ht]
    \centering
    \resizebox{\textwidth}{!}{
        \begin{tabular}{|l|l|l|l|l|}
            \hline
                Athena v24.0.16 (no-lim) Data PHYSLITE branch & number-of-entries & total-file-size (bytes) & compressed-file-size (bytes) & compression factor \\ \hline
                PrimaryVerticesAuxDyn.trackParticleLinks & 160327 & 2250525954 & 23380488 & 96.26 \\ \hline
                AnalysisJetsAuxDyn.GhostTrack & 160327 & 323375652 & 10609747 & 30.48 \\ \hline
                AnalysisJetsAuxDyn.NumTrkPt500 & 160327 & 135658948 & 10152473 & 13.36 \\ \hline
                AnalysisJetsAuxDyn.NumTrkPt1000 & 160327 & 135659862 & 7553049 & 17.96 \\ \hline
                AnalysisJetsAuxDyn.SumPtChargedPFOPt500 & 160327 & 135667174 & 6317135 & 21.48 \\ \hline
                AnalysisLargeRJetsAuxDyn.constituentLinks & 160327 & 93740946 & 5750641 & 16.30 \\ \hline
                TauTracksAuxDyn.trackLinks & 160327 & 60763794 & 3266591 & 18.60 \\ \hline
                InDetTrackParticlesAuxDyn.TTVA\_AMVFVertices & 160327 & 76299634 & 2370547 & 32.19 \\ \hline
                egammaClustersAuxDyn.constituentClusterLinks & 160327 & 15912167 & 1452450 & 10.96 \\ \hline
                AntiKt10UFOCSSKJetsAuxDyn.GhostAntiKtVR30Rmax4Rmin02PV0TrackJets & 160327 & 15345449 & 1218276 & 12.60 \\ \hline
            \end{tabular}
    }
    \caption{Top 10 branches sorted by compressed file size in bytes, Data PHYSLITE [Athena v24.0.16  without limit to the basket buffer.]}
    \label{Tab: 24.0.16-Data-PHYSLITE-no-lim-top-compfs}
\end{table}

% % ======= MC PHYS =======
% % MC-24.0.16 Top branches sorted by CF
% Athena version 24.0.16, default configuration
% input file:
% PHYS/PHYSLITE: PHYS

\begin{table}[!ht]
    \centering
    \resizebox{\textwidth}{!}{
        \begin{tabular}{|l|l|l|l|l|}
            \hline
                Athena v24.0.16 (default) MC PHYS branch & number-of-entries & total-file-size (bytes) & compressed-file-size (bytes) & compression factor \\ \hline
                PrimaryVerticesAuxDyn.trackParticleLinks & 140000 & 2250417773 & 25185195 & 89.35 \\ \hline
                HardScatterVerticesAuxDyn.incomingParticleLinks & 140000 & 124303336 & 1735033 & 71.64 \\ \hline
                HLTNav\_Summary\_DAODSlimmedAuxDyn.linkColNames & 140000 & 822108623 & 12505035 & 65.74 \\ \hline
                HardScatterVerticesAuxDyn.outgoingParticleLinks & 140000 & 113907432 & 1941603 & 58.67 \\ \hline
                ElectronsAuxDyn.caloClusterLinks & 140000 & 63520105 & 1115267 & 56.96 \\ \hline
                PhotonsAuxDyn.caloClusterLinks & 140000 & 90142260 & 1787867 & 50.42 \\ \hline
                TauJetsAuxDyn.tauTrackLinks & 140000 & 342314812 & 7728912 & 44.29 \\ \hline
                TruthBosonsWithDecayVerticesAuxDyn.incomingParticleLinks & 140000 & 33099871 & 760909 & 43.50 \\ \hline
                AntiKt4EMPFlowJetsAuxDyn.NumChargedPFOPt1000 & 140000 & 156091022 & 4059766 & 38.45 \\ \hline
                AntiKt4EMPFlowJetsAuxDyn.NumChargedPFOPt500 & 140000 & 156089396 & 4167516 & 37.45 \\ \hline
            \end{tabular}
    }
    \caption{Top 10 branches sorted by compression factor, MC PHYS [Athena v24.0.16 default configuration.]}
    \label{Tab: 24.0.16-MC-PHYS-default-top-CF}
\end{table}

% MC-24.0.16 Top branches sorted by CF
% Athena version 24.0.16, no-lim configuration
% input file: 
% PHYS/PHYSLITE: PHYS

\begin{table}[!ht]
    \centering
    \resizebox{\textwidth}{!}{
        \begin{tabular}{|l|l|l|l|l|}
            \hline
                Athena v24.0.16 (no-lim) MC PHYS branch & number-of-entries & total-file-size (bytes) & compressed-file-size (bytes) & compression factor \\ \hline
                PrimaryVerticesAuxDyn.trackParticleLinks & 140000 & 2249713773 & 23950434 & 93.93 \\ \hline
                HardScatterVerticesAuxDyn.incomingParticleLinks & 140000 & 124220704 & 1346890 & 92.23 \\ \hline
                HardScatterVerticesAuxDyn.outgoingParticleLinks & 140000 & 113811996 & 1511222 & 75.31 \\ \hline
                HLTNav\_Summary\_DAODSlimmedAuxDyn.linkColNames & 140000 & 821538833 & 12456705 & 65.95 \\ \hline
                ElectronsAuxDyn.caloClusterLinks & 140000 & 63488515 & 999469 & 63.52 \\ \hline
                PhotonsAuxDyn.caloClusterLinks & 140000 & 90081425 & 1533011 & 58.76 \\ \hline
                TauJetsAuxDyn.tauTrackLinks & 140000 & 342114444 & 7475832 & 45.76 \\ \hline
                TruthBosonsWithDecayVerticesAuxDyn.incomingParticleLinks & 140000 & 33099871 & 761017 & 43.49 \\ \hline
                AntiKt4EMPFlowJetsAuxDyn.NumChargedPFOPt1000 & 140000 & 155985758 & 3657524 & 42.65 \\ \hline
                AntiKt4EMPFlowJetsAuxDyn.NumChargedPFOPt500 & 140000 & 155984948 & 3756725 & 41.52 \\ \hline
            \end{tabular}
    }
    \caption{Top 10 branches sorted by compression factor MC PHYS [Athena v24.0.16  without limit to the basket buffer.]}
    \label{Tab: 24.0.16-MC-PHYS-no-lim-top-CF}
\end{table}

% % MC-24.0.16 Top branches sorted by totalfs
% Athena version 24.0.16, default configuration
% input file:
% PHYS/PHYSLITE: PHYS

\begin{table}[!ht]
    \centering
    \resizebox{\textwidth}{!}{
        \begin{tabular}{|l|l|l|l|l|}
            \hline
                Athena v24.0.16 (default) MC PHYS branch & number-of-entries & total-file-size (bytes) & compressed-file-size (bytes) & compression factor \\ \hline
                PrimaryVerticesAuxDyn.trackParticleLinks & 140000 & 2250417773 & 25185195 & 89.35 \\ \hline
                AntiKt10LCTopoJetsAux. & 140000 & 1077455736 & 79438335 & 13.56 \\ \hline
                HLTNav\_Summary\_DAODSlimmedAuxDyn.linkColNames & 140000 & 822108623 & 12505035 & 65.74 \\ \hline
                AntiKt10TruthJetsAux. & 140000 & 641553782 & 51322938 & 12.50 \\ \hline
                AntiKt4EMTopoJetsAuxDyn.GhostTrack & 140000 & 579616988 & 15941253 & 36.36 \\ \hline
                AntiKt10TruthSoftDropBeta100Zcut10JetsAux. & 140000 & 510187983 & 43433379 & 11.75 \\ \hline
                AntiKt4EMPFlowJetsAuxDyn.GhostTrack & 140000 & 433900768 & 13781266 & 31.48 \\ \hline
                HLTNav\_Summary\_DAODSlimmedAuxDyn.linkColClids & 140000 & 409545832 & 11174725 & 36.65 \\ \hline
                HLTNav\_Summary\_DAODSlimmedAuxDyn.linkColKeys & 140000 & 409542270 & 12257923 & 33.41 \\ \hline
                TauJetsAuxDyn.tauTrackLinks & 140000 & 342314812 & 7728912 & 44.29 \\ \hline
            \end{tabular}
    }
    \caption{Top 10 branches sorted by total file size in bytes, MC PHYS [Athena v24.0.16 default configuration.]}
    \label{Tab: 24.0.16-MC-PHYS-default-top-totalfs}
\end{table}

% MC-24.0.16 Top branches sorted by totalfs
% Athena version 24.0.16, no-lim configuration
% input file: 
% PHYS/PHYSLITE: PHYS

\begin{table}[!ht]
    \centering
    \resizebox{\textwidth}{!}{
        \begin{tabular}{|l|l|l|l|l|}
            \hline
                Athena v24.0.16 (no-lim) MC PHYS branch & number-of-entries & total-file-size (bytes) & compressed-file-size (bytes) & compression factor \\ \hline
                PrimaryVerticesAuxDyn.trackParticleLinks & 140000 & 2249713773 & 23950434 & 93.93 \\ \hline
                AntiKt10LCTopoJetsAux. & 140000 & 1076842198 & 77693658 & 13.86 \\ \hline
                HLTNav\_Summary\_DAODSlimmedAuxDyn.linkColNames & 140000 & 821538833 & 12456705 & 65.95 \\ \hline
                AntiKt10TruthJetsAux. & 140000 & 641194442 & 50285540 & 12.75 \\ \hline
                AntiKt4EMTopoJetsAuxDyn.GhostTrack & 140000 & 579253800 & 15645276 & 37.02 \\ \hline
                AntiKt10TruthSoftDropBeta100Zcut10JetsAux. & 140000 & 509846734 & 42519713 & 11.99 \\ \hline
                AntiKt4EMPFlowJetsAuxDyn.GhostTrack & 140000 & 433629208 & 13589471 & 31.91 \\ \hline
                HLTNav\_Summary\_DAODSlimmedAuxDyn.linkColClids & 140000 & 409266982 & 11508289 & 35.56 \\ \hline
                HLTNav\_Summary\_DAODSlimmedAuxDyn.linkColKeys & 140000 & 409265565 & 11855323 & 34.52 \\ \hline
                TauJetsAuxDyn.tauTrackLinks & 140000 & 342114444 & 7475832 & 45.76 \\ \hline
            \end{tabular}
    }
    \caption{Top 10 branches sorted by total file size in bytes, MC PHYS [Athena v24.0.16  without limit to the basket buffer.]}
    \label{Tab: 24.0.16-MC-PHYS-no-lim-top-totalfs}
\end{table}

% % MC-24.0.16 Top branches sorted by compfs
% Athena version 24.0.16, default configuration
% input file:
% PHYS/PHYSLITE: PHYS

\begin{table}[!ht]
    \centering
    \resizebox{\textwidth}{!}{
        \begin{tabular}{|l|l|l|l|l|}
            \hline
                Athena v24.0.16 (default) MC PHYS branch & number-of-entries & total-file-size (bytes) & compressed-file-size (bytes) & compression factor \\ \hline
                AntiKt10LCTopoJetsAux. & 140000 & 1077455736 & 79438335 & 13.56 \\ \hline
                AntiKt10TruthJetsAux. & 140000 & 641553782 & 51322938 & 12.50 \\ \hline
                AntiKt10TruthSoftDropBeta100Zcut10JetsAux. & 140000 & 510187983 & 43433379 & 11.75 \\ \hline
                AntiKt10TruthTrimmedPtFrac5SmallR20JetsAux. & 140000 & 320409096 & 31167984 & 10.28 \\ \hline
                PrimaryVerticesAuxDyn.trackParticleLinks & 140000 & 2250417773 & 25185195 & 89.35 \\ \hline
                AntiKt10UFOCSSKJetsAux. & 140000 & 312449480 & 24102594 & 12.96 \\ \hline
                TruthHFWithDecayParticlesAuxDyn.childLinks & 140000 & 263956951 & 18072164 & 14.61 \\ \hline
                AntiKt4EMTopoJetsAuxDyn.NumTrkPt500 & 140000 & 220265908 & 17807961 & 12.37 \\ \hline
                TruthHFWithDecayVerticesAuxDyn.outgoingParticleLinks & 140000 & 251575299 & 15998478 & 15.72 \\ \hline
                AntiKt4EMTopoJetsAuxDyn.GhostTrack & 140000 & 579616988 & 15941253 & 36.36 \\ \hline
            \end{tabular}
    }
    \caption{Top 10 branches sorted by compressed file size in bytes, MC PHYS [Athena v24.0.16 default configuration.]}
    \label{Tab: 24.0.16-MC-PHYS-default-top-compfs}
\end{table}

% MC-24.0.16 Top branches sorted by compfs
% Athena version 24.0.16, no-lim configuration
% input file: 
% PHYS/PHYSLITE: PHYS

\begin{table}[!ht]
    \centering
    \resizebox{\textwidth}{!}{
        \begin{tabular}{|l|l|l|l|l|}
            \hline
                Athena v24.0.16 (no-lim) MC PHYS branch & number-of-entries & total-file-size (bytes) & compressed-file-size (bytes) & compression factor \\ \hline
                AntiKt10LCTopoJetsAux. & 140000 & 1076842198 & 77693658 & 13.86 \\ \hline
                AntiKt10TruthJetsAux. & 140000 & 641194442 & 50285540 & 12.75 \\ \hline
                AntiKt10TruthSoftDropBeta100Zcut10JetsAux. & 140000 & 509846734 & 42519713 & 11.99 \\ \hline
                AntiKt10TruthTrimmedPtFrac5SmallR20JetsAux. & 140000 & 320195336 & 30366856 & 10.54 \\ \hline
                PrimaryVerticesAuxDyn.trackParticleLinks & 140000 & 2249713773 & 23950434 & 93.93 \\ \hline
                AntiKt10UFOCSSKJetsAux. & 140000 & 312270524 & 22712285 & 13.75 \\ \hline
                AntiKt4EMTopoJetsAuxDyn.NumTrkPt500 & 140000 & 220130308 & 17779771 & 12.38 \\ \hline
                TruthHFWithDecayParticlesAuxDyn.childLinks & 140000 & 263779913 & 17291176 & 15.26 \\ \hline
                AntiKt4EMTopoJetsAuxDyn.GhostTrack & 140000 & 579253800 & 15645276 & 37.02 \\ \hline
                TruthHFWithDecayVerticesAuxDyn.outgoingParticleLinks & 140000 & 251408159 & 15384701 & 16.34 \\ \hline
            \end{tabular}
    }
    \caption{Top 10 branches sorted by compressed file size in bytes, MC PHYS [Athena v24.0.16  without limit to the basket buffer.]}
    \label{Tab: 24.0.16-MC-PHYS-no-lim-top-compfs}
\end{table}

% ====== MC PHYSLITE =======

% MC-24.0.16 Top branches sorted by CF
% Athena version 24.0.16, default configuration
% input file: # TODO
% PHYS/PHYSLITE: PHYSLITE

\begin{table}[!ht]
    \centering
    \vspace{10px}
    \caption{Top 10 branches sorted by compression factor, MC PHYSLITE [Athena v24.0.16 default configuration.]}
    \label{Tab: 24.0.16-MC-PHYSLITE-default-top-CF}
    \resizebox{\textwidth}{!}{
        \begin{tabular}{|l|c|c|c|c|}
    \hline
        Athena v24.0.16 (default) MC branch & Branch size (kB) & Total-file-size (MB) & Compressed-file-size (MB)  & Compression factor \\ \hline
        PrimaryVerticesAuxDyn.trackParticleLinks & 128 & 2146.2 & 24.0 & 89.4 \\ \hline
        HardScatterVerticesAuxDyn.incomingParticleLinks & 128 & 118.5 & 1.7 & 71.6 \\ \hline
        HLTNav\_Summary\_DAODSlimmedAuxDyn.linkColNames & 128 & 784.0 & 11.9 & 65.7 \\ \hline
        HardScatterVerticesAuxDyn.outgoingParticleLinks & 128 & 108.6 & 1.9 & 58.7 \\ \hline
        TruthBosonsWithDecayVerticesAuxDyn.incomingParticleLinks & 96 & 31.6 & 0.7 & 43.5 \\ \hline
        HLTNav\_Summary\_DAODSlimmedAuxDyn.linkColClids & 128 & 390.6 & 10.7 & 36.6 \\ \hline
        AnalysisTauJetsAuxDyn.tauTrackLinks & 128 & 75.0 & 2.0 & 36.6 \\ \hline
        HLTNav\_Summary\_DAODSlimmedAuxDyn.linkColKeys & 128 & 390.6 & 11.7 & 33.4 \\ \hline
        AnalysisJetsAuxDyn.GhostTrack & 128 & 413.8 & 13.1 & 31.5 \\ \hline
        TruthBosonsWithDecayVerticesAuxDyn.outgoingParticleLinks & 83.5 & 27.3 & 0.9 & 31.0 \\ \hline
    \end{tabular}
    }
    \vspace{10px}
\end{table}

% MC-24.0.16 Top branches sorted by CF
% Athena version 24.0.16, no-lim configuration
% input file: # TODO
% PHYS/PHYSLITE: PHYSLITE

\begin{table}[!ht]
    \centering
    % \vspace{10px} % already 10 px from above
    \caption{Top 10 branches sorted by compression factor, MC PHYSLITE [Athena v24.0.16  without limit to the basket buffer.]}
    \label{Tab: 24.0.16-MC-PHYSLITE-no-lim-top-CF}
    \resizebox{\textwidth}{!}{
       \begin{tabular}{|l|c|c|c|c|}
    \hline
        Athena v24.0.16 (no-lim) MC branch & Branch size (kB) & Total-file-size (MB) & Compressed-file-size (MB)  & Compression factor \\ \hline
        PrimaryVerticesAuxDyn.trackParticleLinks & 1293.5 & 2145.5 & 22.9 & 93.5 \\ \hline
        HardScatterVerticesAuxDyn.incomingParticleLinks & 693.0 & 118.5 & 1.3 & 90.1 \\ \hline
        HardScatterVerticesAuxDyn.outgoingParticleLinks & 635.5 & 108.5 & 1.5 & 74.0 \\ \hline
        HLTNav\_Summary\_DAODSlimmedAuxDyn.linkColNames & 1293.5 & 783.5 & 11.9 & 65.8 \\ \hline
        TruthBosonsWithDecayVerticesAuxDyn.incomingParticleLinks & 96.0 & 31.6 & 0.7 & 43.5 \\ \hline
        AnalysisTauJetsAuxDyn.tauTrackLinks & 447.0 & 74.9 & 1.9 & 39.2 \\ \hline
        HLTNav\_Summary\_DAODSlimmedAuxDyn.linkColClids & 1293.5 & 390.3 & 11.0 & 35.5 \\ \hline
        HLTNav\_Summary\_DAODSlimmedAuxDyn.linkColKeys & 1293.5 & 390.3 & 11.3 & 34.5 \\ \hline
        AnalysisJetsAuxDyn.GhostTrack & 1293.5 & 413.5 & 13.0 & 31.9 \\ \hline
        TruthBosonsWithDecayVerticesAuxDyn.outgoingParticleLinks & 83.5 & 27.3 & 0.9 & 31.0 \\ \hline
    \end{tabular}
    }
\end{table}

% MC-24.0.16 Top branches sorted by totalfs
% Athena version 24.0.16, default configuration
% input file:
% PHYS/PHYSLITE: PHYSLITE

\begin{table}[!ht]
    \centering
    \resizebox{\textwidth}{!}{
        \begin{tabular}{|l|c|c|c|c|}
    \hline
        Athena v24.0.16 (default) MC branch & Branch size (kB) & Total-file-size (MB) & Compressed-file-size (MB)  & Compression factor \\ \hline
        PrimaryVerticesAuxDyn.trackParticleLinks & 128 & 2146.2 & 24.0 & 89.4 \\ \hline
        HLTNav\_Summary\_DAODSlimmedAuxDyn.linkColNames & 128 & 784.0 & 11.9 & 65.7 \\ \hline
        AnalysisJetsAuxDyn.GhostTrack & 128 & 413.8 & 13.1 & 31.5 \\ \hline
        HLTNav\_Summary\_DAODSlimmedAuxDyn.linkColClids & 128 & 390.6 & 10.7 & 36.6 \\ \hline
        HLTNav\_Summary\_DAODSlimmedAuxDyn.linkColKeys & 128 & 390.6 & 11.7 & 33.4 \\ \hline
        AnalysisJetsAuxDyn.SumPtChargedPFOPt500 & 128 & 148.9 & 7.3 & 20.5 \\ \hline
        AnalysisJetsAuxDyn.NumTrkPt1000 & 128 & 148.8 & 8.7 & 17.2 \\ \hline
        AnalysisJetsAuxDyn.NumTrkPt500 & 128 & 148.8 & 11.9 & 12.5 \\ \hline
        HardScatterVerticesAuxDyn.incomingParticleLinks & 128 & 118.5 & 1.7 & 71.6 \\ \hline
        AnalysisLargeRJetsAuxDyn.constituentLinks & 128 & 111.5 & 7.1 & 15.8 \\ \hline
    \end{tabular}
    }
    \caption{Top 10 branches sorted by total file size in bytes, MC PHYSLITE [Athena v24.0.16 default configuration.]}
    \label{Tab: 24.0.16-MC-PHYSLITE-default-top-totalfs}
\end{table}

% MC-24.0.16 Top branches sorted by totalfs
% Athena version 24.0.16, no-lim configuration
% input file: 
% PHYS/PHYSLITE: PHYSLITE

\begin{table}[!ht]
    \centering
    \resizebox{\textwidth}{!}{
         \begin{tabular}{|l|c|c|c|c|}
    \hline
        Athena v24.0.16 (no-lim) MC branch & Branch size (kB) & Total-file-size (MB) & Compressed-file-size (MB)  & Compression factor \\ \hline
        PrimaryVerticesAuxDyn.trackParticleLinks & 1293.5 & 2145.5 & 22.9 & 93.6 \\ \hline
        HLTNav\_Summary\_DAODSlimmedAuxDyn.linkColNames & 1293.5 & 783.5 & 11.9 & 65.8 \\ \hline
        AnalysisJetsAuxDyn.GhostTrack & 1293.5 & 413.5 & 13.0 & 31.9 \\ \hline
        HLTNav\_Summary\_DAODSlimmedAuxDyn.linkColClids & 1293.5 & 390.3 & 11.0 & 35.5 \\ \hline
        HLTNav\_Summary\_DAODSlimmedAuxDyn.linkColKeys & 1293.5 & 390.3 & 11.3 & 34.5 \\ \hline
        AnalysisJetsAuxDyn.SumPtChargedPFOPt500 & 905.5 & 148.8 & 6.8 & 21.9 \\ \hline
        AnalysisJetsAuxDyn.NumTrkPt1000 & 905 & 148.8 & 8.5 & 17.6 \\ \hline
        AnalysisJetsAuxDyn.NumTrkPt500 & 905 & 148.8 & 11.8 & 12.6 \\ \hline
        HardScatterVerticesAuxDyn.incomingParticleLinks & 693 & 118.5 & 1.3 & 90.2 \\ \hline
        AnalysisLargeRJetsAuxDyn.constituentLinks & 950.5 & 111.4 & 6.4 & 17.4 \\ \hline
    \end{tabular}
    }
    \caption{Top 10 branches sorted by total file size in bytes, MC PHYSLITE [Athena v24.0.16  without limit to the basket buffer.]}
    \label{Tab: 24.0.16-MC-PHYSLITE-no-lim-top-totalfs}
\end{table}

% MC-24.0.16 Top branches sorted by compfs
% Athena version 24.0.16, default configuration
% input file:
% PHYS/PHYSLITE: PHYSLITE

\begin{table}[!ht]
    \centering
    \caption{Top 10 branches sorted by compressed file size in bytes, MC PHYSLITE [Athena v24.0.16 default configuration.]}
    \label{Tab: 24.0.16-MC-PHYSLITE-default-top-compfs}
    \resizebox{\textwidth}{!}{
       \begin{tabular}{|l|c|c|c|c|}
    \hline
        Athena v24.0.16 (default) MC branch & Branch size (kB) & Total-file-size (MB) & Compressed-file-size (MB)  & Compression factor \\ \hline
        PrimaryVerticesAuxDyn.trackParticleLinks & 128 & 2146.2 & 24.0 & 89.4 \\ \hline
        AnalysisJetsAuxDyn.GhostTrack & 128 & 413.8 & 13.1 & 31.5 \\ \hline
        AnalysisJetsAuxDyn.NumTrkPt500 & 128 & 148.8 & 11.9 & 12.5 \\ \hline
        HLTNav\_Summary\_DAODSlimmedAuxDyn.linkColNames & 128 & 784.0 & 11.9 & 65.7 \\ \hline
        HLTNav\_Summary\_DAODSlimmedAuxDyn.linkColKeys & 128 & 390.6 & 11.7 & 33.4 \\ \hline
        HLTNav\_Summary\_DAODSlimmedAuxDyn.linkColClids & 128 & 390.6 & 10.7 & 36.6 \\ \hline
        AnalysisJetsAuxDyn.NumTrkPt1000 & 128 & 148.8 & 8.7 & 17.2 \\ \hline
        AnalysisJetsAuxDyn.SumPtChargedPFOPt500 & 128 & 148.9 & 7.3 & 20.5 \\ \hline
        AnalysisLargeRJetsAuxDyn.constituentLinks & 128 & 111.5 & 7.1 & 15.8 \\ \hline
        HLTNav\_Summary\_DAODSlimmedAuxDyn.name & 128 & 80.8 & 4.4 & 18.4 \\ \hline
    \end{tabular}
    }
    \vspace{10px}
\end{table}

% MC-24.0.16 Top branches sorted by compfs
% Athena version 24.0.16, no-lim configuration
% input file: 
% PHYS/PHYSLITE: PHYSLITE

\begin{table}[!ht]
    \centering
    \caption{Top 10 branches sorted by compressed file size in bytes, MC PHYSLITE [Athena v24.0.16  without limit to the basket buffer.]}
    \label{Tab: 24.0.16-MC-PHYSLITE-no-lim-top-compfs}
    \resizebox{\textwidth}{!}{
        \begin{tabular}{|l|c|c|c|c|}
    \hline
        Athena v24.0.16 (no-lim) MC branch & Branch size (kB) & Total-file-size (MB) & Compressed-file-size (MB)  & Compression factor \\ \hline
        PrimaryVerticesAuxDyn.trackParticleLinks & 1293.5 & 2145.5 & 22.9 & 93.5 \\ \hline
        AnalysisJetsAuxDyn.GhostTrack & 1293.5 & 413.5 & 13.0 & 31.9 \\ \hline
        HLTNav\_Summary\_DAODSlimmedAuxDyn.linkColNames & 1293.5 & 783.5 & 11.9 & 65.8 \\ \hline
        AnalysisJetsAuxDyn.NumTrkPt500 & 905 & 148.8 & 11.8 & 12.6 \\ \hline
        HLTNav\_Summary\_DAODSlimmedAuxDyn.linkColKeys & 1293.5 & 390.3 & 11.3 & 34.5 \\ \hline
        HLTNav\_Summary\_DAODSlimmedAuxDyn.linkColClids & 1293.5 & 390.3 & 11.0 & 35.5 \\ \hline
        AnalysisJetsAuxDyn.NumTrkPt1000 & 905 & 148.8 & 8.5 & 17.6 \\ \hline
        AnalysisJetsAuxDyn.SumPtChargedPFOPt500 & 905.5 & 148.8 & 6.8 & 21.9 \\ \hline
        AnalysisLargeRJetsAuxDyn.constituentLinks & 950.5 & 111.4 & 6.4 & 17.4 \\ \hline
        HLTNav\_Summary\_DAODSlimmedAuxDyn.name & 242 & 80.8 & 4.5 & 18.0 \\ \hline
    \end{tabular}
    }
    \vspace{10px}
\end{table}



An immediate observation: with the omission of the basket limit, the compression factor increases--this is inline with the original expectation that increased buffer size limits correlate to better compression. 
$\textit{PrimaryVerticesAuxDyn.trackParticleLinks}$ is a branch that consistently has the highest memory usage to disk-space usage ratio of any branch.
I/O buffers are concerned with the total file size of each branch, as it is the uncompressed size of each branch that is held in memory. 
$\textit{HLTNav\_Summary\_DAODSlimmedAuxDyn.linkColNames}$ and others show similar behavior and are consistent with the other DAOD jobs across data, MC, PHYS, and PHYSLITE (the tables of which were omitted for sake of brevity).
Future work could investigate these branches for further optimization.