% I think the conclusions should include some outlook. 

% You can summarize your findings, give some numbers,
%  but also highlight that it would be important to further 
%  study what goes into those branches that have very high compression
%  factors and see if there are any optimizations that can
%  be done in terms of the data that's being stored. 

% We don't own the data, we just write what's given to us 
%  but if we find that someone writes a vector of zeros over 
%  and over again that we can raise this up and ask the domain 
%  experts. 

Initially, limiting the basket buffer size looked appealing because after a certain point (i.e. basket size of 128 kB) the compression factor would stagnate. 
If we were to limit the basket size to something like 128 kB we might find some optimization effects with disk-space and memory. 

Instead, by removing the upper limit of the basket size a greater decrease in DAOD output file size is achieved. 
The largest decrease in file size came from the PHYSLITE Monte Carlo derivation jobs without setting an upper limit to the basket buffer size. 
While similar decreases in file size appear for derivation jobs using data, it's not as apparent for data as it is for Monte Carlo jobs.  
With the removal of an upper-limit to the basket size, ATLAS stands to gain a $5\%$ decrease for PHYSLITE MC DAOD output file sizes, but an $11-12\%$ increase in memory usage could prove a heavy burden (See Tables \ref{Tab: night-mc-outputfs} and \ref{Tab: 24.0.16-mc-outputfs}).

Lastly, by looking at the branches per configuration, highly compressible branches emerge specifically in the MC PHYSLITE output DAOD.
The branches inside the MC PHYSLITE DAOD are suboptimal as they do not conserve disk space; instead, they consume memory inefficiently.
As seen from Table \ref{Tab: 24.0.16-MC-PHYLITE-default-top-CF} through Table \ref{Tab: 24.0.16-MC-PHYSLITE-no-lim-top-compfs}, we have plenty of branches in MC PHYSLITE that are seemingly empty--as indicated by the compression factor being larger than $\mathcal{O}(1)$.
Optimization of the branch data could further optimize DAOD production by reducing the increased memory-usage while keeping the effects of decreased disk-space. 


