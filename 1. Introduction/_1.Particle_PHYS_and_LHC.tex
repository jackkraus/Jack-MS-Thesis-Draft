Particle physics is the branch of physics that seeks out the origins of the universe by probing the smallest interactions at high energies.
It has roots in electromagnetism, with the discovery of the electron and other particles, and quantum mechanics, that include descriptions of atoms, particles and their interactions both relativistic and non-relativistic speeds. 
There have been many efforts in experimentally probing for unique interactions, and the Large Hadron Collider (LHC) at CERN has been at the forefront in revealing new insights.  
The LHC is a 26.7-kilometer ring that crosses between the France-Switzerland border at a depth between 50 and 175 meters underground. [LHC FAQ the guide]