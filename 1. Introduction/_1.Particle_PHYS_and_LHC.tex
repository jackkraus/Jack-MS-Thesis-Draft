Particle physics is the branch of physics that seeks out the origins of the universe by probing the smallest interactions at high energies.
It has roots in electromagnetism, with the discovery of the electron and other particles, and quantum mechanics, that include descriptions of atoms, particles, and their interactions both relativistic and non-relativistic speeds. 
There have been many efforts in experimentally probing for unique interactions, and the Large Hadron Collider (LHC) at CERN has been at the forefront in revealing new insights.  
The LHC is a 26.7-kilometer ring that crosses between the France-Switzerland border at a depth between 50 and 175 meters underground.\cite{LHC_faq_guide}


\subsection{The ATLAS Experiment}

The ATLAS experiment is the largest LHC general purpose detector, and the largest detector ever made for particle collision experiments. 
It's 46 meters long, 25 meters high and 25 meters wide.\cite{ATLAS_Fact_Sheet}
The ATLAS detector is comprised of three main sections, the inner detector, calorimeters and the muon detector system. 

The inner detector measures the direction, momentum and charge of electrically charged particles.
The first point of contact for ATLAS is the pixel detector. It has over 92 million pixels to help determine the origin and momentum of the particle.
Surrounding the pixel detector is the semiconductor tracker, which uses $4,088$ modules of 6 million implanted silicon readout strips.
The semiconductor tracker helps measure the path particles take, called tracks, with precision up to $25\mu m$. 
The final layer of the inner detector is the transition radiation tracker (TRT). 
The TRT is made of a collection of tubes made with many layers of different materials with varying indices of refraction.  
Particles with relativistic velocities have higher Lorentz $\gamma$-factors, see Eq. \eqref{lorentzGamma}, the TRT uses varying materials to discriminate between heavier particles (with low $\gamma$ and radiate less) and lighter particles (higher $\gamma$ and radiate more). \cite{Mindur:2139567}
\begin{equation}\label{lorentzGamma}
    \gamma = \frac{1}{\sqrt{1 - \frac{v^2}{c^2}}}
\end{equation}

