At any given time, code changes to Athena or the LCG stack could break various core I/O functionality. \cite{atlascollaboration2024softwarecomputingrun3}
A measure often taken to ensure stability and security of essential code functionality is the involvement of unit testing during continuous integration of new software. \cite{athenadocs_CIbuild}
It's not always the case that new core I/O functionality is integrated into Athena with new unit tests. 

\section{Athena and ROOT}
% \subsection{What is Athena}
Athena is the open-source software framework for the ATLAS experiment.\cite{athena}
It relies on other software such as ROOT, Geant4 and other software as part of the LCG software stack. 
Athena manages ATLAS production workflows which include event generation, simulation of data, reconstruction from hits, and derivation of reconstructed hits.\cite{athenadocs}


% \subsection{What is ROOT}
ROOT is an open-source software framework used for high-energy physics analysis at CERN.\cite{ROOT_about} 
It uses C++ objects to save, access, and process data brought in by the various experiments based at the LHC, the ATLAS experiment uses it in conjunction with Athena.

% \subsection{How do they work together}

One of the ways Athena works with ROOT is by taking and manipulating ROOT files in such a way to make them smaller. 

% \subsection{what are the steps to go from data taking to analysis?}

\subsection{TTree}

\subsection{Derivation Production Jobs}
A derivation job is that process that takes Analysis Object Data (AODs), which comes from the reconstruction step at $\mathcal{O}(1 \text{ MB})$ per event, and creates a Derived Analysis Object Data (DAOD) which sits at $\mathcal{O}(10 \text{ kB})$ per event.
% \subsection{Why do we need derivation jobs?}


\section{Event Data Models}
% \subsection{What is an Event Data Model}
An Event Data Model (EDM) is a collection of classes and their relationships to each other that provide a representation of an event detected with the goal of making it easier to use and manipulate by developers.
An EDM is how particles and jets are represented in memory, stored to disk, and manipulated in analysis.
It's useful to have an EDM because it brings a commonality to the code, which is useful when developers reside in different groups with various backgrounds.
An EDM allows those developers to more easily debug and communicate issues when they arise.  

\subsection{Transient/Persistent EDM}
One of the previous EDMs used by ATLAS concerned a dual transient/persistent nature of data.
The transient data was present in memory and could have information attatched to the object, this data could gain complexity the more it was used.
Persistent data needed to be simplified before it could be persistified into long-term storage (sent to disk). 
ROOT had trouble handling the complex inheritance models that would come up the more developers used this EDM. 
Additionally, converting from transient to persistent data was an excessive step which was eventually removed by the adoption of using an EDM that blends the two stages of data together, this was dubbed the xAOD EDM.