At any given time, code changes to Athena or the LCG stack could break various core I/O functionality. \cite{atlascollaboration2024softwarecomputingrun3}
A measure often taken to ensure stability and security of essential code functionality is the involvement of unit testing during continuous integration of new software. \cite{athenadocs_CIbuild}
It's not always the case that new core I/O functionality is integrated into Athena with new unit tests. 

\section{Athena and ROOT}
% \subsection{What is Athena}
Athena is the open-source software framework for the ATLAS experiment.\cite{athena}
It relies on other software such as ROOT, Geant4 and other software as part of the LCG software stack. 
Athena manages ATLAS production workflows which include event generation, simulation of data, reconstruction from hits, and derivation of reconstructed hits.\cite{athenadocs}


% \subsection{What is ROOT}
ROOT is an open-source software framework used for high-energy physics analysis at CERN.\cite{ROOT_about} 
It uses C++ objects to save, access, and process data brought in by the various experiments based at the LHC, the ATLAS experiment uses it in conjunction with Athena.

% \subsection{How do they work together}

One of the ways Athena works with ROOT is by taking and manipulating ROOT files in such a way to make them smaller. 

\subsection{What is a derivation job}

\section{xAOD Event Data Model}
\subsection{What is an Event Data Model}
\subsection{Why do we need an EDM}

