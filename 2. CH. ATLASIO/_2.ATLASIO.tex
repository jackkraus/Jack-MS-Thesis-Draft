At any given time, code changes to Athena or the LCG stack could break various core I/O functionality. \cite{atlascollaboration2024softwarecomputingrun3}
A measure often taken to ensure stability and security of essential code functionality is the involvement of unit testing during continuous integration of new software. [Do I need to cite this ????] 
It's not always the case that new core I/O functionality is integrated into Athena with new unit tests. 

\section{Athena and ROOT}
\subsection{What is Athena}
Athena is the open-source software framework for ATLAS. [Cite here]
It relies on other software such as ROOT, Geant4 and other software as part of the LCG software stack. [cite here]

\subsection{What is ROOT}
ROOT is an 

\subsection{How do they work together}
\subsection{What is a derivation job}

\section{xAOD Event Data Model}
\subsection{What is an Event Data Model}
\subsection{Why do we need an EDM}

