\section{Athena job configuration example}

An Athena job configuration is a script that allows the user to steer a specific program in the framework.
Steering allows one to, at a high-level, configure low-level behavior of any kind of production job. 
A general Athena application using \verb|ComponentAccumulator| written in pseudocode would take the form: 
\begin{lstlisting}[language=Python]
    # Import Packages
    from AthenaConfiguration.AllConfigFlags import initConfigFlags
    from AthenaConfiguration.ComponentFactory import CompFactory
    from OutputStreamAthenaPool.OutputStreamConfig import OutputStreamCfg, outputStreamName

    # Configure Output
    outputStreamName = "StreamA"
    outputFileName = "output.root"

    # Setup flags
    flags = initConfigFlags()
    flags.Input.Files = ["input.root"]
    flags.addFlag(f"Output.{streamName}FileName", outputFileName)
    flags.lock()

    # Main Service(s)
    from AthenaConfiguration.MainServicesConfig import MainServicesCfg
    acc = MainServicesCfg( flags )

    # Add algorithms to the accumulator
    acc.addEventAlgo( CompFactory.MyAlgorithm(MyParameters) )

    # Run
    import sys
    sc = acc.run(flags.Exec.MaxEvents)
\end{lstlisting}

The \verb|acc| is the \verb|ComponentAccumulator|, so here the user might have more than one Algorithm it needs to call, but each one would have a separate \verb|.addEventAlgo| call. 
When \verb|flag.lock()| is called, any previously established flags will be set in place and unable to be changed. 