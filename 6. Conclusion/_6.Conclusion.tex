The work done for this thesis was primarily motivated to find avenues to optimize resource usage for GRID I/O operations. 
The toy model testing allowed us to create branches with data similar compression ratios to real and simulated data, allowing to investigate the hypothesis that modifying the basket buffer limit had an effect on disk and memory usage.
It led to the conclusion that, upon investigating with real data and real MC simulation, that there might be an avenue to look at both ROOT and Athena to limit basket sizes. 

% Recap what Chapter 4 said into 1-2 sentences
% What was the conclusion to the toymodel/basket-size modifications to the derivation jobs? 
Modifying the basket buffer sizes at the Athena level shows there was a balance was struck when using the Athena basket buffer size limited to 128 kB between memory-usage and the size of the DAOD to be saved long-term. 
Removing the basket buffer size limit, the $5.5 \%$ saving in PHYSLITE MC disk-usage at the expense of an $11 \%$ increase in memory-usage could be a trade-off worth making in some scenarios. 
A class of potentially unoptimized AOD branches in MC simulated data was also brought to light during this study.
The leading indicator to potential optimization is the highly compressible nature of these branches post-derivation.
Further work could be done to look into these AOD branches to identify areas where further work can be done to reduce the overall AOD footprint. 

% Recap Chapter 5 into 1-2 sentences. 
The xAOD EDM comes with a number of new additions to bring about optimization the future of analysis work at the ATLAS experiment.
Integrating the new features into a few comprehensive unit tests allow for the nightly CI builds to catch any issues that break core I/O functionality as it pertains to the xAOD EDM, which has not been done before.
These new unit-tests exercise reading and writing select decorations ontop of the already existing data structures attached to an example object called \verb|ExampleElectron|. 

