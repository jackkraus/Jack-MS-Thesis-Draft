% Recap what Chapter 3 into 1-2 sentences
The toy model testing allowed us to create branches with data similar compression ratios to real and simulated data, allowing to investigate the hypothesis that modifying the basket buffer limit had an effect on disk and memory usage.
It led to the conclusion that, upon investigating with real data and real MC simulation, that there might be an avenue to look at both ROOT and Athena to limit basket sizes. 
% Recap what Chapter 4 said into 1-2 sentences
% What was the conclusion to the toymodel/basket-size modifications to the derivation jobs? 
Modifying the basket buffer sizes at the Athena level shows there was a balance struck 

This study also illuminated the possibilty at a class of unoptimized branches in MC simulated data, from which it was not clear 

% Recap Chapter 5 into 1-2 sentences. 
The xAOD EDM comes with a number of new additions to bring about optimization the future of analysis work at the ATLAS experiment.
Integrating the new features into a few comprehensive unit tests allow for the nightly CI builds to catch any issues that break core I/O functionality as it pertains to the xAOD EDM, which has not been done before.
These new unit-tests exercise reading and writing select decorations ontop of the already existing data structures attacted to an example object called \verb|ExampleElectron|.  