\section{Current Derivation Framework}
Derivation production jobs suffer from high memory usage, and DAODs make up a bulk of disk-space usage. 
DAODs are used in physics analyses and ought to be optimized to alleviate stress on the GRID and to lower disk-space usage. 
Optimizing both disk-space and memory usage is a tricky balance as they are typically at odds with one another. 
For example, increasing memory output memory buffers results in lower disk-space usage due to better compression but the memory usage will increase since one will have to load a larger buffer into memory. 
The route we opted to take is by optimizing for disk-space and memory by testing various basket limits and viewing the effects of the branches on both data and Monte Carlo (MC) simulated analysis object data (AODs)

\section{Performance Metrics and Benchmarking}
% RSS/PSS for memory usage and output file size for disk usage

Our initial focus was on the inclusion of a minimum number of entries per buffer and the maximum basket buffer limit.
As we'll see in the following section, we then opted to keep the minimum number of entries set to its default setting (10 entries per buffer). 


For both the nightly and the release testing, the data derivation job comes from a 2022 dataset with four input files 160327 events. 
The MC job comes from a 2023 $t\bar{t}$ standard sample simulation job with six input files with 140k events. 
The specific datasets for both are noted in Appendix \ref{sec: deriv job dataset}.

The corresponding input files for both data and MC jobs were ran with various configurations of Athena (version 24.0.16) and its specified basket buffer limit. 
The four configurations tested all kept minimum 10 entries per basket and modified the basket limitation in the following ways: 

\begin{enumerate}
    \item ``$\textit{default}$" - Athena's default setting, and basket limit of $128\times1024$ bytes
    \item ``$\textit{no-lim}$'' - Removing the Athena basket limit, the ROOT imposed 1.3 MB limit still remains
    \item ``$\textit{256k}$'' - Limit basket buffer to $256\times1024$ bytes
    \item ``$\textit{512k}$'' - Limit basket buffer to $512\times1024$ bytes
\end{enumerate}

Interesting results come from the comparison of "no-lim" and "default" configuration. The "256k" and "512k" configurations were included for completeness and provided to be a helpful sanity check throughout. 
Building and running these configurations of Athena are illustrated in a GitHub repository. \cite{Kraus}

\section{Results}
\subsection{Presence of basket-cap and presence of minimum number of entries}

First batch testing was for data and MC simulation derivation production jobs with and without presence of an upper limit to the basket size and presence of the minimum number of entires per branch. PHYSLITE MC derivation production, from Table 2, sees a 9.9\% increase in output file size when compared to the default Athena configuration. Since this configuration only differs by the elimination of the "min-number-entries" we assume the minimum number of entries per branch should be kept at 10 and left alone. Table \ref{Tab: 24.0.16-data-outputfs} also shows the potential for a PHYSLITE MC DAOD output file size reduction by eliminating our upper basket buffer limit altogether.  


% Data file sizes 
% Athena version: Nightly from v22.0.16 
% input file: # TODO
% PHYS/PHYSLITE

\begin{table}[!ht]
    \centering
    \resizebox{\textwidth}{!}{
        \begin{tabular}{|c|c|c|c|}
            \hline
                Presence of features (Data) & Max PSS (MB) ($\Delta$\% default) & PHYS outFS (GB) ($\Delta$\%) & PHYSLITE outFS (GB) ($\Delta$\%) \\ \hline
                basket-cap, min-num-entries (default)  & 27.1 ( + 0.0 \%) & 3.22 ( + 0.0 \%) & 1.03 ( + 0.0 \%) \\ \hline
                \st{basket-cap} \st{min-num-entries} & 27.8 ( + 2.5 \%) & 3.22 ( + 0.2 \%) & 1.04 ( + 0.2 \%) \\ \hline
                \st{basket-cap} min-num-entries & 27.8  ( + 2.5 \%) & 3.22 ( - 0.0 \%) & 1.03 ( - 0.4 \%) \\ \hline
                basket-cap,  \st{min-num-entries} & 27.3 ( + 0.7 \%) & 3.22 ( + 0.2 \%) & 1.04 ( + 0.7 \%) \\ \hline
            \end{tabular}
    }
    \label{Tab: night-data-outputfs}
    \caption{Comparing the maximum proportional set size (PSS) and PHYS/PHYSLITE output file sizes (outFS) for data jobs while varying the presence of features in Athena PoolWriteConfig.py for 160327 entries.}
\end{table}

% Data file sizes 
% Athena version: Nightly v22.0.16 
% input file: # TODO
% PHYS/PHYSLITE: PHYSLITE

% \begin{table}[!ht]
%     \centering
%     \resizebox{\textwidth}{!}{
%         \begin{tabular}{|l|l|l|l|l|l|}
%             \hline
%                 Athena configurations - PHYS-LITE (data) - File sizes & Max PSS (MB)  & \% of MaxPSS from default & nentries & Output File Size (GB) & \% of output from default \\ \hline
%                 (default) With basket-cap and min-num-entries & 27.109 & 0.00\% & 160327 & 1.034 & 0.00\% \\ \hline
%                 Without both basket-cap and min-num-entries & 27.813 & 2.53\% & 160327 & 1.036 & 0.21\% \\ \hline
%                 No basket-cap but with min-num-entries & 27.814 & 2.53\% & 160327 & 1.030 & -0.39\% \\ \hline
%                 With basket-cap but without min-num-entries & 27.298 & 0.69\% & 160327 & 1.042 & 0.71\% \\ \hline
%             \end{tabular}
%     }
%     \caption{Athena v22.0.16: Comparing the difference in PHYSLITE derivation jobs and their Athena configurations and results of output file size}
%     \label{Tab: night-data-physlite-outputfs}
% \end{table}

% -------------------------------------------------------------------------------------------------------------------------------



% mc file sizes 
% Athena version: Nightly v22.0.16 
% input file: # TODO
% PHYS/PHYSLITE: PHYS
\begin{center}
    \begin{table}[!ht]
        \centering
        \resizebox{\textwidth}{!}{
            \begin{tabular}{|c|c|c|c|}
                \hline
                Presence of features (MC) & Max PSS (MB) ($\Delta$\% default) & PHYS outFS (GB) ($\Delta$\%) & PHYSLITE outFS (GB) ($\Delta$\%) \\ \hline
                basket-cap, min-num-entries (default)  & 14.1 ( + 0.0 \%) & 5.8 ( + 0.0 \%) & 2.6 ( + 0.0 \%) \\ \hline
                \st{basket-cap} \st{min-num-entries} & 16.1 ( + 12.1 \%) & 6.0 ( + 2.9 \%) & 2.7 ( + 5.1 \%) \\ \hline
                \st{basket-cap} min-num-entries & 16.0  ( + 11.5 \%) & 5.7 ( - 2.8 \%) & 2.5 ( - 5.6 \%) \\ \hline
                basket-cap,  \st{min-num-entries} & 14.2 ( + 0.4 \%) & 6.2 ( + 5.4 \%) & 2.9 ( + 9.9 \%) \\ \hline
                \end{tabular}
        }
        \caption{Comparing the maximum proportional set size (PSS) and PHYS/PHYSLITE output file sizes (outFS) for MC jobs while varying the presence of features in Athena PoolWriteConfig.py for 140000 entries.}
        \label{Tab: night-mc-outputfs}
    \end{table}
\end{center}

% mc file sizes 
% Athena version: Nightly v22.0.16 
% input file: # TODO
% PHYS/PHYSLITE: PHYS

% \begin{center}
%     \begin{table}[!ht]
%         \centering
%         \resizebox{\textwidth}{!}{
%             \begin{tabular}{|l|l|l|l|l|l|}
%                 \hline
%                     Athena configurations - PHYS-LITE (mc) - File sizes & Max PSS (MB) & \% of MaxPSS from default & nentries & Output File Size (GB) & \% of outputfs from default \\ \hline
%                     (default) With basket-cap and min-num-entries & 14.13 & 0.00\% & 160327 & 2.59 & 0.00\% \\ \hline
%                     Without both basket-cap and min-num-entries & 16.08 & 12.13\% & 160327 & 2.72 & 5.06\% \\ \hline
%                     No basket-cap but with min-num-entries & 15.97 & 11.51\% & 160327 & 2.45 & -5.58\% \\ \hline
%                     With basket-cap but without min-num-entries & 14.19 & 0.42\% & 160327 & 2.87 & 9.90\% \\ \hline
%                 \end{tabular}
%         }
%         \caption{Athena v22.0.16: Comparing the difference in PHYSLITE derivation jobs and their Athena configurations and results of output file size}
%         \label{Tab: night-mc-physlite-outputfs}
%     \end{table}
% \end{center}


\subsection{Comparing different basket sizes}

Pre-existing derivation jobs were ran for data and MC simulations to compare between configurations of differing basket sizes limits. The results for this set of testing are found from Table 3 through Table 10. The following tables are the DAOD output-file sizes of the various Athena configurations for PHYS/PHYSLITE over their respective data/MC AOD input files. 

% data-24.0.16 file sizes
% Athena version 24.0.16
% input file: # TODO
% PHYS/PHYSLITE 

\begin{table}[!ht]
    \centering
    \resizebox{\textwidth}{!}{
    \begin{tabular}{|c|c|c|c|}
    \hline
        Athena Configs (Data) & Max PSS (MB) ($\Delta$\% \text{default}) & PHYS outFS (GB) ($\Delta$\% ) & PHYSLITE outFS (GB) ($\Delta$\% ) \\ \hline
        (default)  & 27.9 ( + 0.0 \%) & 3.3 ( + 0.0 \%) & 1.0 ( + 0.0 \%) \\ \hline
        256k\_basket & 28.2  ( + 1.3 \%) & 3.3 ( - 0.1 \%) & 1.0 ( - 0.3 \%) \\ \hline
        512k\_basket & 28.5 ( + 2.2 \%) & 3.3 ( + 0.0 \%) & 1.0 ( - 0.3 \%) \\ \hline
        1.3 MB (ROOT MAX) & 28.6 ( + 2.7 \%) & 3.3 ( - 0.1 \%) & 1.0 ( - 0.3 \%) \\ \hline
    \end{tabular}
    }
    \caption{Comparing the maximum proportional set size (PSS) and PHYS/PHYSLITE output file sizes (outFS) for Data jobs over various Athena configurations for 160327 entries.}
    \label{Tab: 24.0.16-data-outputfs}
\end{table}

% data-24.0.16 file sizes
% Athena version 24.0.16
% input file: # TODO
% PHYS/PHYSLITE: PHYSLITE


% \begin{table}[!ht]
%     \centering
%     \resizebox{\textwidth}{!}{
%         \begin{tabular}{|l|l|l|l|l|l|}
%             \hline
%                 Athena configurations - PHYS-LITE (data) - File sizes & Max PSS (MB) & \% of MaxPSS from default & nentries & Output File Size (GB) & \% of outputfs from default \\ \hline
%                 data-test-athena-default & 27.8591 & 0.00\% & 160327 & 1.0334 & 0.00\% \\ \hline
%                 data-test-athena-no-limit & 28.6432 & 2.74\% & 160327 & 1.0302 & -0.31\% \\ \hline
%                 data-test-athena-256k-basket & 28.2166 & 1.27\% & 160327 & 1.0303 & -0.30\% \\ \hline
%                 data-test-athena-512k-basket & 28.4852 & 2.20\% & 160327 & 1.0307 & -0.26\% \\ \hline
%             \end{tabular}
%     }
%     \caption{Athena v24.0.16 Data PHYSLITE output file size}
%     \label{Tab: 24.0.16-data-physlite-outputfs}
% \end{table}


\subsection{Monte Carlo PHYSLITE branch comparison}

Derivation production jobs work with initially large, memory-consuming branches, compressing them to a reduced size. These derivation jobs are memory intensive because they first have to load the uncompressed branches into readily-accessed memory. Once they're loaded, only then are they able to be compressed. The compression factor is the ratio of pre-derivation branch size (Total-file-size) to post-derivation branch size (Compressed-file-size). The compressed file size is the size of the branch that is permanently saved into the DAOD.  

Branches with highly repetitive data are better compressed than non-repetitive data, leading to high compression factors--the initial size of the branch contains more data than it needs pre-derivation. If pre-derivation branches are larger than necessary, there should be an opportunity to save memory usage during the derivation job. 

The following tables look into some highly compressible branches and might lead to areas where simulation might save some space. (AOD pre compression?)

[Table 5 from the int note]

An immediate observation: with the omission of the Athena basket limit (solely relying on ROOTs 1.3 MB basket limit), the compression factor increases. This is inline with the original expectation that an increased buffer size limit correlate to better compression. *PrimaryVerticesAuxDyn.trackParticleLinks* is a branch where, among each configuration of Athena MC derivation, has the highest compression factor of any branch in this dataset. 

Some branches, like *HLTNav Summary DAODSlimmedAuxDyn.linkColNames* show highly compressible behavior and are consistent with the other job configurations (data, MC, PHYS, and PHYSLITE). Further work could investigate these branches for further optimization 

\subsection{Conclusion}

Initially, limiting the basket buffer size looked appealing; after 128kB basket buffer size the compression ratio would begin to plateau, increasing the memory-usage without saving much in disk-usage. The optimal balance could be the basket buffer limit of 128 kB. 

Instead, by removing the upper limit of the basket size, a greater decrease in DAOD output file size is achieved. The largest decrease in file size came from the PHYSLITE MC derivation jobs without setting an upper limit to the basket buffer size. While similar decreases in file size appear for derivation jobs using data, it is not as apparent for data as it is for MC jobs. With the removal of an upper-limit to the basket size, ATLAS stands to gain a 5\% decrease for PHYSLITE MC DAOD output file sizes, but an 11 - 12% increase in memory usage could prove a heavy burden (See Tables 2 and 4)

By looking at the branches per configuration, specifically in MC PHYSLITE output DAOD, highly compressible branches emerge. The branches inside the MC PHYSLITE DAOD are suboptimal as they do not conserve disk space; instead, they consume memory inefficiently. As seen from (Table 5) through (Table 10), we have plenty of branches in MC PHYSLITE that are seemingly empty--as indicated by the compression factor being O(10). Reviewing and optimizing the branch data could further reduce GRID load during DAOD production by reducing the increased memory-usage while keeping the effects of decreased disk-space. 
