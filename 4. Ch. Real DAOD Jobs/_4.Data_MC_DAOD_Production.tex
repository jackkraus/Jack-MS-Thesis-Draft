Derivation production demands high memory usage, and DAODs make up a bulk of disk-space usage. 
DAODs are used in physics analyses and ought to be optimized to alleviate stress on the GRID and to lower disk-space usage. 
Optimizing both disk-space and memory usage is a tricky balance as they are typically at odds with one another. 
For example, increasing memory output memory buffers results in lower disk-space usage due to better compression but the memory usage will increase since the user will load a larger buffer into memory. 
This project opted to take is by optimizing for disk-space and memory by testing various basket limits and viewing the effects of the branches on both data and Monte Carlo (MC) simulated analysis object data (AODs).

\section{Basket-size Configuration}
\label{sec:DAODProd_Analysis}
% RSS/PSS for memory usage and output file size for disk usage

As the toy model ruled out, the focus here was on optimizing Athena and not ROOTs contribution for optimization.
The initial focus was on the inclusion of a minimum number of entries per buffer and the maximum basket buffer limit.
The AthenaPOOL script directly involved with these buffer settings is the \verb|PoolWriteConfig.py| found in the path \verb|athena/Database/AthenaPOOL/AthenaPoolCnvSvc/python/|.
As discussed in Section $\S$\ref{sec:DAODProd_Results}, further testing opted to keep the minimum number of entries set to its default setting, 10 entries per buffer. 

Throughout the duration of this testing, the results of compression or file size are independent of any changes to the release or the nightly version of Athena.  
The data derivation job comes from a 2022 dataset with four input files and 160,327 events. 
The MC job comes from a 2023 $t\bar{t}$ standard sample simulation job with six input files and 140,000 events. 
The datasets are noted in Appendix \ref{app: deriv job dataset}.

\subsection{Derivation Job Command}
To run a derivation job, AODs need to be downloaded by data-management service, such as Rucio, to a users local machine. 
Rucio is the data-management solution used for this project to procure the various AOD input files used for the derivation jobs.
The machine running the Rucio client will need to have a valid proxy added for Rucio to run correctly.  
A sample command would look like:
\begin{lstlisting}[language=bash]
rucio download data22_13p6TeV:AOD.31407809._000898.pool.root.1
\end{lstlisting}
This downloads the AOD file from Rucio and saves it to the user's local directory.

The command used by Athena to run a derivation job takes the form of the following example: 
\begin{lstlisting}[language=bash]
ATHENA_CORE_NUMBER=4 Derivation_tf.py \
--CA True \
--inputAODFile mc23_13p6TeV.601229.PhPy8EG_A14_ttbar_hdamp258p75_SingleLep.merge.AOD.e8514_e8528_s4162_s4114_r14622_r14663/AOD.33799166._001224.pool.root.1 \
--outputDAODFile art.pool.root \
--formats PHYSLITE \
--maxEvents 2000 \
--sharedWriter True \
--multiprocess True ;  
\end{lstlisting}
Where Athena allows one to specify the number of cores to use with the \verb|ATHENA_CORE_NUMBER| environment variable.
\verb|Derivation_tf.py| is a script that runs the derivation job and is part of the Athena release.
The \verb|--inputAODFile| is the input file for the derivation job, in this case an AOD file.
The user can specify multiple input files at a time by enclosing the input files in quotes and separating each file with a comma, like the following:
\begin{lstlisting}[language=bash]
    --inputAODFile="AOD.A.pool.root.1,AOD.B.pool.root.1,AOD.C.pool.root.1,AOD.D.pool.root.1"
\end{lstlisting}
The \verb|--outputDAODFile| is the output file for the derivation job, in this case a DAOD file.
The \verb|--formats PHYSLITE| flag allows the job to use the PHYSLITE format for the DAOD.
Here is where the user may choose to include PHYS or PHYSLITE simply by inlusion of one or both.
The \verb|--maxEvents| flag allows one to specify the maximum number of events to run the job on.
The \verb|--sharedWriter True| flag allows the job to utilize SharedWriter. 
The \verb|--multiprocess True| flag allows the job to use AthenaMP tools.




The input files for both data and MC jobs were ran with various configurations of Athena by modifying the basket buffer limit. 
The four configurations tested all kept minimum number of basket buffer entries at 10 and modified the basket limitation in the following ways: 

\begin{enumerate}
    \item ``$\textit{default}$" - Athena's default setting, and basket limit of $128\times1024$ bytes
    \item ``$\textit{256k}$'' - Limit basket buffer to $256\times1024$ bytes
    \item ``$\textit{512k}$'' - Limit basket buffer to $512\times1024$ bytes
    \item ``$\textit{no-lim}$'' - Removing the Athena basket limit, the ROOT imposed 1.3 MB limit still remains
\end{enumerate}

% ====== 
% RESULTS
% ======

\section{Results}
\label{sec:DAODProd_Results}

\subsection{Presence of basket-cap and presence of minimum number of entries}
\label{sec:DAODProd_Results_presence}

First batch testing was for data and MC simulation derivation production jobs with and without presence of an upper limit to the basket size and presence of the minimum number of basket buffer entries. 
PHYSLITE MC derivation production, from Table \ref{Tab: night-mc-outputfs}, sees a 9.9\% increase in output file size when compared to the default Athena configuration. 
Since this configuration only differs by the omission of the ``min-number-entries" requirement, we assume the minimum number of basket buffer entries should be kept at 10 and left alone. 
Table \ref{Tab: night-mc-outputfs} also shows the potential for a PHYSLITE MC DAOD output file size reduction by eliminating our upper basket buffer limit altogether.  


% Data file sizes 
% Athena version: Nightly from v22.0.16 
% input file: # TODO
% PHYS/PHYSLITE

\begin{table}[!ht]
    \centering
    \resizebox{\textwidth}{!}{
        \begin{tabular}{|c|c|c|c|}
            \hline
                Presence of features (Data) & Max PSS (MB) ($\Delta$\% default) & PHYS outFS (GB) ($\Delta$\%) & PHYSLITE outFS (GB) ($\Delta$\%) \\ \hline
                basket-cap, min-num-entries (default)  & 27.1 ( + 0.0 \%) & 3.22 ( + 0.0 \%) & 1.03 ( + 0.0 \%) \\ \hline
                \st{basket-cap} \st{min-num-entries} & 27.8 ( + 2.5 \%) & 3.22 ( + 0.2 \%) & 1.04 ( + 0.2 \%) \\ \hline
                \st{basket-cap} min-num-entries & 27.8  ( + 2.5 \%) & 3.22 ( - 0.0 \%) & 1.03 ( - 0.4 \%) \\ \hline
                basket-cap,  \st{min-num-entries} & 27.3 ( + 0.7 \%) & 3.22 ( + 0.2 \%) & 1.04 ( + 0.7 \%) \\ \hline
            \end{tabular}
    }
    \label{Tab: night-data-outputfs}
    \caption{Comparing the maximum proportional set size (PSS) and PHYS/PHYSLITE output file sizes (outFS) for data jobs while varying the presence of features in Athena PoolWriteConfig.py for 160327 entries.}
\end{table}

% Data file sizes 
% Athena version: Nightly v22.0.16 
% input file: # TODO
% PHYS/PHYSLITE: PHYSLITE

% \begin{table}[!ht]
%     \centering
%     \resizebox{\textwidth}{!}{
%         \begin{tabular}{|l|l|l|l|l|l|}
%             \hline
%                 Athena configurations - PHYS-LITE (data) - File sizes & Max PSS (MB)  & \% of MaxPSS from default & nentries & Output File Size (GB) & \% of output from default \\ \hline
%                 (default) With basket-cap and min-num-entries & 27.109 & 0.00\% & 160327 & 1.034 & 0.00\% \\ \hline
%                 Without both basket-cap and min-num-entries & 27.813 & 2.53\% & 160327 & 1.036 & 0.21\% \\ \hline
%                 No basket-cap but with min-num-entries & 27.814 & 2.53\% & 160327 & 1.030 & -0.39\% \\ \hline
%                 With basket-cap but without min-num-entries & 27.298 & 0.69\% & 160327 & 1.042 & 0.71\% \\ \hline
%             \end{tabular}
%     }
%     \caption{Athena v22.0.16: Comparing the difference in PHYSLITE derivation jobs and their Athena configurations and results of output file size}
%     \label{Tab: night-data-physlite-outputfs}
% \end{table}

% -------------------------------------------------------------------------------------------------------------------------------



% mc file sizes 
% Athena version: Nightly v22.0.16 
% input file: # TODO
% PHYS/PHYSLITE: PHYS
\begin{center}
    \begin{table}[!ht]
        \centering
        \resizebox{\textwidth}{!}{
            \begin{tabular}{|c|c|c|c|}
                \hline
                Presence of features (MC) & Max PSS (MB) ($\Delta$\% default) & PHYS outFS (GB) ($\Delta$\%) & PHYSLITE outFS (GB) ($\Delta$\%) \\ \hline
                basket-cap, min-num-entries (default)  & 14.1 ( + 0.0 \%) & 5.8 ( + 0.0 \%) & 2.6 ( + 0.0 \%) \\ \hline
                \st{basket-cap} \st{min-num-entries} & 16.1 ( + 12.1 \%) & 6.0 ( + 2.9 \%) & 2.7 ( + 5.1 \%) \\ \hline
                \st{basket-cap} min-num-entries & 16.0  ( + 11.5 \%) & 5.7 ( - 2.8 \%) & 2.5 ( - 5.6 \%) \\ \hline
                basket-cap,  \st{min-num-entries} & 14.2 ( + 0.4 \%) & 6.2 ( + 5.4 \%) & 2.9 ( + 9.9 \%) \\ \hline
                \end{tabular}
        }
        \caption{Comparing the maximum proportional set size (PSS) and PHYS/PHYSLITE output file sizes (outFS) for MC jobs while varying the presence of features in Athena PoolWriteConfig.py for 140000 entries.}
        \label{Tab: night-mc-outputfs}
    \end{table}
\end{center}

% mc file sizes 
% Athena version: Nightly v22.0.16 
% input file: # TODO
% PHYS/PHYSLITE: PHYS

% \begin{center}
%     \begin{table}[!ht]
%         \centering
%         \resizebox{\textwidth}{!}{
%             \begin{tabular}{|l|l|l|l|l|l|}
%                 \hline
%                     Athena configurations - PHYS-LITE (mc) - File sizes & Max PSS (MB) & \% of MaxPSS from default & nentries & Output File Size (GB) & \% of outputfs from default \\ \hline
%                     (default) With basket-cap and min-num-entries & 14.13 & 0.00\% & 160327 & 2.59 & 0.00\% \\ \hline
%                     Without both basket-cap and min-num-entries & 16.08 & 12.13\% & 160327 & 2.72 & 5.06\% \\ \hline
%                     No basket-cap but with min-num-entries & 15.97 & 11.51\% & 160327 & 2.45 & -5.58\% \\ \hline
%                     With basket-cap but without min-num-entries & 14.19 & 0.42\% & 160327 & 2.87 & 9.90\% \\ \hline
%                 \end{tabular}
%         }
%         \caption{Athena v22.0.16: Comparing the difference in PHYSLITE derivation jobs and their Athena configurations and results of output file size}
%         \label{Tab: night-mc-physlite-outputfs}
%     \end{table}
% \end{center}


\subsection{Comparing different basket sizes}
\label{sec:DAODProd_Results_comparing}

Pre-existing derivation jobs were ran for data and MC simulations to compare between configurations of differing basket sizes limits. 
The results for this set of testing are found from Table \ref{Tab: 24.0.16-data-outputfs} through Table \ref{Tab: 24.0.16-MC-PHYSLITE-no-lim-top-compfs}. 
The following tables are the DAOD output-file sizes of the various Athena configurations for PHYS/PHYSLITE over their respective data/MC AOD input files. 

% data-24.0.16 file sizes
% Athena version 24.0.16
% input file: # TODO
% PHYS/PHYSLITE 

\begin{table}[!ht]
    \centering
    \resizebox{\textwidth}{!}{
    \begin{tabular}{|c|c|c|c|}
    \hline
        Athena Configs (Data) & Max PSS (MB) ($\Delta$\% \text{default}) & PHYS outFS (GB) ($\Delta$\% ) & PHYSLITE outFS (GB) ($\Delta$\% ) \\ \hline
        (default)  & 27.9 ( + 0.0 \%) & 3.3 ( + 0.0 \%) & 1.0 ( + 0.0 \%) \\ \hline
        256k\_basket & 28.2  ( + 1.3 \%) & 3.3 ( - 0.1 \%) & 1.0 ( - 0.3 \%) \\ \hline
        512k\_basket & 28.5 ( + 2.2 \%) & 3.3 ( + 0.0 \%) & 1.0 ( - 0.3 \%) \\ \hline
        1.3 MB (ROOT MAX) & 28.6 ( + 2.7 \%) & 3.3 ( - 0.1 \%) & 1.0 ( - 0.3 \%) \\ \hline
    \end{tabular}
    }
    \caption{Comparing the maximum proportional set size (PSS) and PHYS/PHYSLITE output file sizes (outFS) for Data jobs over various Athena configurations for 160327 entries.}
    \label{Tab: 24.0.16-data-outputfs}
\end{table}

% data-24.0.16 file sizes
% Athena version 24.0.16
% input file: # TODO
% PHYS/PHYSLITE: PHYSLITE


% \begin{table}[!ht]
%     \centering
%     \resizebox{\textwidth}{!}{
%         \begin{tabular}{|l|l|l|l|l|l|}
%             \hline
%                 Athena configurations - PHYS-LITE (data) - File sizes & Max PSS (MB) & \% of MaxPSS from default & nentries & Output File Size (GB) & \% of outputfs from default \\ \hline
%                 data-test-athena-default & 27.8591 & 0.00\% & 160327 & 1.0334 & 0.00\% \\ \hline
%                 data-test-athena-no-limit & 28.6432 & 2.74\% & 160327 & 1.0302 & -0.31\% \\ \hline
%                 data-test-athena-256k-basket & 28.2166 & 1.27\% & 160327 & 1.0303 & -0.30\% \\ \hline
%                 data-test-athena-512k-basket & 28.4852 & 2.20\% & 160327 & 1.0307 & -0.26\% \\ \hline
%             \end{tabular}
%     }
%     \caption{Athena v24.0.16 Data PHYSLITE output file size}
%     \label{Tab: 24.0.16-data-physlite-outputfs}
% \end{table}

% Label:     \label{Tab: 24.0.16-data-outputfs}

% mc-24.0.16 file sizes
% Athena version 24.0.16
% input file: # TODO
% PHYS/PHYSLITE 


\begin{table}[!ht]
    \centering
    \resizebox{\textwidth}{!}{
        \begin{tabular}{|c|c|c|c|}
            \hline
            Athena configurations (Data) & Max PSS (MB) ($\Delta$\% \text{default}) & PHYS outFS (GB) ($\Delta$\% ) & PHYSLITE outFS (GB) ($\Delta$\% ) \\ \hline
            (default) & 15.00 ( + 0.00 \%) & 5.88 ( + 0.00 \%) & 2.59 ( + 0.00 \%) \\ \hline
            no\_limit & 16.90 ( + 11.27 \%) & 5.72 ( - 2.80 \%) & 2.45 ( - 5.55 \%) \\ \hline
            256k\_basket & 15.28  ( + 1.87 \%) & 5.80 ( - 1.35 \%) & 2.51 ( - 3.11 \%) \\ \hline
            512k\_basket & 16.41 ( + 8.60 \%) & 5.74 ( - 2.46 \%) & 2.46 ( - 5.11 \%) \\ \hline
            \end{tabular}
    }
    \caption{Athena v24.0.16: Comparing the maximum proportional set size (PSS) and PHYS/PHYSLITE output file sizes (outFS) for MC jobs over various Athena configurations for 140000 entries.}
    \label{Tab: 24.0.16-mc-outputfs}
\end{table}

% mc-24.0.16 file sizes
% Athena version 24.0.16
% input file: # TODO
% PHYS/PHYSLITE: PHYSLITE


% \begin{table}[!ht]
%     \centering
%     \resizebox{\textwidth}{!}{
%         \begin{tabular}{|l|l|l|l|l|l|}
%             \hline
%                 Athena configurations - PHYS-LITE (MC) - File sizes & Max PSS (MB) & \% of MaxPSS from default & nentries & Output File Size (GB) & \% of outputfs from default \\ \hline
%                 mc-test-athena-default & 15.00 & 0.00\% & 140000 & 2.59 & 0.00\% \\ \hline
%                 mc-test-athena-no-limit & 16.90 & 11.27\% & 140000 & 2.45 & -5.55\% \\ \hline
%                 mc-test-athena-256k-basket & 15.28 & 1.87\% & 140000 & 2.51 & -3.11\% \\ \hline
%                 mc-test-athena-512k-basket & 16.41 & 8.60\% & 140000 & 2.46 & -5.11\% \\ \hline
%             \end{tabular}
%     }
%      \caption{Athena v24.0.16 MC PHYSLITE output file size}
%     \label{Tab: 24.0.16-mc-physlite-outputfs}
% \end{table}

% Label:     \label{Tab: 24.0.16-mc-outputfs}


``Max PSS" refers to the maximum proportional set size, which is the maximum memory usage of the job.
Table \ref{Tab: 24.0.16-data-outputfs} tells us that with this $t\bar{t}$ data sample, there are marginal changes in both the memory usage for the job and the output file size of the DAODs. 
Whereas Table \ref{Tab: 24.0.16-mc-outputfs} shows a much more drastic change, with a 5.6\% reduction in output file size for the MC PHYSLITE DAOD when compared to the default Athena configuration.
While there's a 5.6\% reduction in output file size for the MC PHYSLITE DAOD, there's also a 11.3\% increase in memory usage. 


\subsection{Monte Carlo PHYSLITE branch comparison}
\label{sec:DAODProd_Results_Monte}

Derivation production jobs work with initially large, memory-consuming branches, compressing them to a reduced size. 
These derivation jobs are memory intensive because they first have to load the uncompressed branches into readily-accessed memory. 
Once they're loaded, only then are they able to be compressed. 
The compression factor is the ratio of pre-derivation branch size (Total-file-size) to post-derivation branch size (Compressed-file-size). 
The compressed file size is the size of the branch that is permanently saved into the DAOD.  

Branches with highly repetitive data are better compressed than non-repetitive data, leading to high compression factors--the initial size of the branch contains more data than it needs pre-derivation. 
If pre-derivation branches are larger than necessary, there should be an opportunity to save memory usage during the derivation job. 

The following tables look into some highly compressible branches and might lead to areas where simulation might save some space. (AOD pre compression?)

% MC-24.0.16 Top branches sorted by CF
% Athena version 24.0.16, default configuration
% input file: # TODO
% PHYS/PHYSLITE: PHYSLITE

\begin{table}[!ht]
    \centering
    \vspace{10px}
    \caption{Top 10 branches sorted by compression factor, MC PHYSLITE [Athena v24.0.16 default configuration.]}
    \label{Tab: 24.0.16-MC-PHYSLITE-default-top-CF}
    \resizebox{\textwidth}{!}{
        \begin{tabular}{|l|c|c|c|c|}
    \hline
        Athena v24.0.16 (default) MC branch & Branch size (kB) & Total-file-size (MB) & Compressed-file-size (MB)  & Compression factor \\ \hline
        PrimaryVerticesAuxDyn.trackParticleLinks & 128 & 2146.2 & 24.0 & 89.4 \\ \hline
        HardScatterVerticesAuxDyn.incomingParticleLinks & 128 & 118.5 & 1.7 & 71.6 \\ \hline
        HLTNav\_Summary\_DAODSlimmedAuxDyn.linkColNames & 128 & 784.0 & 11.9 & 65.7 \\ \hline
        HardScatterVerticesAuxDyn.outgoingParticleLinks & 128 & 108.6 & 1.9 & 58.7 \\ \hline
        TruthBosonsWithDecayVerticesAuxDyn.incomingParticleLinks & 96 & 31.6 & 0.7 & 43.5 \\ \hline
        HLTNav\_Summary\_DAODSlimmedAuxDyn.linkColClids & 128 & 390.6 & 10.7 & 36.6 \\ \hline
        AnalysisTauJetsAuxDyn.tauTrackLinks & 128 & 75.0 & 2.0 & 36.6 \\ \hline
        HLTNav\_Summary\_DAODSlimmedAuxDyn.linkColKeys & 128 & 390.6 & 11.7 & 33.4 \\ \hline
        AnalysisJetsAuxDyn.GhostTrack & 128 & 413.8 & 13.1 & 31.5 \\ \hline
        TruthBosonsWithDecayVerticesAuxDyn.outgoingParticleLinks & 83.5 & 27.3 & 0.9 & 31.0 \\ \hline
    \end{tabular}
    }
    \vspace{10px}
\end{table}

% MC-24.0.16 Top branches sorted by CF
% Athena version 24.0.16, no-lim configuration
% input file: # TODO
% PHYS/PHYSLITE: PHYSLITE

\begin{table}[!ht]
    \centering
    % \vspace{10px} % already 10 px from above
    \caption{Top 10 branches sorted by compression factor, MC PHYSLITE [Athena v24.0.16  without limit to the basket buffer.]}
    \label{Tab: 24.0.16-MC-PHYSLITE-no-lim-top-CF}
    \resizebox{\textwidth}{!}{
       \begin{tabular}{|l|c|c|c|c|}
    \hline
        Athena v24.0.16 (no-lim) MC branch & Branch size (kB) & Total-file-size (MB) & Compressed-file-size (MB)  & Compression factor \\ \hline
        PrimaryVerticesAuxDyn.trackParticleLinks & 1293.5 & 2145.5 & 22.9 & 93.5 \\ \hline
        HardScatterVerticesAuxDyn.incomingParticleLinks & 693.0 & 118.5 & 1.3 & 90.1 \\ \hline
        HardScatterVerticesAuxDyn.outgoingParticleLinks & 635.5 & 108.5 & 1.5 & 74.0 \\ \hline
        HLTNav\_Summary\_DAODSlimmedAuxDyn.linkColNames & 1293.5 & 783.5 & 11.9 & 65.8 \\ \hline
        TruthBosonsWithDecayVerticesAuxDyn.incomingParticleLinks & 96.0 & 31.6 & 0.7 & 43.5 \\ \hline
        AnalysisTauJetsAuxDyn.tauTrackLinks & 447.0 & 74.9 & 1.9 & 39.2 \\ \hline
        HLTNav\_Summary\_DAODSlimmedAuxDyn.linkColClids & 1293.5 & 390.3 & 11.0 & 35.5 \\ \hline
        HLTNav\_Summary\_DAODSlimmedAuxDyn.linkColKeys & 1293.5 & 390.3 & 11.3 & 34.5 \\ \hline
        AnalysisJetsAuxDyn.GhostTrack & 1293.5 & 413.5 & 13.0 & 31.9 \\ \hline
        TruthBosonsWithDecayVerticesAuxDyn.outgoingParticleLinks & 83.5 & 27.3 & 0.9 & 31.0 \\ \hline
    \end{tabular}
    }
\end{table}

% MC-24.0.16 Top branches sorted by totalfs
% Athena version 24.0.16, default configuration
% input file:
% PHYS/PHYSLITE: PHYSLITE

\begin{table}[!ht]
    \centering
    \resizebox{\textwidth}{!}{
        \begin{tabular}{|l|c|c|c|c|}
    \hline
        Athena v24.0.16 (default) MC branch & Branch size (kB) & Total-file-size (MB) & Compressed-file-size (MB)  & Compression factor \\ \hline
        PrimaryVerticesAuxDyn.trackParticleLinks & 128 & 2146.2 & 24.0 & 89.4 \\ \hline
        HLTNav\_Summary\_DAODSlimmedAuxDyn.linkColNames & 128 & 784.0 & 11.9 & 65.7 \\ \hline
        AnalysisJetsAuxDyn.GhostTrack & 128 & 413.8 & 13.1 & 31.5 \\ \hline
        HLTNav\_Summary\_DAODSlimmedAuxDyn.linkColClids & 128 & 390.6 & 10.7 & 36.6 \\ \hline
        HLTNav\_Summary\_DAODSlimmedAuxDyn.linkColKeys & 128 & 390.6 & 11.7 & 33.4 \\ \hline
        AnalysisJetsAuxDyn.SumPtChargedPFOPt500 & 128 & 148.9 & 7.3 & 20.5 \\ \hline
        AnalysisJetsAuxDyn.NumTrkPt1000 & 128 & 148.8 & 8.7 & 17.2 \\ \hline
        AnalysisJetsAuxDyn.NumTrkPt500 & 128 & 148.8 & 11.9 & 12.5 \\ \hline
        HardScatterVerticesAuxDyn.incomingParticleLinks & 128 & 118.5 & 1.7 & 71.6 \\ \hline
        AnalysisLargeRJetsAuxDyn.constituentLinks & 128 & 111.5 & 7.1 & 15.8 \\ \hline
    \end{tabular}
    }
    \caption{Top 10 branches sorted by total file size in bytes, MC PHYSLITE [Athena v24.0.16 default configuration.]}
    \label{Tab: 24.0.16-MC-PHYSLITE-default-top-totalfs}
\end{table}

% MC-24.0.16 Top branches sorted by totalfs
% Athena version 24.0.16, no-lim configuration
% input file: 
% PHYS/PHYSLITE: PHYSLITE

\begin{table}[!ht]
    \centering
    \resizebox{\textwidth}{!}{
         \begin{tabular}{|l|c|c|c|c|}
    \hline
        Athena v24.0.16 (no-lim) MC branch & Branch size (kB) & Total-file-size (MB) & Compressed-file-size (MB)  & Compression factor \\ \hline
        PrimaryVerticesAuxDyn.trackParticleLinks & 1293.5 & 2145.5 & 22.9 & 93.6 \\ \hline
        HLTNav\_Summary\_DAODSlimmedAuxDyn.linkColNames & 1293.5 & 783.5 & 11.9 & 65.8 \\ \hline
        AnalysisJetsAuxDyn.GhostTrack & 1293.5 & 413.5 & 13.0 & 31.9 \\ \hline
        HLTNav\_Summary\_DAODSlimmedAuxDyn.linkColClids & 1293.5 & 390.3 & 11.0 & 35.5 \\ \hline
        HLTNav\_Summary\_DAODSlimmedAuxDyn.linkColKeys & 1293.5 & 390.3 & 11.3 & 34.5 \\ \hline
        AnalysisJetsAuxDyn.SumPtChargedPFOPt500 & 905.5 & 148.8 & 6.8 & 21.9 \\ \hline
        AnalysisJetsAuxDyn.NumTrkPt1000 & 905 & 148.8 & 8.5 & 17.6 \\ \hline
        AnalysisJetsAuxDyn.NumTrkPt500 & 905 & 148.8 & 11.8 & 12.6 \\ \hline
        HardScatterVerticesAuxDyn.incomingParticleLinks & 693 & 118.5 & 1.3 & 90.2 \\ \hline
        AnalysisLargeRJetsAuxDyn.constituentLinks & 950.5 & 111.4 & 6.4 & 17.4 \\ \hline
    \end{tabular}
    }
    \caption{Top 10 branches sorted by total file size in bytes, MC PHYSLITE [Athena v24.0.16  without limit to the basket buffer.]}
    \label{Tab: 24.0.16-MC-PHYSLITE-no-lim-top-totalfs}
\end{table}

% MC-24.0.16 Top branches sorted by compfs
% Athena version 24.0.16, default configuration
% input file:
% PHYS/PHYSLITE: PHYSLITE

\begin{table}[!ht]
    \centering
    \caption{Top 10 branches sorted by compressed file size in bytes, MC PHYSLITE [Athena v24.0.16 default configuration.]}
    \label{Tab: 24.0.16-MC-PHYSLITE-default-top-compfs}
    \resizebox{\textwidth}{!}{
       \begin{tabular}{|l|c|c|c|c|}
    \hline
        Athena v24.0.16 (default) MC branch & Branch size (kB) & Total-file-size (MB) & Compressed-file-size (MB)  & Compression factor \\ \hline
        PrimaryVerticesAuxDyn.trackParticleLinks & 128 & 2146.2 & 24.0 & 89.4 \\ \hline
        AnalysisJetsAuxDyn.GhostTrack & 128 & 413.8 & 13.1 & 31.5 \\ \hline
        AnalysisJetsAuxDyn.NumTrkPt500 & 128 & 148.8 & 11.9 & 12.5 \\ \hline
        HLTNav\_Summary\_DAODSlimmedAuxDyn.linkColNames & 128 & 784.0 & 11.9 & 65.7 \\ \hline
        HLTNav\_Summary\_DAODSlimmedAuxDyn.linkColKeys & 128 & 390.6 & 11.7 & 33.4 \\ \hline
        HLTNav\_Summary\_DAODSlimmedAuxDyn.linkColClids & 128 & 390.6 & 10.7 & 36.6 \\ \hline
        AnalysisJetsAuxDyn.NumTrkPt1000 & 128 & 148.8 & 8.7 & 17.2 \\ \hline
        AnalysisJetsAuxDyn.SumPtChargedPFOPt500 & 128 & 148.9 & 7.3 & 20.5 \\ \hline
        AnalysisLargeRJetsAuxDyn.constituentLinks & 128 & 111.5 & 7.1 & 15.8 \\ \hline
        HLTNav\_Summary\_DAODSlimmedAuxDyn.name & 128 & 80.8 & 4.4 & 18.4 \\ \hline
    \end{tabular}
    }
    \vspace{10px}
\end{table}

% MC-24.0.16 Top branches sorted by compfs
% Athena version 24.0.16, no-lim configuration
% input file: 
% PHYS/PHYSLITE: PHYSLITE

\begin{table}[!ht]
    \centering
    \caption{Top 10 branches sorted by compressed file size in bytes, MC PHYSLITE [Athena v24.0.16  without limit to the basket buffer.]}
    \label{Tab: 24.0.16-MC-PHYSLITE-no-lim-top-compfs}
    \resizebox{\textwidth}{!}{
        \begin{tabular}{|l|c|c|c|c|}
    \hline
        Athena v24.0.16 (no-lim) MC branch & Branch size (kB) & Total-file-size (MB) & Compressed-file-size (MB)  & Compression factor \\ \hline
        PrimaryVerticesAuxDyn.trackParticleLinks & 1293.5 & 2145.5 & 22.9 & 93.5 \\ \hline
        AnalysisJetsAuxDyn.GhostTrack & 1293.5 & 413.5 & 13.0 & 31.9 \\ \hline
        HLTNav\_Summary\_DAODSlimmedAuxDyn.linkColNames & 1293.5 & 783.5 & 11.9 & 65.8 \\ \hline
        AnalysisJetsAuxDyn.NumTrkPt500 & 905 & 148.8 & 11.8 & 12.6 \\ \hline
        HLTNav\_Summary\_DAODSlimmedAuxDyn.linkColKeys & 1293.5 & 390.3 & 11.3 & 34.5 \\ \hline
        HLTNav\_Summary\_DAODSlimmedAuxDyn.linkColClids & 1293.5 & 390.3 & 11.0 & 35.5 \\ \hline
        AnalysisJetsAuxDyn.NumTrkPt1000 & 905 & 148.8 & 8.5 & 17.6 \\ \hline
        AnalysisJetsAuxDyn.SumPtChargedPFOPt500 & 905.5 & 148.8 & 6.8 & 21.9 \\ \hline
        AnalysisLargeRJetsAuxDyn.constituentLinks & 950.5 & 111.4 & 6.4 & 17.4 \\ \hline
        HLTNav\_Summary\_DAODSlimmedAuxDyn.name & 242 & 80.8 & 4.5 & 18.0 \\ \hline
    \end{tabular}
    }
    \vspace{10px}
\end{table}


An immediate observation: with the omission of the Athena basket limit (solely relying on ROOTs $1.3 MB$ basket limit), the compression factor increases. 
This is inline with the original expectation that an increased buffer size limit correlate to better compression. 
$\textit{PrimaryVerticesAuxDyn.trackParticleLinks}$ is a branch where, among each configuration of Athena MC derivation, has the highest compression factor of any branch in this dataset. 

Some branches, like \textit{HLTNav Summary DAODSlimmedAuxDyn.linkColNames} show highly compressible behavior and are consistent with the other job configurations (data, MC, PHYS, and PHYSLITE). 
Further work could investigate these branches for further optimization of derivation jobs.

\subsection{Conclusion to derivation job optimization}
\label{sec:DAODProd_Results_conclusion}
Initially, limiting the basket buffer size looked appealing; after the 128 kB basket buffer size limit was set, the compression ratio would begin to plateau, increasing the memory-usage without saving much in disk-usage. 
The optimal balance is met with the setting of 128 kB basket buffers for derivation production. 

Instead, by removing the upper limit of the basket size, a greater decrease in DAOD output file size is achieved. 
The largest decrease in file size came from the PHYSLITE MC derivation jobs without setting an upper limit to the basket buffer size. 
While similar decreases in file size appear for derivation jobs using data, it is not as apparent for data as it is for MC jobs. 
With the removal of an upper-limit to the basket size, ATLAS stands to gain a 5\% decrease for PHYSLITE MC DAOD output file sizes, but an $11 - 12$\% increase in memory usage could prove a heavy burden (See Tables 2 and 4).

By looking at the branches per configuration, specifically in MC PHYSLITE output DAOD, highly compressible branches emerge. 
The branches inside the MC PHYSLITE DAOD are suboptimal as they do not conserve disk space; instead, they consume memory inefficiently. 
As seen from (Table 5) through (Table 10), we have plenty of branches in MC PHYSLITE that are seemingly empty--as indicated by the compression factor being $\mathcal{O}(10)$. 
Reviewing and optimizing the branch data could further reduce GRID load during DAOD production by reducing the increased memory-usage while keeping the effects of decreased disk-space. 
