Athena uses a number of unit tests during the development lifecycle to ensure core I/O functionality does not break.
Many of the I/O tests were originally created for the old EDM and haven't been updated to test the xAOD EDMs core I/O functions.
This project took in track information from a unit test using the T/P EDM, writes the data into an example xAOD object to file and reads it back.

\section{CI Unit Tests}
Unit tests are programs that act as a catch during the continuous integration of a codebase and exhaust features that need to remain functional. 
Athena has a number of unit tests which check every merge request and nightly build for issues in the new code that could break core I/O functionality, either at the level of Athena, ROOT, or any other software in the LCG stack.
With the adoption of the xAOD EDM, there were no unit tests to cover core I/O functionality related to this new EDM. 

Specifically there were no unit tests to handle selection of dynamic attributes, or decorations, on xAOD objects created during writing and read back.
To address this, a new xAOD test object needed to be created and written during a new unit test that fit into the existing unit tests.
The list of unit tests that are currently executed during a nightly build can be found in Table \ref{tab:CI_Unit_Tests}.


\begin{table}[h]
    \centering
    % \resizebox{\textwidth}{!}{
    \begin{tabular}{|c|c|}
        \hline
        $\textbf{Unit Test}$ & $\textbf{Employed Algorithms}$ \\
        \hline
        Write & WriteData \\
        \hline
        ReadWrite & ReadData \\
        \hline
        Read & ReadData \\
        \hline
        Copy & None \\
        \hline
        ReadWriteNext & ReadData, ReWriteData \\
        \hline
        WritexAODElectron & ReadData, WriteExampleElectron \\
        \hline
        ReadxAODElectron & ReadExampleElectron \\
        \hline
        ReadAgan & ReadData \\
        \hline
        WriteConcat & WriteData, ReWriteData \\
        \hline
        ReadConcat & ReadData \\
        \hline
        WriteCond & ReadData, WriteCond \\
        \hline
        ReadCond & ReadData, ReadCond \\
        \hline
        WriteMeta & WriteData, WriteCond \\
        \hline
        ReadMeta & ReadData \\
        \hline
    \end{tabular}
    % }
    \caption{List of unit tests in the AthenaPoolExample package that are currently executed during a nightly build.}
    \label{tab:CI_Unit_Tests}
\end{table}


The mechanism for passing a unit test done automatically by building the framework, running the unit tests, and comparing the diff of the output file to the unit test with a reference file associated with that particular unit test. 
If the unit test passes, then the diff will be empty and the unit test will be marked as passing.
Conversely, if the unit test fails, then the diff will be non-empty and the unit test will be marked as failing.

\section{xAOD Test Object}

The object used to employ the new unit test is the xAOD::ExampleElectron object, where the "xAOD::" is a declaration of the namespace and simply identifies the object as an xAOD object.
An individual xAOD::ExampleElectron object only has a few parameters for sake of testing, its transvese momentum, \verb|pt|, and its charge, \verb|charge|.
A collection of xAOD::ExampleElectron objects are stored in the xAOD::ExampleElectronContainer object, which is just a DataVector of xAOD::ExampleElectron objects.
The xAOD EDM permits the use of an AuxStore

\section{Unit Test}

The way the xAOD::ExampleElectron object is accessed in the unit test is by utilizing StoreGate. 

\section{Results}
