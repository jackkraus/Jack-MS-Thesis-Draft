\documentclass[12pt]{niuthesis}
\usepackage[utf8]{inputenc}
\usepackage{graphicx}
\usepackage{amsmath}
\usepackage{physics}
\usepackage{caption}
\usepackage{subcaption}
\usepackage{multirow}
\usepackage{xcolor}
\usepackage{listings}
\usepackage{braket}
\usepackage{hyperref}
\usepackage{pdflscape}
\usepackage{soul}


\usepackage{listings}
\usepackage{xcolor}

% OMIT FOR FINAL PRODUCT
\usepackage{lineno}
\linenumbers

\definecolor{codegreen}{rgb}{0,0.6,0}
\definecolor{codegray}{rgb}{0.5,0.5,0.5}
\definecolor{codepurple}{rgb}{0.58,0,0.82}
\definecolor{backcolour}{rgb}{0.95,0.95,0.92}

\lstdefinestyle{mystyle}{
    backgroundcolor=\color{backcolour},   
    commentstyle=\color{codegreen},
    keywordstyle=\color{magenta},
    numberstyle=\tiny\color{codegray},
    stringstyle=\color{codepurple},
    basicstyle=\ttfamily\footnotesize,
    breakatwhitespace=false,         
    breaklines=true,                 
    captionpos=b,
    frame=single,                    
    keepspaces=true,                 
    numbers=left,                    
    numbersep=5pt,                  
    showspaces=false,                
    showstringspaces=false,
    showtabs=false,                  
    tabsize=2
}

\lstset{style=mystyle}

%%%%%%%%%%%%%%%%%%%%%%%%%%%%%%%%%%%%%%%%%%%%%%%%%%%%%%%%%%%%%%%%%%



\usepackage[hyperref,backend=biber,backref,sorting=none]{biblatex}
\addbibresource{main.bib}




\title{Optimization of derivation jobs and modernization of I/O integration tests for the ATLAS Experiment}
\author{Arthur C. Kraus}
\major{Physics}
\degree{Thesis}{M.S.}{Master of Science}
\degreedate{May}{2025}
\department{Department of Physics}
\director{Dr. Jahred Adelman}


\begin{document}


\begin{abstract}
The High-Luminosity LHC (HL-LHC) is a phase of the LHC that is expected to start toward the end of the decade. 
With this comes an increase in data taken per year that current software and computing infrastructure, including I/O, is being prepared to handle. 
The ATLAS experiment's Software Performance Optimization Team has areas in development to improve the Athena software framework that is scalable in performance and ready for wide-spread HL-LHC era data taking. 
One area of interest is optimization of derivation production jobs by improving derived object data stored to disk by about 4-5\% by eliminating the upper-limit on TTree basket buffers, at the expense of an increase in memory usage by about 11\%. 

Athena and the software it depends on are updated frequently, and to synthesize changes cohesively there are scripts, unit tests, that run which test core I/O functionality. 
This thesis upgrades existing I/O unit tests to now exercise features exclusive to the xAOD Event Data Model (EDM) such as writing and reading object data from the previous EDM using transient and persistent data. 
These new unit tests also include and omit select dynamic attributes to object data during the component accumulator step. 
\end{abstract}

\begin{dedication}
% I'd like to dedicate this thesis to my brother, Joseph, and to my sister, Emily. To my parents, Joyce Kraus D.S.W and Art Kraus, who have sacrificed more than I will ever know. 

To all of the fluffy kitties. To all of the fluffy kitties. To all of the fluffy kitties. To all of
the fluffy kitties. 


\end{dedication}

\begin{acknowledgements}
% Many thanks to Jahred Adelman for bringing this project to me and having boundless patience. I'd like to thank Alaettin Serhan Mete, for helping me realize the learning curve is not as steep as anticipated. Thanks to Peter van Gemmeren for the wisdom imparted every step of the way. 

% I also want to thank the professors I studied under during this degree: Stephen Martin, Bela Erdyli, Andreas Glatz and Hector de la Torre Perez. Each of your patience and wisdom have helped me shape the 

% I have to thank the other students I met at NIU, as it was through their support I was able to rekindle my passion for collaborating with the brightest and kindest people one could want to work alongside.  

Here’s where you acknowledge folks who helped. Here’s where you acknowledge folks who helped. Here’s where you acknowledge folks who helped. Here’s where you acknowledge folks who helped.
\end{acknowledgements}
\MakeThesisPrologue



% 1. Introduction
% -- At most 1-3 pages
% -- no need to talk about everything, just a high
% -- level breakdown 

\chapter{Introduction}
\label{chap:Intro}
Particle physics is the branch of physics that studies the fundamental constituents of matter and the forces governing their interactions.  
The field started as studies in electromagnetism, radiation, and further developed with the discovery of the electron.
What followed was more experiments to search for new particles, new models to describe the results, and new search techniques which demanded more data.
The balance in resources for an experiment bottlenecks how much data can be taken, so steps need to be taken to identify interesting interactions and optimize the storage and processing of this data.
This thesis investigates software performance optimization of the ATLAS experiment at CERN. 
Specifically, ways to modernize and optimize areas of the software framework, Athena, to improve input/output (I/O) performance during derivation production and create new tests that catch when specific core I/O functionality is broken.

\section{LHC and The ATLAS Detector}

\begin{figure}[h]
    \centering
    \includegraphics[width=.8\textwidth]{content/img/LHC illustration.jpg}
    \caption{Illustration of the LHC experiment sites on the France-Switzerland border.\cite{LHC_Illustration}}
    \label{fig:intro_LHC_sites}
\end{figure}

The Large Hadron Collider (LHC), shown in Figure \ref{fig:intro_LHC_sites},  is a particle accelerator spanning a 26.7-kilometer ring that crosses between the France-Switzerland border at a depth between 50 and 175 meters underground.\cite{Brüning:782076}
The ATLAS experiment, shown in Figure \ref{fig:intro_ATLAS_detector}, is the largest LHC general purpose detector, and the largest detector ever made for particle collision experiments. 
The detector lies in a cavern 92.5 m underground at a length of 46 m, height and width of 25 m.\cite{ATLAS_Tech_Proposal}
A quadrant of the detector is shown in Figure \ref{fig:ATLAS_quadrant}, where $\eta$ is a measure of the pseudo-rapidity.
Pseudo-rapidity is a parameter representing the the angle relative to the beamline and is defined as
\begin{equation}
    \eta \equiv -\ln\left[ \tan\left(\frac{\theta}{2}\right)\right],
\end{equation}  
where if $\theta = 0$ then $\eta = \infty$ and if $\theta = \frac{\pi}{2}$ then $\eta = 0$.
Pseudo-rapidity is used, as opposed to traditional Cartesian angles, because it's Lorentz invariant under boosts along the beam axis, making it easier to identify collisions due to symmetry.  
\begin{figure}[h!]
    \centering
    \includegraphics[width=.8\textwidth]{content/img/ATLAS_quadrant.png}
    \caption{One quadrant of the ATLAS detector.\cite{ATLAS_Tech_Proposal}}
    \label{fig:ATLAS_quadrant}
\end{figure}

The ATLAS detector is comprised of three main sections, the inner detector, calorimeters and the muon detector system. 
\begin{figure}[h]
    \centering
    \includegraphics[width=.8\textwidth]{content/img/ATLAS_Detector.jpg}
    \caption{Overview of the ATLAS detectors main components.\cite{ATLAS_Illustration}}
    \label{fig:intro_ATLAS_detector}
\end{figure}
The inner detector measures the direction, momentum and charge of electrically charged particles.
Its main function is to measure the track of the charged particles without destroying the particle itself.
The first point of contact for particles emerging from $pp$-collisions from the center of the ATLAS detector is the pixel detector.\cite{PixelDetector_2008}
It has over 92 million pixels to aid in particle track and vertex reconstruction.
Since the pixels are the first point of contact to the incident particles they have to be radiation hard so the electronics may function without fault.
When a charged particle passes through a pixel sensor it ionizes the one-sided doped-silicon wafer to produce an excited electron will then occupy the conduction band of the semiconductor producing an electron-hole pair, leaving the valence band empty.\cite{KnollRadDetection}
This hole in the valence band together with the excited electron in the conduction band is called an electron-hole pair.
The electron-hole pair is in the presence of an electric field, which will induce drifting of the electron-hole pair, drifting that will generate the electric current to be measured.

Surrounding the pixel detector is the SemiConductor Tracker (SCT), which uses 4,088 modules of 6 million implanted silicon readout strips.\cite{ABDESSELAM2006642}
Both the pixel detector and SCT measure the path particles take, called tracks.
While the pixel detector has measurement precision up to $10 \mu m$ in the $r\phi$ direction and $70 \mu m$ in the $z$-coordinate direction,\cite{Andreazza:1287089} the SCT has resolution up to$17 \mu m$ in the $r\phi$-direction and $580 \mu m$ in the $z$-direction. 

The final layer of the inner detector is the transition radiation tracker (TRT). 
The TRT is made of a collection of tubes made with many layers of different materials with varying indices of refraction.
The TRT's straw walls are made of two $35\mu m$ layers comprised of $6\mu m$ carbon-polymide, $0.20 \mu m$ aluminum, and a $25\mu m$ Kapton film reflected back.\cite{TRT_2008}
The straws are filled with a gas mixture of $70\% \text{Xe} + 27\% \text{CO}_2 + 3\% \text{O}_2$. 
Its measurement precision is around $170 \mu m$. 
Particles with relativistic velocities have higher Lorentz $\gamma$-factors (see Equation \eqref{lorentzGamma}). 
The TRT uses varying materials to discriminate between heavier particles, which have low $\gamma$ and radiate less, and lighter particles, which have higher $\gamma$ and radiate more.\cite{Mindur:2139567}
\begin{equation}\label{lorentzGamma}
    \gamma = \frac{1}{\sqrt{1 - \frac{v^2}{c^2}}}
\end{equation}

There are two main calorimeters for ATLAS, the Liquid Argon (LAr) calorimeter and the Tile Hadronic calorimeter.
The LAr calorimeter surrounds the inner detector and measures the energy deposits of electrons, photons and hadrons (quark bound states, such as baryons $qqq$ and mesons $q\bar{q}$). 
It layers various metals to intercept the incoming particles to produce a shower of lower energy particles. 
The lower energy particles then ionize the liquid argon that fill the barrier in between the metal layers to produce a current that can be read out.
The Tile calorimeter surrounds the LAr calorimeter and is the largest part of the ATLAS detector weighing in around 2900 tons. 
Particles then traverse through the layers of steel and plastic scintillating tiles. 
When a particle hits the steel, a cascade of secondary particles is generated, and the plastic scintillators will produce photons whose current can be measured.

\section{ATLAS Trigger and Data Acquisition (DAQ)}

The LHC produces $pp$-collisions at a rate of 40 MHz, each collision is an ``event". 
% What is the ATLAS Trigger System
The ATLAS Trigger system is responsible for quickly deciding what events are interesting for physics analysis.
The Trigger system is divided into the first- and second-level triggers and when a particle activates a trigger, the trigger makes a decision to tell the Data Acquisition System (DAQ) to save the data produced by the detector. 
The first-level trigger is a hardware trigger that decides, within $2.5 \mu s$ after the event, if it's a good event to put into a storage buffer for the second-level trigger.
The second-level trigger is a software trigger that decides within $200 \mu s$ and uses around 40,000 CPU-cores and analyses the event to decide if it is worth keeping. 
The second-level trigger selects about 1000 events per second to keep and store long-term.\cite{Trigger-DAQ}
The data taken by this Trigger/DAQ system is raw and not yet in a state that is ready for analysis, but it is ready for the reconstruction stage. 

% How is this relevant to the thesis?
The amount of data taken at ATLAS is substantial.
ATLAS sees more than 3.2 PB of raw data each year, each individual event being around 1.6 MB.\cite{ATLAS_Fact_Sheet} 
All of the data produced by LHC experiments, especially ATLAS, has to be sent to the LHC Computing Grid (LCG). 
The increase in data means more resources from the Grid will be needed, so optimization is an essential part of ensuring scalability of the data able to be taken in by the experiment.
Reconstructed AOD are then processed through derivation jobs that reduced AODs from  $\mathcal{O}(1)$ MB per event to $\mathcal{O}(10)$ kB per event, creating Derived AOD (DAOD). 

\section{HL-LHC and Future Needs in Computation}
The High-Luminosity LHC (HL-LHC) is the upgrade to LHC that anticipates more events and more data taken than ever before.
% How does high luminosity affect the number of collisions? 
The goal is to reach a luminosity of $350 fb^{-1}$, which is forecasted to be reached gradually by around 2040.\cite{HL-LHC_Tech_design}
The HL-LHC era is projected to demand anywhere from 6-10 times data stored per year, so any attempt to save on disk storage will help.\cite{ATLAS_HL-LHC_projections}

One area of research to account for this flood of new data is in the development of the ROOT N-Tuple (RNTuple) I/O subsystem, which is a new storage format for high-energy physics data seeking to replace ROOT TTree. 
The RNTuple is a columnar-based storage format that is optimized for data storage and processing.
It's been shown to outperform TTree I/O subsystem and other storage formats in file size (by about 15\%), throughput, and compression, but still has more development before full implementation into the analysis pipeline.\cite{RNTuple_Lopez-Gomez_2023}\cite{RNTuple_Blomer}
Additionally, there's a push to utilize GPUs and other accelerators in conjunction with CPUs to process track reconstruction and AOD derivation.
Also being developed are software framework updates, such as AthenaMT, to make the single-threaded CPU programs multi-thread ready.\cite{AthenaMT_Leggett_2017}



% 2. ATLAS I/O 
% -- 
\chapter{I/O Tools}
\label{chap:IO_Tools}
% CHAPTER START
The Trigger/DAQ system sends and saves data from the detector to a persistent data storage solution.
The data at this stage needs to be reconstructed and consolidated into physics objects, or Analysis Object Data (AOD) files.
Creating AODs from data requires significant computation power and is undertaken by a software framework maintained by ATLAS, Athena.
This chapter will cover important tools and concepts used by ATLAS to run derivation jobs, as well as introduce data structures that represent event information. 

\section{Athena and ROOT}
% \subsection{What is Athena}
Athena is the open-source software framework for the ATLAS experiment.\cite{athena}
It is based off the Gaudi project and uses on other software such as ROOT and other software as part of the LHC Computing Grid (LCG) software stack.\cite{WLCG_Tech_design_report}
The LCG software stack is a set of software frameworks that provide general solutions for the LHC experiment's computing needs. 
It contains on the order of 500 packages, which include binary builders and compilers, language libraries and dependencies, simulation and analysis software, and more.
Athena also provides some in-house based analysis tools as well as tools for specifically ROOT based analysis.

An Athena application relies on $\textit{components}$: Algorithms, Tools, Services and Properties. \cite{Aad:2895022}  
Each component plays a role in executing an Athena application or job, where $\textsc{Python}$ is used.
ATLAS uses $\textsc{Python}$ for job configuration and steering.\footnote{Job transforms are $\textsc{Python}$ scripts that steer Athena production jobs by configuring arguments that would alter low-level behavior of the entire job. }
Specifically, an Algorithm accesses data objects in the event store, as shown with the solid lines in Figure \ref{fig:ATLAS_Athena_Job_flow}, but does not own or provide any data itself.
Algorithms can ``own" Tools, which serve as helpers exclusive to Algorithms or other components that call them.\footnote{``Ownership" here refers to the components' exclusive access or control of a Tool or Service.} 
Services are not as exclusive with its access, as they can be used by other components to provide a service such as Athena-ROOT conversion, random number generators, and others. 
Properties are able to be called at initialization of the job configuration and include flag definitions, input and output file names, and other algorithm specific options.
\verb|ComponentAccumulator| (CA) is a python class that put into Athena production as a way to prevent extra calls of setting flags during configuration. 

\begin{figure}[ht]
  \centering
  \includegraphics[width=0.8\textwidth]{content/img/athena job.png}
  \caption{An Athena application's general structure.\cite{Aad:2895022}}
  \label{fig:ATLAS_Athena_Job_flow}
\end{figure}

An important step throughout the development of Athena is to ensure any new changes to the codebase will not overrule the functionality of core features to the present workflows.
One of the areas needed to be tested before and upon merging of any new changes to Athena is the I/O functionality, or the performance of reading and writing of stored objects within a broader context of various jobs, i.e. reconstruction or derivation.
While CA is a more general mechanism to run any kind of job with Athena, it's within the scope of this thesis where the focus is on testing core I/O functionality of the new event data model. 
An example Athena job configuration is found in Appendix \ref{app:athena-job-config}. 

% What is ROOT
ROOT is an open-source software framework used for high-energy physics analysis at CERN.\cite{ROOT_about} 
It uses C++ objects to save, access, and process data brought in by the various experiments based at the LHC, the ATLAS experiment uses it in conjunction with Athena.
ROOT largely revolves around organization and manipulation of TFiles and TTrees into ROOT files.
A TTree represents a columnar dataset, and the list of columns are called branches. 
The branches have memory buffers that are automatically allocated by ROOT. 
These memory buffers are divided into corresponding baskets, whose size is designated during memory allocation.
More detail on branch baskets are explored in Chapter \ref{chap:Toy_Model} and \ref{chap:DAODProd}.

% % CMake
CMake and Make are open-source software that is used to build Athena, ROOT, and other software.
A sparse build is a way to make changes to an individual package of code without having to recompile the entire framework at once, which saves time and resources. 
A user can create a text file identifying the path to the package modified, and the sparse build for Athena will proceed upon issuing the following commands:
\begin{lstlisting}[language=bash]
    cmake -DATLAS_PACKAGE_FILTER_FILE=../package_filters.txt ../athena/Projects/WorkDir/ 
    make -j
\end{lstlisting}

% How we used it
% % Athena POOL
The POOL framework is part of a larger framework known as the Persistency Framework (PF). \cite{Trentadue_2012}
The PF was developed with the intent to be independent of any individual experiment, and the goal was to address data access requirements of LHC experiments in different ways.
POOL was in charge of C++ object storage, collection of metadata, and file catalogs by using streaming and relational technologies. 
POOL provided highly scalable object serialization to framework evolving PF files. 
It was eventually discontinued by other experiments in favor of a newer persistency mechanism that uses ROOT in a more streamlined way.
ATLAS then became the sole supporter of POOL and integrated it within Athena to support persistent navigation of the ROOT storage layer.
Now, Athena has both the original PF POOL functionality and a separate modern AthenaPool functionality. 
AthenaPool resides in the ATLAS I/O framework and controls ROOT TTree and TBranch properties such as compression and basket buffer sizing.

% What we looked at/How we used it

\subsection{Continuous Integration (CI) and Development}
CI is a software development practice where new code is tested and validated upon each merge to the main branch of a repository. 
Every commit to the main branch is automatically built and tested for specific core features that are required to work with the codebase. 
This helps to ensure that the codebase is working as intended and that any new code is compatible with the existing codebase.

Athena is hosted on GitLab and developed using CI with an instance of Jenkins, called ATLAS Robot, which builds and tests the new changes within a merge request interface.\cite{athena-gitlab}\cite{Jenkins}
ATLAS Robot will then provide a report of the build and test results.
If the build or test fail, ATLAS Robot will provide a report of which steps failed and why.
This allows for early detection of issues before the nightly build is compiled and tested.



\section{TTree Object} \label{section: ATLASIO_TTreeObject}
A TTree is a ROOT object that organizes physically distinct types of event data into TBranches, or just branches.
Event data could range from information about a specific type of interaction, this includes tracks, position of particles at one point in the detector. 

\begin{figure}[ht]
  \centering
  \includegraphics[width=\textwidth]{content/img/branches_in_TTree.png}
  \caption{A snapshot of the TBranches composing a TTree, from a PHYSLITE DAOD}
  \label{fig:Branches_in_TTree}
\end{figure}

Branches hold data into dedicated contiguous memory buffers, and those memory buffers, upon compression, become baskets.
These baskets can have a limited size and a set minimum number of entries. 
The Athena default basket size at present is 128 kB, and the default minimum number of entries is 10. 
% TODO: The REASON we choose these numbers by default is ____ unclear, but the point is these are the relevant parameters that we're investigating.
Prior to this thesis, the original implementation of both the basket size and minimum number of entries had not yet been investigated for avenues of optimization, this is explored in Section \ref{sec:DAODProd_Analysis}.


One function relevant to TTree is \verb|Fill()|. 
\verb|Fill()| will loop over all of the branches in the TTree and compresses the baskets that make up the branch.
This removes the basket from memory as it is then compressed and written to disk.
It makes reading back branches faster as all of the baskets are stored near each other on the same disk region.\cite{ROOT_TTree}

% AutoFlush
\verb|AutoFlush| is a function that tells the \verb|Fill()| function after a designated number of entries of the branch, in this case vectors, to flush all branch buffers from memory and save them to disk. 


\section{Derivation Production Jobs}
 \label{section: ATLASIO_Derivation}

A derivation production job takes AODs, which comes from the reconstruction step at $\mathcal{O}(1 \text{ MB})$ per event, and creates a derived AOD (DAOD) which sits at $\mathcal{O}(10 \text{ kB})$ per event.
Derivation production is a necessary step to make all data accessible for physicists doing analysis as well as reducing the amount of data that needs to be processed.
While derivations are reduced AODs, they often contain additional information useful for analysis, such as jet collections and high-level discriminants.\cite{PHYSLITE_A_new_2024}
The two mainstream output file formats Athena is capable of handling are PHYS and PHYSLITE.  
Figure \ref{fig:IO_tt_PHYS_vs_PHYSLITE} shows the object composition of a PHYS and PHYSLITE $t\bar{t}$ sample. 
\begin{figure}[h]
    \centering
    \begin{subfigure}{.5\textwidth}
        \centering
        \includegraphics[width=\textwidth]{content/img/tt_PHYS.png}
        % \caption{A subfigure}
        \label{fig:IO_tt_PHYS_subA}
      \end{subfigure}%     
      \begin{subfigure}{.5\textwidth}
        \centering
        \includegraphics[width=\textwidth]{content/img/tt_PHYSLITE.png}
        % \caption{B subfigure}
        \label{fig:IO_tt_PHYSLITE_subB}
      \end{subfigure}% 
    \caption{Object composition of a PHYS and PHYSLITE $t\bar{t}$ sample from Run 3.}
    \label{fig:IO_tt_PHYS_vs_PHYSLITE}
\end{figure}
PHYS output files, at 40.0 kB per event, is predominantly made of jet collections, while PHYSLITE, at 16.1 kB per event, has more trigger and track information.
There is ongoing work to reduce the amount of Trigger information in PHYSLITE which would help further reduce the file size saved to disk.
PHYSLITE, being the smaller file of the two, sees the largest effect upon attempts of optimization. 
These jobs can demand heavy resource usage on the GRID, so optimization of the AOD/DAODs for derivation jobs can be vital. 

\begin{figure}[h]
    \centering
    \includegraphics[width=0.8\textwidth]{content/img/catmore-derivation.png}
    \caption{Derivation production from Reconstruction to Final N-Tuple\cite{DAOD_Laycock_2014}}
    \label{fig:IO_derivation_framework}
\end{figure}

The derivation framework is sequence of steps that are performed on the AODs to create the DAODs.
Skimming is the first step in the derivation framework, and it's responsible for removing whole events based on pre-defined criteria.
Thinning is the second step, and it removes whole objects based on pre-defined criteria.
Lastly slimming removes variables from objects uniformly across events. 


\section{Event Data Models} \label{sec:IO_EDM}
% \subsection{What is an Event Data Model}
An Event Data Model (EDM) is a collection of classes and their relationships to each other that provide a representation of an event detected with the goal of making it easier to use and manipulate by developers.
An EDM is how particles and jets are represented in memory, stored to disk, and manipulated in analysis.
It's useful to have an EDM because it brings a commonality to the code, which is useful when developers reside in different groups with various backgrounds.
An EDM allows those developers to more easily debug and communicate issues when they arise.  

\subsection{Transient/Persistent (T/P) EDM}
One of the previous EDM schemas used by ATLAS concerned a dual transient/persistent status of AOD.
With this EDM, the AOD was converted into an ntuple based format called D3PDs. 
While this conversion allowed for fast readability and partial read for efficient analysis in ROOT, it left the files disconnected from the reconstruction tools found in Athena.\cite{Athena_xAOD_design}
When transient data was present in memory, it could have information attached to the object and gain in complexity the more it was used.
Transient data needed to be simplified before it could become persistent into long-term storage (sent to disk). 
ROOT had trouble handling the complex inheritance models that would come up the more developers used this EDM. 
Before the successor to the T/P EDM was created, ATLAS physicists would convert data samples using the full EDM to a simpler one that would be directly readable by ROOT.
This would lead to duplication of data and made it challenging to develop and maintain the analysis tools to be used on both the full EDM and the reduced ones.
Additionally, converting from transient to persistent data was an excessive step which was eventually removed by the adoption of using an EDM that blends the two stages of data together, this was dubbed the xAOD EDM.


\subsection{xAOD EDM}
The xAOD EDM is the successor to the T/P EDM and brings a number of improvements.\cite{Buckley_2015}
This EDM, unlike T/P, is usable both on Athena and ROOT.
It's easier to pick up for analysis and reconstruction. 
The xAOD EDM has the ability to add and remove variables within an \verb|ItemList| at runtime, specified in the CA script, these variables are ``decorations."

The xAOD EDM use two types of objects that handle data, interface objects and payload objects. 
Interfaces act as an interface for the user to access the object but without its stored data. 
This differs from T/P where the user wants to load an object into memory to access the object. 
If the user wanted to delay the loading of data into memory, they could use the interface object to do so. 
The payload object contains the data for the interface object and allocates contiguous blocks of memory. 
Payload classes are often referred to as auxiliary storage. 

The specific data structure used by ATLAS is the ROOT TTree, but the EDM is agnostic to the type of data structure used. 
ATLAS specific libraries are not required to handle files written in the xAOD format since the payload can be read directly from the contiguous allocation of memory, a central tenent of the xAOD EDM.
This allows for the separation of ATLAS specific analysis frameworks and the preferred analysis tool of the user.
More information on how the xAOD EDM is deployed into unit tests in Section \ref{sec:Mod_utests_xAOD_object}.



% 3. Toy Model Branches
% -- 
\chapter{Toy Model Branch Study}
\label{chap:Toy_Model}
A toy model of AOD provides a simple-to-understand representation for how real and Monte Carlo simulated data will react under optimization conditions for derivation production jobs. 
One commonality between both data and MC is the branch data within both is made of a mixture between repeated integer-like data and randomized floating-point data (i.e. data that has both easily and difficult to compress.)
Replicating this mixture of data in a branch would give us an effective model that would resemble how current derivation jobs would act on real and MC simulated data. 
These toy model mixtures would provide an avenue to test opportunities for optimizing the demand on the GRID by first looking at limiting basket sizes and their effects on compression of branches. 


\section{Toy Model Compression}

\subsection{Random Float Branches} \label{sec:toy_compression_random_float_branches}
There were a number of iterations to the toy model, but the first was constructed by filling up a TTree with branches that each have vectors with varying number of random floats written and read back.
This original model had four distinct branches, each with a set number of events (\verb|N=1000|), and each event having a number of entries, vectors with 1, 10, 100, and 1000 floats each.
Specifics of how this was done will now be illustrated.

The script file can be compiled with \verb|gcc| and it requires all of the dependencies that come with \verb|ROOT|. 
To begin this script, there are a number of included \verb|ROOT| and \verb|C++| standard library headers needed to be included. 
\begin{lstlisting}[language=C]
  // C++ Standard Library
  #include <iostream>
  #include <memory>
  #include <ostream>
  #include <vector>
  
  // Necessary ROOT Headers
  #include "TBranch.h"
  #include "TCanvas.h"
  #include "TFile.h"
  #include "TH1.h"
  #include "TRandom.h"
  #include "TStyle.h"
  #include "TTree.h"
  ...
\end{lstlisting}


The following function \verb|VectorTree()| is the main function in this code.
What is needed first is an output file, which will be called \verb|VectorTreeFile.root|, and the name of the tree can simply be \verb|myTree|.
\begin{lstlisting}[language=C]
  void VectorTree() {
    std::unique_ptr<TFile> myFile =
    std::make_unique<TFile>("VectorTreeFile.root", "RECREATE");
    TTree *tree = new TTree("myTree", "myTree");
    ...
  }
\end{lstlisting}

Initializing variables can start with the total number of events (total number of vectors) in each branch, \verb|N|. 
Additionally the branches have a number of floats per vector, this size will need to be defined as \verb|NEntries0|, \verb|NEntries1|, etc.  
The actual vectors that are being stored into each branch need to be defined as well as the temporary placeholder variable for our randomized floats, \verb|vtenX| and \verb|fX| respectively. 
\begin{lstlisting}[language=C]  
  void VectorTree() {
    ...
    const int N = 1e4; // N = 1000
    // Set Number of Entries with 10^# of random floats
    int NEntries0 = 1;
    int NEntries1 = 10;
    int NEntries2 = 100;
    int NEntries3 = 1000;

    // vectors
    std::vector<float> vten0; // 10^0 = 1 entry
    std::vector<float> vten1; // 10^1 = 10 entries
    std::vector<float> vten2; // 10^2 = 100 entries
    std::vector<float> vten3; // 10^3 = 1000 entries

    // variables
    float f0;
    float f1;
    float f2;
    float f3;
    ...
  }
\end{lstlisting}

From here, initialize the branches so each one knows where its vector pair resides in memory.
\begin{lstlisting}[language=C]  
  void VectorTree() {
    ...
    // Initializing branches
    std::cout << "creating branches" << std::endl;
    tree->Branch("branch_of_vectors_size_one", &vten0);
    tree->Branch("branch_of_vectors_size_ten", &vten1);
    tree->Branch("branch_of_vectors_size_hundred", &vten2);
    tree->Branch("branch_of_vectors_size_thousand", &vten3);
    ...
  }
\end{lstlisting}

Following branch initialization comes the event loop where data is generated and emplaced into vectors.

\begin{lstlisting}[language=C]  
  void VectorTree() {
    ...
    // Events Loop
    std::cout << "generating events..." << std::endl;
    for (int j = 0; j < N; j++) {
        // Clearing entries from previous iteration
        vten0.clear();
        vten1.clear();
        vten2.clear();
        vten3.clear();

        // Generating vector elements, filling vectors
        // Fill vten0
        for (int m = 0, m < NEntries0; m++) {
            f0 = gRandom->Rndm() * 10; // Create random float value
            vten0.emplace_back(f0);    // Emplace that float into vector
        }
        // Fill vten1
        for (int n = 0, n < NEntries1; n++) {
            f1 = gRandom->Rndm() * 10;
            vten1.emplace_back(f1);
        }
        // Fill vten2
        for (int a = 0, a < NEntries2; a++) {
            f2 = gRandom->Rndm() * 10;
            vten2.emplace_back(f2);
        }
        // Fill vten3
        for (int b = 0, b < NEntries3; b++) {
            f3 = gRandom->Rndm() * 10;
            vten3.emplace_back(f3);
        }
        tree->Fill(); // Fill our TTree with all the new branches
    }
    // Saving tree and file
    tree->Write();
    ...
  }
\end{lstlisting}
Once the branches were filled, \verb|ROOT| then will loop over each of the branches in the TTree and at regular intervals will remove the baskets from memory, compress, and write the baskets to disk (flushed), as was discussed in Section $\S \ref{section: ATLASIO_TTreeObject}$.

As illustrated, the \verb|TTree| is written to the file which allows for the last steps within this script. 

\begin{lstlisting}[language=C]  
  void VectorTree() {
    ...

    // Look in the tree
    tree->Scan();
    tree->Print();

    myFile->Save();
    myFile->Close();
  }

  int main() {
    VectorTree();
    return 0;
  } 
\end{lstlisting}

Upon reading back the \verb|ROOT| file, the user can view the original size of the file (Total-file-size), the compressed file size (File-size), the ratio between Total-file-size and File-size (Compression Factor), the number of baskets per branch, the basket size, and other information. 
Since the branches had vectors with exclusively random floats, it becomes apparent that the more randomization in the branches the harder it is to compress. 
Filling vectors with entirely random values was believed to yield compression ratios close to real data, but from the results in Figure \ref{fig:toymodel_compF_rndm_vectors} it's clear some changes needed to be made to bring the branches closer to a compression ratio of $\mathcal{O}(5)$.  

\begin{figure}[h]
    \centering
    \includegraphics[width=.8\textwidth]{content/toymodel_content/branch_compfacts_nomix.png}
    \caption{Compression factors of $N=1000$ entries per branch with random-valued vectors of varying size.}
    \label{fig:toymodel_compF_rndm_vectors}
\end{figure}

% \begin{figure}[h]
%     \caption{File size of $N=1000$ entries per branch with random-valued vectors of varying size.}
%     \label{fig:toymodel_filesize_rndm_vectors}
%     \centering
%     \includegraphics[width=.8\textwidth]{content/toymodel_content/branch_fileSize_nomix.png}
% \end{figure}

Figure \ref{fig:toymodel_compF_rndm_vectors} shows compression drop-off as the branches with more randomized floats per vector were present.
This is the leading indication that there needs to be more compressible data within the branches. 

\subsection{Mixed-Random Float Branches}
The branches needed to have some balance between compressible and incompressible data to mimic the compression ratio found in real data.
How this was achieved was by filling each vector with different ratios of random floats and repeating integers, which will now be described in detail.

The first change was increasing the total number of events per branch from \verb|N = 1e4| to \verb|1e5|, or from 1000 to 100,000. 
Mixing of random floats and repeated integer values takes the same script structure as Section $\S$ \ref{sec:toy_compression_random_float_branches} but adjusts the event generation loop.
\begin{lstlisting}[language=C]  
  void VectorTree() {
    ...
    // Events Loop
    for (int j = 0; j < N; j++) {
        // Clearing entries from previous iteration
        vten0.clear();
        vten1.clear();
        vten2.clear();
        vten3.clear();

        // Generating vector elements, filling vectors
        // Generating vten0
        for (int a = 0; a < NEntries0; a++) {
            if (a < (NEntries0 / 2)) {
              f0 = gRandom->Gaus(0, 1) * gRandom->Rndm();
              vten0.emplace_back(f0);
            } else {
              f0 = 1; 
              vten0.emplace_back(f0);
            }
        }

        // Generating vten1
        for (int b = 0; b < NEntries1; b++) {
            if (b < NEntries1 / 2) {
              f1 = gRandom->Rndm() * gRandom->Gaus(0, 1);
              vten1.emplace_back(f1);
            } else {
              f1 = 1;
              vten1.emplace_back(f1);
            }
        }

        // Generating vten2
        for (int c = 0; c < NEntries2; c++) {
            if (c < NEntries2 / 2) {
              f2 = gRandom->Rndm() * gRandom->Gaus(0, 1);
              vten2.emplace_back(f2);
            } else {
              f2 = 1;
              vten2.emplace_back(f2);
            }
        }

        // Generating vten3
        for (int d = 0; d < NEntries3; d++) {
            if (d < NEntries3 / 2) {
              f3 = gRandom->Rndm() * gRandom->Gaus(0, 1);
              vten3.emplace_back(f3);
            } else {
              f3 = 1;
              vten3.emplace_back(f3);
            }
        }
        tree->Fill(); // Fill our TTree with all the new branches
    }
    // Saving tree and file
    tree->Write();
    ...
  }
\end{lstlisting}

As shown in the \verb|if|-statements in lines \verb|14|, \verb|25|, \verb|36| and \verb|47|, if the iterator was less than half of the total number of entries in the branch then that entry had a randomized float put in that spot in the vector, otherwise it would be filled with the integer \verb|1|.
Having a mixture of half random floats and half integer \verb|1| led to the larger branches still seeing poor compression, so a new mixture of 1/4 random data was introduced. 
Even though \verb|N=10e5| had the larger branches closer to the desired compression ratio, testing at \verb|N=10e6| events improves the accuracy of the overall file size to more closely resemble real data.

Figure \ref{fig:toymodel_compF_1e6_mix_random} shows the difference between compression between the two mixtures. 
When the number of events is increased from $N=10^5$ to $N=10^6$, branches with only half of the mixture is random data become larger and the branches with more vectors per entry become more difficult to compress. 
Figure \ref{fig:toymodel_compF_1e5_mix_random} shows a compression ratio hovering around 3 for the larger branches, whereas Figure \ref{fig:toymodel_compF_1e6_mix_random} shows the same branches hovering around 2. 

\begin{figure}[h]
    \centering
    \includegraphics[width=.8\textwidth]{content/toymodel_content/Compression Ratios for (1_2 random) and (1_4 random) branches at (N=1,000,000 events).png}
    \caption{Compression Ratios for ($\frac{1}{2}$ random) and ($\frac{1}{4}$ random) branches at ($N=10^6$ events)}
    \label{fig:toymodel_compF_1e6_mix_random}
\end{figure}

\begin{figure}[h]
    \centering
    \includegraphics[width=.8\textwidth]{content/toymodel_content/Compression Ratios for (1_2 random) and (1_4 random) branches at (N=100,000 events).png}
    \caption{Compression Ratios for ($\frac{1}{2}$ random) and ($\frac{1}{4}$ random) branches at ($N=10^5$ events)}
    \label{fig:toymodel_compF_1e5_mix_random}
\end{figure}

Unlike the mixture of branches having 1/2 random data, the 1/4 mixture does not see the same compression effect, but with this mixture we see a compression ratio that is in-line with real data.
Here is where tuning the basket size can begin to start.

\section{Basket-Size Investigation}
\label{sec: toy-model basket-size investigation}

Investigating how compression is affected by the basket size requires us to change the basket size, refill the branch and read it out.
The lower bound set for the basket size was 1 kB and the upper bound was 16 MB.
The first branch looked at closely was the branch with a thousand vectors with half of them being random floats, see Figure \ref{fig:toymodel_CFvsBranchSize_1/2mixture}.

\begin{figure}[h]
    \centering
    \includegraphics[width=.8\textwidth]{content/toymodel_content/Compression Factor vs. Branch Size (KB).png}
    \caption{Compression Factors vs Branch Size (1/2 Mixture $N=10^6$ events)}
    \label{fig:toymodel_CFvsBranchSize_1/2mixture}
\end{figure}

\begin{figure}[h]
    \centering
    \includegraphics[width=.8\textwidth]{content/toymodel_content/Number of Baskets vs Branch Size.png}
    \caption{Number of Baskets vs Branch Size (1/2 Mixture $N=10^6$ events)}
    \label{fig:toymodel_NumBasketsvsBranchSize_1/2mixture}
\end{figure}

Figure \ref{fig:toymodel_CFvsBranchSize_1/2mixture} and Figure \ref{fig:toymodel_NumBasketsvsBranchSize_1/2mixture} is the first indication that the lower basket sizes are too small to effectively compress the data. 
For the baskets under 16 kB, it is required to have as many baskets as events to effectively store all the data--this will cause problems later on with memory usage so many of these basket sizes can be ignored.

There were more variations in the data that were looked at. 
For instance, looking further into the types of mixtures and how those mixtures would affect compression are shown in Figure \ref{fig:toymodel_328_compF_vs_basketsize}. 
Another instance looked into the same mixtures but decreasing the precision of the floating point values that we used from the standard 32 floating-point precision to 16 and 8 which made compression easier. 

\begin{figure}[h]
    \centering
    \begin{subfigure}{.5\textwidth}
        \centering
        \includegraphics[width=\textwidth]{content/toymodel_content/3.28/1_of_2.png}
        \caption{A subfigure}
        \label{fig:toymodel_328_compF_vs_basketsize_subA}
      \end{subfigure}%     
      \begin{subfigure}{.5\textwidth}
        \centering
        \includegraphics[width=\textwidth]{content/toymodel_content/3.28/1_of_3.png}
        \caption{B subfigure}
        \label{fig:toymodel_328_compF_vs_basketsize_subB}
      \end{subfigure}% 
      \linebreak
      \begin{subfigure}{.5\textwidth}
        \centering
        \includegraphics[width=\textwidth]{content/toymodel_content/3.28/1_of_4.png}
        \caption{C subfigure}
        \label{fig:toymodel_328_compF_vs_basketsize_subC}
      \end{subfigure}% 
      \begin{subfigure}{.5\textwidth}
        \centering
        \includegraphics[width=\textwidth]{content/toymodel_content/3.28/1_of_5.png}
        \caption{D subfigure}
        \label{fig:toymodel_328_compF_vs_basketsize_subD}
      \end{subfigure}% 
    \caption{Varying Mixtures in 8 point precision - Number of Baskets vs Branch Size ($N=10^6$ events)}
    \label{fig:toymodel_328_compF_vs_basketsize}
\end{figure}

\begin{figure}[h]
    \centering
    \begin{subfigure}{.5\textwidth}
        \centering
        \includegraphics[width=\textwidth]{content/toymodel_content/4.18/1_of_2.png}
        \caption{A subfigure}
        \label{fig:toymodel_418_compF_vs_basketsize_subA}
      \end{subfigure}%     
      \begin{subfigure}{.5\textwidth}
        \centering
        \includegraphics[width=\textwidth]{content/toymodel_content/4.18/1_of_3.png}
        \caption{B subfigure}
        \label{fig:toymodel_418_compF_vs_basketsize_subB}
      \end{subfigure}% 
      \linebreak
      \begin{subfigure}{.5\textwidth}
        \centering
        \includegraphics[width=\textwidth]{content/toymodel_content/4.18/1_of_4.png}
        \caption{C subfigure}
        \label{fig:toymodel_418_compF_vs_basketsize_subC}
      \end{subfigure}% 
      \begin{subfigure}{.5\textwidth}
        \centering
        \includegraphics[width=\textwidth]{content/toymodel_content/4.18/1_of_5.png}
        \caption{D subfigure}
        \label{fig:toymodel_418_compF_vs_basketsize_subD}
      \end{subfigure}% 
      \caption{Varying Mixtures in 16 point precision - Number of Baskets vs Branch Size ($N=10^6$ events)}
      \label{fig:toymodel_418_compF_vs_basketsize}
\end{figure}


Each of these sets of tests indicate that after a certain basket size, i.e. 128 kB, there is no significant increase in compression. 
Having an effective compression at 128 kB, it's useful to stick to that basket size to keep memory usage down. 
Knowing that increasing the basket size beyond 128 kB yields diminishing returns, it's worth moving onto the next phase of testing with actual derivation production jobs.



% 4. Real DAOD Jobs 
% -- 
\chapter{Data and Monte Carlo Derivation Production}
\label{chap:DAODProd}
\section{Current Derivation Framework}
Derivation production jobs suffer from high memory usage, and DAODs make up a bulk of disk-space usage. 
DAODs are used in physics analyses and ought to be optimized to alleviate stress on the GRID and to lower disk-space usage. 
Optimizing both disk-space and memory usage is a tricky balance as they are typically at odds with one another. 
For example, increasing memory output memory buffers results in lower disk-space usage due to better compression but the memory usage will increase since one will have to load a larger buffer into memory. 
The route we opted to take is by optimizing for disk-space and memory by testing various basket limits and viewing the effects of the branches on both data and Monte Carlo (MC) simulated analysis object data (AODs)

\section{Performance Metrics and Benchmarking}
% RSS/PSS for memory usage and output file size for disk usage

Our initial focus was on the inclusion of a minimum number of entries per buffer and the maximum basket buffer limit.
As we'll see in the following section, we then opted to keep the minimum number of entries set to its default setting (10 entries per buffer). 


For both the nightly and the release testing, the data derivation job comes from a 2022 dataset with four input files 160327 events. 
The MC job comes from a 2023 $t\bar{t}$ standard sample simulation job with six input files with 140k events. 
The specific datasets for both are noted in Appendix \ref{sec: deriv job dataset}. 

The corresponding input files for both data and MC jobs were ran with various configurations of Athena (version 24.0.16) and its specified basket buffer limit. 
The four configurations tested all kept minimum 10 entries per basket and modified the basket limitation in the following ways: 

\begin{enumerate}
    \item ``$\textit{default}$" - Athena's default setting, and basket limit of $128\times1024$ bytes
    \item ``$\textit{no-lim}$'' - Removing the Athena basket limit, the ROOT imposed 1.3 MB limit still remains
    \item ``$\textit{256k}$'' - Limit basket buffer to $256\times1024$ bytes
    \item ``$\textit{512k}$'' - Limit basket buffer to $512\times1024$ bytes
\end{enumerate}

Interesting results come from the comparison of "no-lim" and "default" configuration. The "256k" and "512k" configurations were included for completeness and provided to be a helpful sanity check throughout. 
Building and running these configurations of Athena are illustrated in a GitHub repository. \ref{kraus}

\section{Results}
\subsection{Presence of basket-cap and presence of minimum number of entries}

First batch testing was for data and MC simulation derivation production jobs with and without presence of an upper limit to the basket size and presence of the minimum number of entires per branch. PHYSLITE MC derivation production, from Table 2, sees a 9.9\% increase in output file size when compared to the default Athena configuration. Since this configuration only differs by the elimination of the "min-number-entries" we assume the minimum number of entries per branch should be kept at 10 and left alone. Table 2 also shows the potential for a PHYSLITE MC DAOD output file size reduction by eliminating our upper basket buffer limit altogether.  

[Table 1 and 2 from the Int note]

\subsection{Comparing different basket sizes}

Pre-existing derivation jobs were ran for data and MC simulations to compare between configurations of differing basket sizes limits. The results for this set of testing are found from Table 3 through Table 10. The following tables are the DAOD output-file sizes of the various Athena configurations for PHYS/PHYSLITE over their respective data/MC AOD input files. 

\subsection{Monte Carlo PHYSLITE branch comparison}

Derivation production jobs work with initially large, memory-consuming branches, compressing them to a reduced size. These derivation jobs are memory intensive because they first have to load the uncompressed branches into readily-accessed memory. Once they're loaded, only then are they able to be compressed. The compression factor is the ratio of pre-derivation branch size (Total-file-size) to post-derivation branch size (Compressed-file-size). The compressed file size is the size of the branch that is permanently saved into the DAOD.  

Branches with highly repetitive data are better compressed than non-repetitive data, leading to high compression factors--the initial size of the branch contains more data than it needs pre-derivation. If pre-derivation branches are larger than necessary, there should be an opportunity to save memory usage during the derivation job. 

The following tables look into some highly compressible branches and might lead to areas where simulation might save some space. (AOD pre compression?)

[Table 5 from the int note]

An immediate observation: with the omission of the Athena basket limit (solely relying on ROOTs 1.3 MB basket limit), the compression factor increases. This is inline with the original expectation that an increased buffer size limit correlate to better compression. *PrimaryVerticesAuxDyn.trackParticleLinks* is a branch where, among each configuration of Athena MC derivation, has the highest compression factor of any branch in this dataset. 

Some branches, like *HLTNav Summary DAODSlimmedAuxDyn.linkColNames* show highly compressible behavior and are consistent with the other job configurations (data, MC, PHYS, and PHYSLITE). Further work could investigate these branches for further optimization 

\subsection{Conclusion}

Initially, limiting the basket buffer size looked appealing; after 128kB basket buffer size the compression ratio would begin to plateau, increasing the memory-usage without saving much in disk-usage. The optimal balance could be the basket buffer limit of 128 kB. 

Instead, by removing the upper limit of the basket size, a greater decrease in DAOD output file size is achieved. The largest decrease in file size came from the PHYSLITE MC derivation jobs without setting an upper limit to the basket buffer size. While similar decreases in file size appear for derivation jobs using data, it is not as apparent for data as it is for MC jobs. With the removal of an upper-limit to the basket size, ATLAS stands to gain a 5\% decrease for PHYSLITE MC DAOD output file sizes, but an 11 - 12% increase in memory usage could prove a heavy burden (See Tables 2 and 4)

By looking at the branches per configuration, specifically in MC PHYSLITE output DAOD, highly compressible branches emerge. The branches inside the MC PHYSLITE DAOD are suboptimal as they do not conserve disk space; instead, they consume memory inefficiently. As seen from (Table 5) through (Table 10), we have plenty of branches in MC PHYSLITE that are seemingly empty--as indicated by the compression factor being O(10). Reviewing and optimizing the branch data could further reduce GRID load during DAOD production by reducing the increased memory-usage while keeping the effects of decreased disk-space. 


%---
% 5. Modernizing I/O CI Unit-tests
% -- 
\chapter{Modernizing I/O CI Unit-Tests}
\label{chap:Modernize}
\section{Continuous integration unit tests}
Unit tests are programs that act as a catch during the continuous integration of a codebase and exhaust features that need to remain functional. 
Athena has a number of unit tests which check every new merge request and nightly build for issues in the new code that could break core I/O functionality, either at the level of Athena, ROOT, or any other software in the LCG stack.
With the adoption of the xAOD EDM, there were no unit tests to cover core I/O functionality related to this new EDM. 


%---

%---
% 6. Conclusion
% -- 
\chapter{Conclusion}
The work done for this thesis was primarily motivated to find avenues to optimize resource usage for GRID I/O operations. 
The toy model testing allowed us to create branches with data similar compression ratios to real and simulated data, allowing to investigate the hypothesis that modifying the basket buffer limit had an effect on disk and memory usage.
It led to the conclusion that, upon investigating with real data and real MC simulation, that there might be an avenue to look at both ROOT and Athena to limit basket sizes. 

% Recap what Chapter 4 said into 1-2 sentences
% What was the conclusion to the toymodel/basket-size modifications to the derivation jobs? 
Modifying the basket buffer sizes at the Athena level shows there was a balance was struck when using the Athena basket buffer size limited to 128 kB between memory-usage and the size of the DAOD to be saved long-term. 
Removing the basket buffer size limit, the $5.5 \%$ saving in PHYSLITE MC disk-usage at the expense of an $11 \%$ increase in memory-usage could be a trade-off worth making in some scenarios. 
A class of potentially unoptimized AOD branches in MC simulated data was also brought to light during this study.
The leading indicator to potential optimization is the highly compressible nature of these branches post-derivation.
Further work could be done to look into these AOD branches to identify areas where further work can be done to reduce the overall AOD footprint. 

% Recap Chapter 5 into 1-2 sentences. 
The xAOD EDM comes with a number of new additions to bring about optimization the future of analysis work at the ATLAS experiment.
Integrating the new features into a few comprehensive unit tests allow for the nightly CI builds to catch any issues that break core I/O functionality as it pertains to the xAOD EDM, which has not been done before.
These new unit-tests exercise reading and writing select decorations ontop of the already existing data structures attached to an example object called \verb|ExampleElectron|. 


%---



% ===== PRINT BIBLIOGRAPHY =====  
\printbibliography

%-------------------------------------------------------------------------------
\clearpage
\appendix

% \chapter{Toy Model AOD Code}
% \section{Toy Model Code}

This is a part of the code that was run on ROOT to fill up branches with entirely random float values into branches of sizes, 1, 10, 100, and 1000 entries.
\lstinputlisting{3. CH. Toy Model/toy_model_code-02-07-23.cxx}

\chapter{Derivation Production Data}
% \section{Derivation production datasets} 
\label{app: deriv job dataset}


For both the nightly and the release testing, the data derivation job, which comes from the dataset 

\lstinputlisting{content/dataset-data.txt}

was ran with the input files 

\lstinputlisting{content/inputfiles-data.txt}

Similarly, the MC derivation job, comes from the dataset 

\lstinputlisting{content/dataset-mc.txt}

was ran with input files

\lstinputlisting{content/inputfiles-mc.txt}


\end{document}
