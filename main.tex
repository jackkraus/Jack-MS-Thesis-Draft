\documentclass[12pt]{niuthesis}
\usepackage[utf8]{inputenc}
\usepackage{graphicx}
\usepackage{amsmath}
\usepackage{physics}
\usepackage{caption}
\usepackage{subcaption}
\usepackage{multirow}
\usepackage{xcolor}
\usepackage{listings}
\usepackage{braket}
\usepackage{hyperref}
\usepackage{pdflscape}
\usepackage{soul}
\usepackage{setspace}


\usepackage{listings}
\usepackage{xcolor}


% OMIT FOR FINAL PRODUCT
\usepackage{lineno}
%\linenumbers

\definecolor{codegreen}{rgb}{0,0.6,0}
\definecolor{codegray}{rgb}{0.5,0.5,0.5}
\definecolor{codepurple}{rgb}{0.58,0,0.82}
\definecolor{backcolour}{rgb}{0.95,0.95,0.92}

\lstdefinestyle{mystyle}{
    backgroundcolor=\color{backcolour},   
    commentstyle=\color{codegreen},
    keywordstyle=\color{magenta},
    numberstyle=\tiny\color{codegray},
    stringstyle=\color{codepurple},
    basicstyle=\ttfamily\footnotesize,
    breakatwhitespace=false,         
    breaklines=true,                 
    captionpos=b,
    frame=single,                    
    keepspaces=true,                 
    numbers=left,                    
    numbersep=5pt,                  
    showspaces=false,                
    showstringspaces=false,
    showtabs=false,                  
    tabsize=2
}

\lstset{style=mystyle}

%%%%%%%%%%%%%%%%%%%%%%%%%%%%%%%%%%%%%%%%%%%%%%%%%%%%%%%%%%%%%%%%%%



\usepackage[hyperref,backend=biber,backref,sorting=none]{biblatex}
\addbibresource{main.bib}




\title{Optimization of derivation jobs and modernization of I/O integration tests for the ATLAS Experiment}
\author{Arthur C. Kraus}
\major{Physics}
\degree{Thesis}{M.S.}{Master of Science}
\degreedate{May}{2025}
\department{Department of Physics}
\director{Jahred Adelman}


\begin{document}


\begin{abstract}
The High-Luminosity LHC (HL-LHC) is a phase of the LHC that is expected to start toward the end of the decade. 
With this comes an increase in data taken per year that current software and computing infrastructure, including I/O, is being prepared to handle. 
The ATLAS experiment's Software Performance Optimization Team has areas in development to improve the Athena software framework that is scalable in performance and ready for wide-spread HL-LHC era data taking. 
One area of interest is optimization of derivation production jobs by improving derived object data stored to disk by about 4-5\% by eliminating the upper-limit on TTree basket buffers, at the expense of an increase in memory usage by about 11\%. 

%:
Athena and the software it depends on are updated frequently, and to synthesize changes cohesively there are scripts, unit tests, that run which test core I/O functionality. 
This thesis upgrades existing I/O unit tests to now exercise features exclusive to the xAOD Event Data Model (EDM) such as writing and reading object data from the previous EDM using transient and persistent data. 
These new unit tests also include and omit select dynamic attributes to object data during the component accumulator step. 
\end{abstract}

\begin{dedication}
This thesis is dedicated to my family, present and future.

\end{dedication}

\begin{acknowledgements}
% Many thanks to Jahred Adelman for bringing this project to me and having boundless patience. I'd like to thank Alaettin Serhan Mete, for helping me realize the learning curve is not as steep as anticipated. Thanks to Peter van Gemmeren for the wisdom imparted every step of the way. 

% I also want to thank the professors I studied under during this degree: Stephen Martin, Bela Erdyli, Andreas Glatz and Hector de la Torre Perez. Each of your patience and wisdom have helped me shape the 

% I have to thank the other students I met at NIU, as it was through their support I was able to rekindle my passion for collaborating with the brightest and kindest people one could want to work alongside.  

Here’s where you acknowledge folks who helped. Here’s where you acknowledge folks who helped. Here’s where you acknowledge folks who helped. Here’s where you acknowledge folks who helped.
\end{acknowledgements}
\MakeThesisPrologue



% 1. Introduction
% -- At most 1-3 pages
% -- no need to talk about everything, just a high
% -- level breakdown 

\chapter{Introduction}
\label{chap:Intro}
\input{1intro/intro.tex}

% 2. ATLAS I/O 
% -- 
\chapter{I/O Tools}
\label{chap:IO_Tools}
% CHAPTER START
The Trigger/DAQ system sends and saves data from the detector to a persistent data storage solution.
The data at this stage needs to be reconstructed and consolidated into physics objects, or Analysis Object Data (AOD) files.
Creating AODs from data requires significant computation power and is undertaken by a software framework maintained by ATLAS, Athena.
This chapter will cover important tools and concepts used by ATLAS to run derivation jobs, as well as introduce data structures that represent event information. 

\section{Event Data Models} \label{sec:IO_EDM}
% \subsection{What is an Event Data Model}
An Event Data Model (EDM) is a collection of classes and their relationships to each other that provide a representation of an event detected with the goal of making it easier to use and manipulate by developers.
An EDM is how particles and jets are represented in memory, stored to disk, and manipulated in analysis.
It's useful to have an EDM because it brings a commonality to the code, aiding developers who reside in different groups often with various background experience.
An EDM allows those developers to more easily debug and communicate issues when they arise.  

\subsection{Transient/Persistent (T/P) EDM}
ATLAS used an EDM schema for Run-1 which had a separate transient and persistent status of the AOD.
AODs would often be converted to an ``ntuple" based D3DP format allowed for fast readability and partial read for efficient analysis in ROOT, though it left the files disconnected from the reconstruction tools found in Athena.\cite{Athena_xAOD_design}
When transient data was present in memory, it could have information attached to the object and gain in complexity the more it was used.
Transient data needed to be simplified before it could become persistent into long-term storage (sent to disk). 
ROOT had trouble handling the complex inheritance models that would come up the more developers used this EDM. 
Before the successor to the T/P EDM was created, ATLAS physicists would convert data samples using the full EDM to a simpler one that would be directly readable by ROOT.
This would lead to duplication of data and made it challenging to develop and maintain the analysis tools to be used on both the full EDM and the reduced ones.
Additionally, converting from transient to persistent data was an excessive step which was eventually removed by the adoption of using an EDM that blends the two stages of data together, this was dubbed the xAOD EDM.


\subsection{xAOD EDM}
The xAOD EDM is the successor to the T/P EDM and brings a number of improvements.\cite{Buckley_2015}
This EDM, unlike T/P, is usable both on Athena and ROOT.
It's easier to pick up for analysis and reconstruction. 
The xAOD EDM has the ability to add and remove variables within an \verb|ItemList| at runtime, specified in the CA script, these variables are ``decorations."

The xAOD EDM use two types of objects that handle data, interface objects and payload objects. 
Interfaces act as an interface for the user to access the object but without its stored data. 
This differs from T/P where the user wants to load an object into memory to access the object. 
If the user wanted to delay the loading of data into memory, they could use the interface object to do so. 
The payload object contains the data for the interface object and allocates contiguous blocks of memory. 
Payload classes are often referred to as auxiliary storage. 

The specific data structure used by ATLAS is the ROOT TTree, but the EDM is agnostic to the type of data structure used. 
The ROOT TTree is discussed further in the next section.
ATLAS specific libraries are not required to handle files written in the xAOD format since the payload can be read directly from the contiguous allocation of memory, a central tenent of the xAOD EDM.
This allows for the separation of ATLAS specific analysis frameworks and the preferred analysis tool of the user.
More information on how the xAOD EDM is deployed into unit tests in Section \ref{sec:Mod_utests_xAOD_object}.

\section{Athena and ROOT}
% \subsection{What is Athena}
Athena is the open-source software framework for the ATLAS experiment.\cite{athena}
It is based off the Gaudi project and uses ROOT and other software as part of the LHC Computing Grid (LCG) software stack.\cite{WLCG_Tech_design_report}
The LCG software stack is a set of software frameworks that provide general solutions for the LHC experiment's computing needs. 
It contains on the order of 500 packages, which include binary builders and compilers, language libraries and dependencies, simulation and analysis software, and more.
Athena also provides some in-house based analysis tools as well as tools for specifically ROOT based analysis.

An Athena application relies on $\textit{components}$: Algorithms, Tools, Services and Properties. \cite{Aad:2895022}  
Each component plays a role in executing an Athena application or job, where $\textsc{Python}$ is used.
ATLAS uses $\textsc{Python}$ for job configuration and steering.\footnote{Job transforms are $\textsc{Python}$ scripts that steer Athena production jobs by configuring arguments that would alter low-level behavior of the entire job. }
Specifically, an Algorithm accesses data objects in the event store, as shown with the solid lines in Figure \ref{fig:ATLAS_Athena_Job_flow}, but does not own or provide any data itself.
Algorithms can ``own" Tools, which serve as helpers exclusive to Algorithms or other components that call them.\footnote{``Ownership" here refers to the components' exclusive access or control of a Tool or Service.} 
Services are not as exclusive with its access, as they can be used by other components to provide a service such as Athena-ROOT conversion, random number generators, and others. 
Properties are able to be called at initialization of the job configuration and include flag definitions, input and output file names, and other algorithm specific options.
\verb|ComponentAccumulator| (CA) is a python class that put into Athena production as a way to prevent extra calls of setting flags during configuration. 

\begin{figure}[ht]
  \centering
  \includegraphics[width=0.8\textwidth]{content/img/athena job.png}
  \caption{An Athena application's general structure.\cite{Aad:2895022}}
  \label{fig:ATLAS_Athena_Job_flow}
\end{figure}

An important step throughout the development of Athena is to ensure any new changes to the codebase will not overrule the functionality of core features to the present workflows.
One of the areas needed to be tested before and upon merging of any new changes to Athena is the I/O functionality, or the performance of reading and writing of stored objects within a broader context of various jobs, i.e. reconstruction or derivation.
While CA is a more general mechanism to run many kinds of job with Athena, the scope of this thesis is using CA to test core I/O functionality of the new event data model. 
An example Athena job configuration is found in Appendix \ref{app:athena-job-config}. 

ROOT is an open-source software framework used for high-energy physics analysis at CERN.\cite{ROOT_about} 
It uses C++ objects to save, access, and process data brought in by the various experiments based at the LHC, the ATLAS experiment uses it in conjunction with Athena.
ROOT largely revolves around organization and manipulation of TFiles and TTrees into ROOT files.
A TTree represents a columnar dataset, and the list of columns are called branches. 
A TTree is a ROOT object that organizes physically distinct types of event data into TBranches, or just branches.
Event data could range from information about a specific type of interaction, this includes tracks, position of particles at one point in the detector. 

\begin{figure}[ht]
  \centering
  \includegraphics[width=\textwidth]{content/img/branches_in_TTree.png}
  \caption{A snapshot of the TBranches composing a TTree, from a PHYSLITE DAOD}
  \label{fig:Branches_in_TTree}
\end{figure}

One function relevant to TTree is \verb|Fill()|. 
\verb|Fill()| will loop over all of the branches in the TTree and compresses the baskets that make up the branch.\cite{ROOT_TTree}
This initiates the data in memory to start filling a branch's basket buffer (or just ``baskets").
While this first buffer is always unoptimized, it allows opportunity to calculate an optimal basket buffer size.
At regular intervals, dictated either by number of bytes written or by number of entries written, \verb|AutoFlush| will start moving basket buffers from memory and saving them to disk. 
It's this ``flushing" mechanism that allows for easy access to the branch data as each of the baskets will be stored contiguously in memory.
The Athena default basket maximum size at present is 128 kB, and the default minimum number of entries is 10.
The minimum number of entries helps reduce processing on every entry which might be empty, and the maximum basket size is in place to prevent baskets from using too much memory throughout a Grid job.  
Prior to this thesis, the original implementation of both the basket size and minimum number of entries had not yet been fully investigated for avenues of optimization, this is explored in Section \ref{sec:DAODProd_Analysis}.

% % CMake
CMake and Make are open-source software that is used to build Athena, ROOT, and other software.
A sparse build is a way to make changes to an individual package of code without having to recompile the entire framework at once, which saves time and resources. 
A user can create a text file identifying the path to the package modified, and the sparse build for Athena will proceed upon issuing the following commands:
\begin{lstlisting}[language=bash]
    cmake -DATLAS_PACKAGE_FILTER_FILE=../package_filters.txt ../athena/Projects/WorkDir/ 
    make -j
\end{lstlisting}

% How we used it
% % Athena POOL
The POOL framework is part of a larger framework known as the Persistency Framework (PF). \cite{Trentadue_2012}
The PF was developed with the intent to be independent of any individual experiment, and the goal was to address data access requirements of LHC experiments in different ways.
POOL was in charge of C++ object storage, collection of metadata, and file catalogs by using streaming and relational technologies. 
POOL provided highly scalable object serialization to framework evolving PF files. 
It was eventually discontinued by other experiments in favor of a newer persistency mechanism that uses ROOT in a more streamlined way.
ATLAS then became the sole supporter of POOL and integrated it within Athena to support persistent navigation of the ROOT storage layer.
Now, Athena has both the original PF POOL functionality and a separate modern AthenaPool functionality. 
AthenaPool resides in the ATLAS I/O framework and controls ROOT TTree and TBranch properties such as compression and basket buffer sizing.

% What we looked at/How we used it

\subsection{Continuous Integration (CI) and Development}
CI is a software development practice where new code is tested and validated upon each merge to the main branch of a repository. 
Every commit to the main branch is automatically built and tested for specific core features that are required to work with the codebase. 
This helps to ensure that the codebase is working as intended and that any new code is compatible with the existing codebase.

Athena is hosted on GitLab and developed using CI with an instance of Jenkins, called ATLAS Robot, which builds and tests the new changes within a merge request interface.\cite{athena-gitlab}\cite{Jenkins}
ATLAS Robot will then provide a report of the build and test results.
If the build or test fail, ATLAS Robot will provide a report of which steps failed and why.
This allows for early detection of issues before the nightly build is compiled and tested.


\section{Derivation Production Jobs}
 \label{section: ATLASIO_Derivation}

A derivation production job takes AODs, which comes from the reconstruction step at $\mathcal{O}(1 \text{ MB})$ per event, and creates a derived AOD (DAOD) which sits at $\mathcal{O}(10 \text{ kB})$ per event.
Derivation production is a necessary step to make all data accessible for physicists doing analysis as well as reducing the amount of data that needs to be processed.
While derivations are reduced AODs, they often contain additional information useful for analysis, such as jet collections and high-level discriminants.\cite{PHYSLITE_A_new_2024}
The two mainstream output file formats Athena is capable of handling are PHYS and PHYSLITE.  
Figure \ref{fig:IO_tt_PHYS_vs_PHYSLITE} shows the object composition of a PHYS and PHYSLITE $t\bar{t}$ sample. 

% TODO: How does MC differ from Data in terms of composition? 
\begin{figure}[h]
    \centering
    \begin{subfigure}{.5\textwidth}
        \centering
        \includegraphics[width=\textwidth]{content/img/tt_PHYS.png}
        % \caption{A subfigure}
        \label{fig:IO_tt_PHYS_subA}
      \end{subfigure}%     
      \begin{subfigure}{.5\textwidth}
        \centering
        \includegraphics[width=\textwidth]{content/img/tt_PHYSLITE.png}
        % \caption{B subfigure}
        \label{fig:IO_tt_PHYSLITE_subB}
      \end{subfigure}% 
    \caption{Object composition of a PHYS and PHYSLITE $t\bar{t}$ MC simulated sample from Run 3.}
    \label{fig:IO_tt_PHYS_vs_PHYSLITE}
\end{figure}
PHYS output files, at 40.0 kB per event, is predominantly made of jet collections, while PHYSLITE, at 16.1 kB per event, has more trigger and track information.
PHYSLITE, being the smaller file of the two, had a higher concentration of 
% Is it true that PHYSLITE compresses more?  

These jobs can demand heavy resource usage on the GRID, so optimization of the AOD/DAODs for derivation jobs can be vital. 

\begin{figure}[h]
    \centering
    \includegraphics[width=0.8\textwidth]{content/img/catmore-derivation.png}
    \caption{Derivation production from Reconstruction to Final N-Tuple\cite{DAOD_Laycock_2014}}
    \label{fig:IO_derivation_framework}
\end{figure}

The derivation framework is sequence of steps that are performed on the AODs to create the DAODs.
Skimming is the first step in the derivation framework, and is responsible for removing whole events based on pre-defined criteria.
Thinning is the second step, and it removes whole objects based on pre-defined criteria.
Lastly slimming removes variables from objects uniformly across events. 





% 3. Toy Model Branches
% -- 
\chapter{Toy Model Branch Study}
\label{chap:Toy_Model}
Building a toy model for derivation production jobs offers a simplified framework to effectively simulate and analyze the behavior of real and Monte Carlo (MC) data under techniques of optimization aimed to study.
One commonality between both data and MC is the data types stored in branches for both is made of a mixture between repeated integer-like data and randomized floating-point data. 
Integers are easier to compress than floating-point numbers, so adjusting the mixture of each can yield compression ratios closer to real and MC data.  
Replicating this mixture in a branch give us an effective model that resembles how current derivation jobs act on real and MC simulated data. 
These toy model mixtures provide an avenue to test opportunities for optimizing the memory and storage demands of the GRID by first looking at limiting basket sizes and their effects on compression of branches. 


\section{Toy Model Compression}

\subsection{Random Float Branches} \label{sec:toy_compression_random_float_branches}
There were a number of iterations to the toy model, but the first was constructed by filling a TTree with branches that each have vectors with varying number of random floats to write and read.
Vectors are used in this toy model, as opposed to arrays, because vectors are dynamically allocated and deallocated, which allows for more flexibility when synthesizing AODs. 
This original model had four distinct branches, each with a set number of events (\verb|N=1000|), and each event having a number of entries, vectors with 1, 10, 100, and 1000 floats each.

The script can be compiled with \verb|gcc| or \verb|g++| and it requires all of the dependencies that come with ROOT. 
Alternatively, the script can be run directly within ROOT.
% To begin this script, there are a number of included ROOT and \verb|C++| standard library headers. 
% \begin{lstlisting}[language=C]
%   // C++ Standard Library
%   #include <iostream>
%   #include <memory>
%   #include <ostream>
%   #include <vector>
  
%   // Necessary ROOT Headers
%   #include "TBranch.h"
%   #include "TCanvas.h"
%   #include "TFile.h"
%   #include "TH1.h"
%   #include "TRandom.h"
%   #include "TStyle.h"
%   #include "TTree.h"
%   ...
% \end{lstlisting}
The following function \verb|VectorTree()| is the main function in this code.
What is needed first is an output file, which will be called \verb|VectorTreeFile.root|, and the name of the tree can simply be \verb|myTree|.
% \begin{lstlisting}[language=C]
%   void VectorTree() {
%     std::unique_ptr<TFile> myFile =
%     std::make_unique<TFile>("VectorTreeFile.root", "RECREATE");
%     TTree *tree = new TTree("myTree", "myTree");
%     ...
%   }
% \end{lstlisting}
The toy model starts variable initialization with the total number of events in the branch, i.e. the number of times a branch is filled with the specified numbers per vectors, \verb|N|. 
Additionally the branches have a number of floats per vector, this size will need to be defined as \verb|size_vec_0|, \verb|size_vec_1|, etc.  
The actual vectors that are being stored into each branch need to be defined as well as the temporary placeholder variable for our randomized floats, \verb|vec_tenX| and \verb|float_X|, respectively. 
\begin{lstlisting}[language=C]  
  void VectorTree() {
    ...
    const int N = 1e4; // N = 10000, number of events
    // Set size of vectors with 10^# of random floats
    int size_vec_0 = 1;
    int size_vec_1 = 10;
    int size_vec_2 = 100;
    int size_vec_3 = 1000;

    // vectors
    std::vector<float> vec_ten0; // 10^0 = 1 entry
    std::vector<float> vec_ten1; // 10^1 = 10 entries
    std::vector<float> vec_ten2; // 10^2 = 100 entries
    std::vector<float> vec_ten3; // 10^3 = 1000 entries

    // variables
    float float_0;
    float float_1;
    float float_2;
    float float_3;
    ...
  }
\end{lstlisting}

From here, branches are initialized so each one knows where its vector pair resides in memory.
\begin{lstlisting}[language=C]  
  void VectorTree() {
    ...
    // Initializing branches
    std::cout << "creating branches" << std::endl;
    tree->Branch("branch_of_vectors_size_one", &vec_ten0);
    tree->Branch("branch_of_vectors_size_ten", &vec_ten1);
    tree->Branch("branch_of_vectors_size_hundred", &vec_ten2);
    tree->Branch("branch_of_vectors_size_thousand", &vec_ten3);
    ...
  }
\end{lstlisting}

One extra step taken during this phase of testing is the disabling of \verb|AutoFlush|. 
\begin{lstlisting}
  void VectorTree() {
    ...
    tree->SetAutoFlush(0);
    ...
\end{lstlisting}
Disabling \verb|AutoFlush| allows for more consistent compression across the various sizes of branch baskets. 
The toy model needed this consistency more than the later tests as these early tests were solely focused on mimicking data procured by the detector and event simulation. 
The derivation production jobs tested in Chapter $\ref{chap:DAODProd}$ were tested with \verb|AutoFlush| enabled because those tests are not as concerned with compression as they are with memory and disk usage.
Following branch initialization comes the event loop where data is generated and emplaced into vectors.

\begin{lstlisting}[language=C]  
  void VectorTree() {
    ...
    // Events Loop
    std::cout << "generating events..." << std::endl;
    for (int j = 0; j < N; j++) {
        // Clearing entries from previous iteration
        vec_ten0.clear();
        vec_ten1.clear();
        vec_ten2.clear();
        vec_ten3.clear();

        // Generating vector elements, filling vectors
        // Fill vec_ten0 
        // Contents of the vector:
        //    {float_0}
        //    Only one float of random value
        float_0 = gRandom->Rndm() * 10; // Create random float value
        vec_ten0.emplace_back(float_0); // Emplace float into vector
        
        // Fill vec_ten1
        // Contents of the vector:
        //    {float_1_0, ... , float_1_10}
        //    Ten floats, each float is random
        for (int n = 0, n < size_vec_1; n++) {
            float_1 = gRandom->Rndm() * 10;
            vec_ten1.emplace_back(float_1);
        }

        // Do the same with vec_ten2 and vec_ten3, except for 
        //     vectors with size 100 and 1000 respectively. 
        
        // After all branches are filled, fill the TTree with 
        //     new branches
        tree->Fill();
    }
    // Saving tree and file
    tree->Write();
    ...
  }
\end{lstlisting}
Once the branches were filled, ROOT then will loop over each of the branches in the TTree and at regular intervals will remove the baskets from memory, compress, and write the baskets to disk (flushed).

As illustrated, the \verb|TTree| is written to the file which allows for the last steps within this script. 

\begin{lstlisting}[language=C]  
  void VectorTree() {
    ...

    // Look in the tree
    tree->Scan();
    tree->Print();

    myFile->Save();
    myFile->Close();
  }

  int main() {
    VectorTree();
    return 0;
  } 
\end{lstlisting}

Upon reading back the ROOT file, the user can view the original size of the file (Total-file-size), the compressed file size (File-size), the ratio between Total-file-size and File-size (Compression Factor), the number of baskets per branch, the basket size, and other information. 
Filling vectors with entirely random values was believed to yield compression ratios close to real data, but the results in Figure \ref{fig:toymodel_compF_rndm_vectors} show changes needed to be made to bring the branches closer to a compression ratio of $\mathcal{O}(5)$.  
It is evident that branches containing vectors with purely random floats are more difficult to compress due to the high level of randomization.

\begin{figure}[h]
    \centering
    \includegraphics[width=.8\textwidth]{content/toymodel_content/branch_compfacts_nomix.png}
    \caption{Compression factors of $N=1000$ entries per branch with random-valued vectors of varying size.}
    \label{fig:toymodel_compF_rndm_vectors}
\end{figure}

% \begin{figure}[h]
%     \caption{File size of $N=1000$ entries per branch with random-valued vectors of varying size.}
%     \label{fig:toymodel_filesize_rndm_vectors}
%     \centering
%     \includegraphics[width=.8\textwidth]{content/toymodel_content/branch_fileSize_nomix.png}
% \end{figure}

Figure \ref{fig:toymodel_compF_rndm_vectors} shows compression drop-off as the branches with more randomized floats per vector were present.
This is the leading indication that there needs to be more compressible data within the branches. 

\subsection{Mixed-Random Float Branches}
The branches needed to have some balance between compressible and incompressible data to mimic the compression ratio found in real data.
How this was achieved was by filling each vector with different ratios of random floats and repeating integers, which will now be described in detail.

The first change was increasing the total number of events per branch from $N = 10^4$ to $N = 10^5$. 
Mixing of random floats and repeated integer values takes the same script structure as Section \ref{sec:toy_compression_random_float_branches} but adjusts the event generation loop.
\begin{lstlisting}[language=C]  
  void VectorTree() {
    ...
    // Events Loop
    for (int j = 0; j < N; j++) {
        // Clearing entries from previous iteration
        vec_ten0.clear();
        vec_ten1.clear();
        vec_ten2.clear();
        vec_ten3.clear();

        // Generating vector elements, filling vectors
        // Generating vec_ten0
        // Contents of the vector:
        //    {float_0}
        //    Only one float of random value
        // And since there's only one entry, we don't mix the entries. 
        float_0 = gRandom->Gaus(0, 1) * gRandom->Rndm();
        vec_ten0.emplace_back(float_0);
        

        // Generating vec_ten1
        // Contents of the vector:
        //    {float_1_0, float_1_1, float_1_2, float_1_3, float_1_4, 1, 1, 1, 1, 1}
        //    5 floats of random values, 5 integers of value 1.
        for (int b = 0; b < size_vec_1; b++) {
            if (b < size_vec_1 / 2) {
              float_1 = gRandom->Rndm() * gRandom->Gaus(0, 1);
              vec_ten1.emplace_back(float_1);
            } else {
              float_1 = 1;
              vec_ten1.emplace_back(float_1);
            }
        }

        // Do the same with vec_ten2 and vec_ten3, except for 
        //     vectors with size 100 and 1000 respectively. 


        // After all branches are filled, fill the TTree with 
        //     new branches
        tree->Fill(); 
    }
    // Saving tree and file
    tree->Write();
    ...
  }
\end{lstlisting}

As shown in the \verb|if|-statements in lines \verb|14|, \verb|25|, \verb|36| and \verb|47|, if the iterator was less than half of the total number of entries in the vector then that entry had a randomized float put in that spot in the vector, otherwise it would be filled with the integer \verb|1|.
Having a mixture of half random floats and half integer \verb|1| led to the larger branches still seeing poor compression, so a new mixture of 1/4 random data was introduced. 
Even though $N=10^5$ had the larger branches closer to the desired compression ratio, testing at $N=10^6$ events improves the accuracy of the overall file size to more closely resemble real data.

Figure \ref{fig:toymodel_compF_1e6_mix_random} shows the difference between compression between the two mixtures at $N=10^6$ events. 
When the number of events is increased from $N=10^5$ to $N=10^6$, at the 1/2 random-mixture, the branches with more than one entry per vector see their compression factor worsen. 
Figure \ref{fig:toymodel_compF_1e5_mix_random} shows a compression ratio hovering around 3 for the larger branches, whereas Figure \ref{fig:toymodel_compF_1e6_mix_random} shows the same branches hovering around 2. 

\begin{figure}[h]
    \centering
    \includegraphics[width=.8\textwidth]{content/toymodel_content/Compression Ratios for (1_2 random) and (1_4 random) branches at (N=1,000,000 events).png}
    \caption{Compression Ratios for ($\frac{1}{2}$ random) and ($\frac{1}{4}$ random) branches at ($N=10^6$ events)}
    \label{fig:toymodel_compF_1e6_mix_random}
\end{figure}

\begin{figure}[h]
    \centering
    \includegraphics[width=.8\textwidth]{content/toymodel_content/Compression Ratios for (1_2 random) and (1_4 random) branches at (N=100,000 events).png}
    \caption{Compression Ratios for ($\frac{1}{2}$ random) and ($\frac{1}{4}$ random) branches at ($N=10^5$ events)}
    \label{fig:toymodel_compF_1e5_mix_random}
\end{figure}

Unlike the mixture of branches having 1/2 random data, the 1/4 mixture does not see the same compression effect, but with this mixture we see a compression ratio that is in-line with real data.
This is inline with expectation, more repeated integers within the mixture makes the branch more compressible, and the more random floats in the mixture will make the branch more difficult to compress.
With these mixtures added to the toy model, we can start looking at varying the basket sizes to see how they affect compression.

\section{Basket-Size Investigation}
\label{sec: toy-model basket-size investigation}

Investigating how compression is affected by the basket size requires us to change the basket size, refill the branch and read it out.
Changing the basket buffer size was done at the script level with a simple setting after the branch initialization and before the event loop the following code:
\begin{lstlisting}
    int basketSize = 8192000; // 8 MB
    tree->SetBasketSize("*",basketSize);
\end{lstlisting}
This ROOT-level setting was sufficient for the case of the toy model; testing of the basket size setting both at the ROOT- and Athena-level would be done later using derivation production jobs in Section \ref{sec:DAODProd_Analysis}.
The lower bound set for the basket size was 1 kB and the upper bound was 16 MB.
The first branch looked at closely was the branch with a thousand vectors with half of them being random floats, see Figure \ref{fig:toymodel_CFvsBranchSize_1/2mixture}.

\begin{figure}[h]
    \centering
    \includegraphics[width=.8\textwidth]{content/toymodel_content/Compression Factor vs. Branch Size (KB).png}
    \caption{Compression Factors vs Branch Size (1000 entries per vector, 1/2 Mixture $N=10^6$ events)}
    \label{fig:toymodel_CFvsBranchSize_1/2mixture}
\end{figure}

\begin{figure}[h]
    \centering
    \includegraphics[width=.8\textwidth]{content/toymodel_content/Number of Baskets vs Branch Size.png}
    \caption{Number of Baskets vs Branch Size (1000 entries per vector, 1/2 Mixture $N=10^6$ events)}
    \label{fig:toymodel_NumBasketsvsBranchSize_1/2mixture}
\end{figure}

Figures \ref{fig:toymodel_CFvsBranchSize_1/2mixture} and \ref{fig:toymodel_NumBasketsvsBranchSize_1/2mixture} are the first indication that the lower basket sizes are too small to effectively compress the data. 
For baskets smaller than 16 kB, it is necessary to have as many baskets as events to store all the data effectively.
For a mixed-content vector with one thousand entries, containing 500 floats and 500 integers (both are 4 bytes each), its size is approximately 4 kB.
ROOT creates baskets of at least the size of the smallest branch entry, in this case the size of a single vector.
So even though the basket size was set to 1 or 2 kB, ROOT created baskets of 4 kB.
These baskets $\leq 4$kB have a significantly worse compression than the baskets $\geq 4$kB in size, so the focus was shifted toward baskets.  
Once the basket size is larger than the size of a single vector, more than one vector can be stored in a single basket and the total number of baskets is reduced.

There were different types of configuration to the toy model investigated by this study. 
Looking further into the types of mixtures and how they would affect compression are shown in Figures \ref{fig:toymodel_328_compF_vs_basketsize} and \ref{fig:toymodel_418_compF_vs_basketsize}. 
Here the same mixtures were used but the precision of the floating point numbers was decreased from the standard 32 floating-point precision to 16 and 8, making compression easier. 

\begin{figure}[h]
    \centering
    \begin{subfigure}{.5\textwidth}
        \centering
        \includegraphics[width=\textwidth]{content/toymodel_content/3.28/1_of_2.png}
        % \caption{A subfigure}
        \label{fig:toymodel_328_compF_vs_basketsize_subA}
      \end{subfigure}%     
      \begin{subfigure}{.5\textwidth}
        \centering
        \includegraphics[width=\textwidth]{content/toymodel_content/3.28/1_of_3.png}
        % \caption{B subfigure}
        \label{fig:toymodel_328_compF_vs_basketsize_subB}
      \end{subfigure}% 
      \linebreak
      \begin{subfigure}{.5\textwidth}
        \centering
        \includegraphics[width=\textwidth]{content/toymodel_content/3.28/1_of_4.png}
        % \caption{C subfigure}
        \label{fig:toymodel_328_compF_vs_basketsize_subC}
      \end{subfigure}% 
      \begin{subfigure}{.5\textwidth}
        \centering
        \includegraphics[width=\textwidth]{content/toymodel_content/3.28/1_of_5.png}
        % \caption{D subfigure}
        \label{fig:toymodel_328_compF_vs_basketsize_subD}
      \end{subfigure}% 
    \caption{Varying Mixtures in 8 point precision - Number of Baskets vs Branch Size ($N=10^6$ events)}
    \label{fig:toymodel_328_compF_vs_basketsize}
\end{figure}

\begin{figure}[h]
    \centering
    \begin{subfigure}{.5\textwidth}
        \centering
        \includegraphics[width=\textwidth]{content/toymodel_content/4.18/1_of_2.png}
        % \caption{A subfigure}
        \label{fig:toymodel_418_compF_vs_basketsize_subA}
      \end{subfigure}%     
      \begin{subfigure}{.5\textwidth}
        \centering
        \includegraphics[width=\textwidth]{content/toymodel_content/4.18/1_of_3.png}
        % \caption{B subfigure}
        \label{fig:toymodel_418_compF_vs_basketsize_subB}
      \end{subfigure}% 
      \linebreak
      \begin{subfigure}{.5\textwidth}
        \centering
        \includegraphics[width=\textwidth]{content/toymodel_content/4.18/1_of_4.png}
        % \caption{C subfigure}
        \label{fig:toymodel_418_compF_vs_basketsize_subC}
      \end{subfigure}% 
      \begin{subfigure}{.5\textwidth}
        \centering
        \includegraphics[width=\textwidth]{content/toymodel_content/4.18/1_of_5.png}
        % \caption{D subfigure}
        \label{fig:toymodel_418_compF_vs_basketsize_subD}
      \end{subfigure}% 
      \caption{Varying Mixtures in 16 point precision - Number of Baskets vs Branch Size ($N=10^6$ events)}
      \label{fig:toymodel_418_compF_vs_basketsize}
\end{figure}


Each of these sets of tests indicate that after a certain basket size, i.e. 128 kB, there is no significant increase in compression. 
Having an effective compression at 128 kB, it's useful to stick to that basket size to keep memory usage down. 
Knowing that increasing the basket size beyond 128 kB yields diminishing returns, it's worth moving onto the next phase of testing with actual derivation production jobs.



% 4. Real DAOD Jobs 
% -- 
\chapter{Data and Monte Carlo Derivation Production}
\label{chap:DAODProd}
\input{4derivation/derivation.tex}

%---
% 5. Modernizing I/O CI Unit-tests
% -- 
\chapter{Modernizing I/O Unit-Tests}
\label{chap:Modernize}
\input{5unittests/unittests.tex}
%---

%---
% 6. Conclusion
% -- 
\chapter{Conclusion}
\input{6conclusion/conc.tex}
%---



% ===== PRINT BIBLIOGRAPHY =====  

\begingroup
\setstretch{1}
\setlength\bibitemsep{20pt}
\addcontentsline{main.toc}{chapter}{REFERENCES}
\printbibliography[
    heading=bibintoc,
    title={REFERENCES}
]
\endgroup
%-------------------------------------------------------------------------------
\clearpage
\appendix

% \chapter{Toy Model AOD Code}
% \section{Toy Model Code}

This is a part of the code that was run on ROOT to fill up branches with entirely random float values into branches of sizes, 1, 10, 100, and 1000 entries.
\lstinputlisting{3. CH. Toy Model/toy_model_code-02-07-23.cxx}

\chapter{Derivation Production Data}
% \section{Derivation production datasets} 
\label{app: deriv job dataset}


For both the nightly and the release testing, the data derivation job, which comes from the dataset 

\lstinputlisting{content/dataset-data.txt}

was ran with the input files 

\lstinputlisting{content/inputfiles-data.txt}

Similarly, the MC derivation job, comes from the dataset 

\lstinputlisting{content/dataset-mc.txt}

was ran with input files

\lstinputlisting{content/inputfiles-mc.txt}


\chapter{Athena Configuration Job} \label{app:athena-job-config}
\section{Toy Model Code}

This is a part of the code that was run on ROOT to fill up branches with entirely random float values into branches of sizes, 1, 10, 100, and 1000 entries.
\lstinputlisting{3. CH. Toy Model/toy_model_code-02-07-23.cxx}


\end{document}
